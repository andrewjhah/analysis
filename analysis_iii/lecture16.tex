\documentclass[11pt]{article}
\usepackage{master}
\title{MATH 20510 Lec 16}
\author{Andrew Hah}

\begin{document}

\pagestyle{plain}
\begin{center}
{\Large MATH 20510} \\
{\Large Lecture 16} \\
\vspace{.2in}
April 30, 2025
\end{center}

\begin{definition} A \emph{$2$-surface} is a $C^1$ map $\gamma : I^2 \to \bbR^n$.
\end{definition}

\begin{definition} (Informal) A \emph{$2$-form} on $\bbR^n$ is
  \begin{enumerate}
  \item An object which can be integrated over any $2$-surface.
  \item A rule which assigns a real number to every oriented parallelogram in $\bbR^n$ in a ``suitable'' way.
  \end{enumerate}
\end{definition}

Specify an oriented parallelogram in $\bbR^n$ based at $p \in \bbR^n$ by giving $(v, w)$. We want every $2$-form $\omega$ to satisfy the following for every $p \in \bbR^n$
\begin{enumerate}
\item $\omega_p(tv_1, v_2) = \omega_p(v_1, tv_2) = t \omega_p(v_1, v_2)$.
\item $\omega_p(v_1, v_2 + v_3) = \omega_p (v_1, v_2) + \omega_p(v_1, v_3)$ and $\omega_p(v_1 + v_2, v_3) = \omega_p(v_1, v_3) + \omega_p(v_2, v_3)$.
\item $\omega_p(v_1, v_2) = - \omega_p(v_2, v_1)$.
\end{enumerate}

Basic $2$-forms on $\bbR^n$. $\forall v, w \in \bbR^n$,
\begin{enumerate}
\item $(dx_1 \wedge dx_2)(v, w) = \det \begin{pmatrix} v_1 & w_1 \\ v_2 & w_2 \end{pmatrix}$.
\item $(dx_1 \wedge dx_3)(v, w) = \det \begin{pmatrix} v_1 & w_1 \\ v_3 & w_3 \end{pmatrix}$.
\item $(dx_i \wedge dx_j)(v, w) = \det \begin{pmatrix} v_i & w_i \\ v_j & w_j \end{pmatrix}$.
\end{enumerate}

\begin{remark} If $\omega_p$ satisfies $(i) - (iii)$ then $\omega_p$ can be expressed as $$\omega_p = \sum_{i, j} A_{i, j}(p) (dx_i \wedge dx_j)$$ for constant $A_{i, j}$.
\end{remark}

\begin{definition} A \emph{$2$-form} in $\bbR^n$ is a rule assigning a real number to each oriented parallelogram in $\bbR^n$ that can be written as $$\omega = \sum_{i, j} f_{i, j} (dx_i \wedge dx_j)$$ where $f_{i, j}: \bbR^n \to \bbR$ is $C^2$. \\

For any $p \in \bbR^n$, $v, w \in \bbR^n$, $$\omega_p(v, w) = \sum_{i, j} f_{i, j}(p) (dx_i \wedge dx_j)(v, w).$$
\end{definition}

\begin{example} $\omega$ is a $2$-form in $\bbR^2$, \begin{align*} \omega & = f_{1,1} \underbrace{(dx_1 \wedge dx_1)}_{= 0} + f_{1, 2} (dx_1 \wedge dx_2) + f_{2, 1} \underbrace{(dx_2 \wedge dx_1)}_{= - (dx_1 \wedge dx_2)} + f_{2, 2} \underbrace{(dx_2 \wedge dx_2)}_{= 0} \\ & = (f_{1, 2} - f_{2, 1})(dx_1 \wedge dx_2) \end{align*}
  This implies that every $2$-form in $\bbR^2$ can be written as $\omega = f(dx_1 \wedge dx_2)$ where $f$ is $C^2$.
\end{example}

\begin{example} $\omega$ is a $2$-form in $\bbR^3$, \begin{align*} \omega = f_1 (dx_1 \wedge dx_2) + f_2 (dx_1 \wedge dx_3) + f_3 (dx_2 \wedge dx_3). \end{align*}
\end{example}

\begin{definition} Let $\gamma : I^2 \to \bbR^3$ be $C^1$, and $\omega = f_1 (dx_1 \wedge dx_2) + f_2 (dx_1 \wedge dx_3) + f_3 (dx_2 \wedge dx_3)$ be a $2$-form. Then \begin{align*} \int_{\gamma} \omega & = \int_{I^2} \omega_{\gamma(z)} \left( \frac{\partial \gamma}{\partial x_1}(z), \frac{\partial \gamma}{\partial x_2} (z) \right) dz \\ & = \int_{I^2}  f_1(\gamma (z)) (dx_1 \wedge dx_2) \left( \frac{\partial \gamma}{\partial x_1}(z), \frac{\partial \gamma}{\partial x_2} (z) \right) \\ & \qquad + f_2(\gamma(z))(dx_1 \wedge dx_3) \left( \frac{\partial \gamma}{\partial x_1}(z), \frac{\partial \gamma}{\partial x_2} (z) \right) + f_3 (\gamma (z)) (dx_2 \wedge dx_3) \left( \frac{\partial \gamma}{\partial x_1}(z), \frac{\partial \gamma}{\partial x_2} (z) \right) dz \\ & = \int_{I^2} f_1(\gamma(z)) \det \begin{pmatrix} D_1\gamma_1(z) & D_2 \gamma_1 (z) \\ D_1 \gamma_2(z) & D_2 \gamma_2(z) \end{pmatrix}\\ & \qquad +  f_2 (\gamma(z)) \det \begin{pmatrix} D_1\gamma_1(z) & D_2 \gamma_1 (z) \\ D_1 \gamma_3(z) & D_2 \gamma_3(z) \end{pmatrix} +  f_3(\gamma(z)) \det \begin{pmatrix} D_1\gamma_2(z) & D_2 \gamma_2 (z) \\ D_1 \gamma_3(z) & D_2 \gamma_3(z) \end{pmatrix} \end{align*}
\end{definition}

\end{document}
