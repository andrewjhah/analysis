\documentclass[11pt]{article}
\usepackage{master}
\title{MATH 20510. Lec 14}
\author{Andrew Hah}

\begin{document}

\pagestyle{plain}
\begin{center}
{\Large MATH 20510} \\
{\Large Lecture 14} \\
\vspace{.2in}
April 25, 2025
\end{center}

Recall. $f: E \to \bbR$, $E \subseteq \bbR^n$ open, partials $D_1f, \dots, D_nf$. If the partials are themselves differentiable then the second order derivatives of $f$ are defined by $$D_{ij}f = D_iD_jf, \quad (i, j = 1, \dots, n).$$ If these functions are continuous in $E$, we say $f$ is $C^2$ in $E$.

\begin{theorem} If $f \in C^2$ in $E$ then $$D_{ij}f = D_{ji}f, \quad \forall i, j.$$
\end{theorem}

\begin{definition}
  If $f: E \to \bbR^n$, $E \subseteq \bbR^n$ open, $f$ is differentiable at $x \in E$, the determinant of (the linear operator) $f'(x)$ is called the \emph{Jacobian of $f$ at $x$} $$J_f(x) = \mathrm{det} f'(x)$$
\end{definition}

Notation. We may also use $\frac{\partial (y_1, \dots, y_n)}{\partial (x_1, \dots, x_n)}$; $f(x_1, \dots, x_n) = y_1, \dots, y_n$.

\begin{definition} Let $k \in \bbN$. A \emph{$k$-cell} in $\bbR^k$ is the set of points $I^k = \{ x = (x_1, \dots, x_k) \}$ such that $a_i \le x_i \le b_i$, $\forall i = 1, \dots, k$.
\end{definition}

Suppose $I^k$ is a $k$-cell in $\bbR^k$ and $f: I^k \to \bbR$ is continuous. For every $j \le k$, let $I^j$ be the restriction of $I^k$ to the first $j$ components.\\

Define $g_k: I^k \to \bbR$ by $g_k= f$. Define $g_{k - 1}: I^{k - 1} \to \bbR$ by $$g_{k - 1}(x_1, \dots, x_{k-1}) = \int_{a_k}^{b_k}g_k(x_1, \dots, x_k) d x_k$$

Since $g_k$ is uniformly continuous on $I^k$, $g_{k - 1}$ is (uniformly) continuous on $I^{k - 1}$. Define $g_{k - 2}: I^{k - 2} \to \bbR$ by $$g_{k - 2}(x_1, \dots, x_{k - 2}) = \int_{a_{k - 1}}^{b_{k - 1}}g_{k - 1}(x_1, \dots, x_{k - 1}) dx_{k - 1}$$

We can repeat this process, ultimately arriving at a number $$g_0 = \int_{a_1}^{b_1} g_1(x_1) dx_1$$

We say $g_0$ is the integral of $f$ over $I^k$ and we write $$\int_{I^k} f(x) dx = g_0.$$

\begin{example}
  Let $I^2 = [1, 2] \times [0, 1]$, $f(x_1, x_2) = 2x_1x_2^2$. What is $\int_{I^2} f dx$? \\

  $$g_1(x_1) = \int_0^1 2x_1x_2^2 dx = \left[ \frac{2}{3} x_1x_2^3 \right]_0^1 = \frac{2}{3} x_1 $$

  $$\int_{I^2} f dx = g_0 = \int_1^2 g_1(x_1) dx_1 = \int_1^2 \frac{2}{3} x_1 dx_1 = \left[ \frac{1}{3} x_1^2 \right]_1^2 = 1$$
\end{example}

\\Question. Does this depend on the order of integration?

Answer. No (try the other direction in the example above).

\begin{definition}
  If $f : \bbR^k \to \bbR$, the \emph{support} of $f$ is the closure of the set $\{ x \in \bbR^k : f(x) \neq 0 \}$.
\end{definition}

If $f: \bbR^k \to \bbR$ is continuous with compact support, let $I^k$ be any $k$-cell containing $\mathrm{supp}(f)$. We define $$\int_{\bbR^k} f dx = \int_{I^k} f dx$$

\begin{theorem} (Change of variables) Let $T$ be a 1-1, $C^1$ mapping of $E \subseteq \bbR^n$ open to $\bbR^n$. Also assume $J_T(x) \neq 0$ for all $x \in E$. If $f$ is continuous on $\bbR^n$ with compact support that is contained in $T(E)$, then $$\int_{\bbR^n} f(y) dy = \int_{\bbR^n} f(T(x))|J_T(x)|dx.
\end{theorem}
\end{document}
