\documentclass[11pt]{article}
\usepackage{master}
\title{MATH 20510 Lec 22}
\author{Andrew Hah}

\begin{document}

\pagestyle{plain}
\begin{center}
{\Large MATH 20510} \\
{\Large Lecture 22} \\
\vspace{.2in}
May 14, 2025
\end{center}

\begin{definition} Let $p_0, p_1, \dots, p_k \in \bbR^n$. The \emph{oriented affine $k$-simplex} $\sigma = [p_0, p_1, \dots, p_k]$ is the $k$-surface in $\bbR^n$ with parameter domain $Q^k$ given by the affine map \begin{align*} \sigma (\alpha_1e_1 + \dots + \alpha_k e_k) = p_0 + \sum_1^k \alpha_i (p_i - p_0).
\end{align*}
\end{definition}

\begin{remark} $\sigma(u) = p_0 + Au$, $u \in Q^k$, where $A \in L(\bbR^k, \bbR^n)$. $Ae_i = p_i - p_0$, $\forall 1 \le i \le k$.
\end{remark}

\begin{remark} $\sigma$ is called oriented to emphasize that the order of the points $p_0, p_1, \dots, p_k$ matters. If $\overline{\sigma} = [p_{i_0}, p_{i_1}, \dots, p_{i_k}]$ where $\{ i_0, i_1, \dots, i_k \}$ is a permutation of $\{ 0, 1, \dots, k \}$, then \begin{align*} \overline{\sigma} =  s(i_0, i_1, \dots, i_k) \sigma,
\end{align*} where $s(i_0, i_1, \dots, i_k) = (-1)^{\alpha}$, where $\alpha$ is the minimum number of swaps needed to transform $0, 1, \dots, k$ to $i_0, i_1, \dots, i_k$. If $s(i_0, i_1, \dots, i_k) = 1$, then we say $\sigma$ and $\overline{\sigma}$ have the same orientation, and if $s(i_0, i_1, \dots, i_k) = -1$, we say $\sigma$ and $\overline{\sigma}$ have opposite orientation.
\end{remark}

\begin{definition} An oriented $0$-simplex is a point $p \in \bbR^n$ with a sign attached, and we write $\sigma = +p_0$ or $\sigma = -p_0$. If $f$ is a $0$-form, $\sigma = \eps p_0, \eps = \pm 1$, \begin{align*} \int_{\sigma} f = \eps f (p_0).
\end{align*}
\end{definition}

\begin{theorem} If $\sigma$ is an oriented $k$-simplex in an open set $E \subseteq \bbR^n$ and if $\overline{\sigma} = \eps \sigma, \eps = \pm 1$, then $\forall k$-forms $\omega$ on $E$, \begin{align*} \int_{\sigma} \omega = \eps \int_{\overline{\sigma}} \omega.
\end{align*}
\end{theorem}

\begin{definition} An \emph{affine $k$-chain} $\Gamma$ in an open set $E \subseteq \bbR^n$ is a collection of finitely many oriented affine $k$-simplexes $\sigma_1, \dots, \sigma_r$ in $E$.

  Note. The simplexes need not be distinct.
\end{definition}

\begin{definition} If $\Gamma$ is an an affine $k$-chain in an open set $E \subseteq \bbR^n$ and $\omega$ is a $k$-form on $E$, \begin{align*} \int_{\Gamma} \omega = \sum_1^r \int_{\sigma_i} \omega.
\end{align*}
\end{definition}

Notation. This suggests the following notation, \begin{align*} \Gamma = \sigma_1 + \dots + \sigma_r = \sum_1^r \sigma_i.
\end{align*} Warning. This is \emph{just} notation.

\begin{example} $\sigma_1 = [p_0, p_1, p_2]$ and $\sigma_2 = [p_1, p_0, p_2]$, i.e., $\sigma_1 = -\sigma_2$. Then \begin{align*} \int_{\Gamma} \omega = \int_{\sigma_1} \omega + \int_{\sigma_2} \omega = \int_{\sigma_1} \omega - \int_{\sigma_1} \omega = 0.
\end{align*}
\end{example}

\begin{definition} For $k \ge 1$, the \emph{boundary} of an oriented affine $k$-simplex $\sigma = [p_0, p_1, \dots, p_k]$ is the affine $(k - 1)$-chain \begin{align*} \partial \sigma = \sum_{j = 0}^k (-1)^j [p_0, \dots, p_{j - 1}, p_{j + 1}, \dots, p_k].
\end{align*}
\end{definition}

\begin{example} $\sigma = [p_0, p_1, p_2]$. \begin{align*} \partial \sigma & = [p_1, p_2] - [p_0, p_2] + [p_0, p_1] \\ & = [p_1, p_2] + [p_2, p_0] + [p_0, p_1].
\end{align*}
\end{example}

\begin{example} $\sigma = [p_0, p_1, p_2, p_3]$. \begin{align*} \partial \sigma & = [p_1, p_2, p_3] - [p_0, p_2, p_3] + [p_0, p_1, p_3] - [p_0, p_1, p_2] \\ & = [p_1, p_2, p_3] + [p_2, p_0, p_3] + [p_0, p_1, p_3] + [p_1, p_0, p_2].
\end{align*}
\end{example}

\end{document}
