\documentclass[11pt]{article}
\usepackage{master}

\renewcommand{\abstractname}{}

\title{Accelerated Analysis 3 Lecture Notes}
\author{Andrew Hah}

\begin{document}
\thispagestyle{empty}  % no page number

\begin{center}
  {\Large MATH 20510. Analysis in $\bbR^{n}$ III (accelerated)}\par
  \vspace{3ex}
  {\large Based on lectures by Prof. Donald Stull}\par
  {\normalsize Notes taken by Andrew Hah}\par
  \vspace{2ex}
  {\small The University of Chicago - Spring 2025}
\end{center}

\section{Measure Theory}

\begin{definition}
    A family of sets $\mathscr{A}$ is called a \emph{ring} if, for every $A, B \in \mathscr{A}$, \begin{enumerate} [(i), nosep, left=0pt]
        \item $A \cup B \in \mathscr{A}$
        \item $A \setminus B \in \mathscr{A}$
    \end{enumerate}
\end{definition}

\begin{definition}
    A ring $\mathscr{A}$ is called a \emph{$\sigma$-ring} if for any $\{ A_n \}_1^\infty \subseteq \mathscr{A}$, $$\bigcup_1^\infty A_n \in \mathscr{A}.$$
\end{definition}

\begin{definition}
    $\phi$ is a \emph{set function} on a ring $\mathscr{A}$ if for every $A \in \mathscr{A}$, $$\phi(A) \in [-\infty, \infty].$$
\end{definition}

\begin{definition}
    A set function $\phi$ is \emph{additive} if for any $A, B \in \mathscr{A}$ such that $A \cap B = \emptyset$, $$\phi(A \cup B) = \phi(A) + \phi(B).$$
\end{definition}

\begin{definition}
    A set function $\phi$ is \emph{countably additive} if for any $\{ A_n \} \subseteq \mathscr{A}$ such that $A_i \cap A_j = \emptyset$, $\forall i \neq j$, $$\phi \left( \bigcup_1^n A_n \right) = \sum_1^n \phi(A_n).$$
\end{definition}
In the last two we assume that there are no $A, B \in \mathscr{A}$ such that $\phi(A) = -\infty, \phi(B) = \infty$. 

\begin{remark}
    If $\phi$ is an additive set function,
    \begin{enumerate} [(i), nosep, left=0pt]
        \item $\phi(\emptyset) = 0$.
        \item If $A_1, \dots, A_n$ are pairwise disjoint then $\phi(\bigcup_1^n A_n) = \sum_1^n \phi(A_n)$.
        \item $\phi(A_1 \cup A_2) + \phi(A_1 \cap A_2) = \phi(A_1) + \phi(A_2)$.
        \item If $\phi$ is nonnegative and $A_1 \subseteq A_2$ then $\phi(A_1) \le \phi(A_2)$.
        \item If $B \subseteq A$ and $|\phi(B)| < \infty$ then $\phi(A \setminus B) = \phi(A) - \phi(B)$. 
    \end{enumerate}
\end{remark}

\begin{theorem}
    Let $\phi$ be a countably additive set function on a ring $\mathscr{A}$. Suppose $\{ A_n \} \subseteq \mathscr{A}$ such that $A_1 \subseteq A_2 \subseteq \dots$ and $A = \bigcup_1^\infty A_n \in \mathscr{A}$. Then $\phi(A_n) \to \phi(A)$ as $n \to \infty$. 
\end{theorem}
\begin{proof}
    Set $B_1 = A_1$ and $B_n = A_n \setminus A_{n-1}$. Note \begin{enumerate} [(i), nosep, left=0pt]
        \item $\{ B_n \}$ is pairwise disjoint.
        \item $A_n = B_1 \cup B_2 \cup \dots \cup B_n$.
        \item $A = \bigcup_1^\infty B_n$.
    \end{enumerate}
    Hence $\phi(A_n) = \sum_1^\infty \phi(B_j)$, $\phi(A) = \sum_1^\infty \phi(B_j)$ and the conclusion follows.
\end{proof}

\begin{definition}
    An \emph{interval} $I = \{ (a_i, b_i) \}_1^n$ of $\mathbb{R}^n$ is the set of points $x = (x_1, \dots, x_n)$ such that $a_i \le x_i \le b_i$ or $a_i < x_i \le b_i$, etc. where $a_i \le b_i$. 
\end{definition}
Note. $\emptyset$ is an interval. 

\begin{definition}
    If $A$ is the union of a finite number of intervals, we say $A$ is \emph{elementary}.
\end{definition}
We denote the set of elementary sets by $\mathscr{E}$. 

\begin{definition}
    If $I$ is an interval of $\mathbb{R}^n$, we define the volume of $I$ by $$\mathrm{vol}(I) = \prod_i^n (b_i - a_i).$$
    If $A = I_1 \cup I_2 \cup \dots \cup I_k$ is elementary, and the intervals are disjoint, then $$\mathrm{vol}(A) = \sum_1^k \mathrm{vol}(I_j).$$
\end{definition}

\begin{remark} $\text{}$
  \begin{enumerate} [(i), nosep, left=0pt]
  \item $\mathscr{E}$ is a ring, but not a $\sigma$-ring.
  \item If $A \in \mathscr{E}$, then $A$ can be written as a finite union of disjoint intervals.
  \item If $A \in \sE$, then $\mathrm{vol}(A)$ is well-defined.
  \item $\mathrm{vol}$ is an additive set function on $\mathscr{E}$, and $\mathrm{vol} \ge 0$.
  \end{enumerate}
\end{remark}

\begin{definition}
  A nonnegative set function $\phi$ on $\sE$ is \emph{regular} if $\forall A \in \sE$, $\forall \eps > 0$, $\exists$ open $G \in \sE$, $G \supseteq A$ and closed $F \in \sE$, $F \subseteq A$, such that $$\phi(G) \le \phi(A) + \eps, \qquad \phi(A) \le \phi(F) + \eps.$$
\end{definition}

Note. $\mathrm{vol}$ is regular.

\begin{definition}
  A \emph{countable open cover} of $E \subseteq \bbR^{n}$ is a collection of open elementary sets $\{ A_{n} \}$ such that $E \subseteq \bigcup_{1}^{\infty} A_{n}$.
\end{definition}

\begin{definition}
  The \emph{Lebesgue outer measure} of $E \subseteq \bbR^{n}$ is defined as $$m^{*} (E) = \inf \sum_{1}^{\infty} \mathrm{vol}(A_{n}).$$ where $\inf$ is taken over all countable open covers of $E$.
\end{definition}

\begin{remark} $\text{}$
  \begin{enumerate} [(i), nosep, left=0pt]
       \item  m$^{*}(E)$ is well-defined.
       \item  $m^{*}(E) \ge 0$.
       \item  If $E_{1} \subseteq E_{2}$ then $m^{*}(E_{1}) \le m^{*}(E_{2})$.
  \end{enumerate}
\end{remark}

\begin{theorem} $\text{}$
  \begin{enumerate} [(i), nosep, left=0pt]
  \item If $A \in \sE$, then $m^{*}(A) = \mathrm{vol}(A)$.
  \item If $E = \bigcup_{1}^{\infty} E_{n}$ then $m^{*}(E) \le \sum_{1}^{\infty} m^{*}(E_{n})$.
  \end{enumerate}
\end{theorem}

\begin{proof}
  (i) Let $A \in \mathscr{E}$ and $\epsilon > 0$. Since $\mathrm{vol}$ is regular, $\exists$ open $G \in \mathscr{E}$ such that $A \subseteq G$ and $\mathrm{vol}(G) \le \mathrm{vol}(A) + \epsilon$. Since $G \supseteq A$ and $G \in \mathscr{E}$ is open, $m^*(A) \le \mathrm{vol}(G) \le \mathrm{vol}(A) + \epsilon$. There also $\exists$ closed $F \in \mathscr{E}$ such that $F \subseteq A$ and $\mathrm{vol}(A) \le \mathrm{vol}(F) + \epsilon$. By definition, $\exists$ collection $\{ A_n \}$ of open elementary sets such that $A \subseteq \bigcup A_n$ and $\sum_1^\infty \mathrm{vol}(A_n) \le m^*(A) + \epsilon$. Since $F \subseteq \bigcup A_n$ and $F$ is compact, $F\subseteq A_1 \cup \dots \cup A_N$ from some $N$. 
  \begin{align*} \mathrm{vol}(A) & \le \mathrm{vol}(F) + \epsilon \\ & \le \mathrm{vol}(A_1 \cup \dots \cup A_N) + \epsilon \\ & \le \sum_1^N \mathrm{vol}(A_n) + \epsilon \\ & \le \sum_1^\infty \mathrm{vol}(A_n) + \epsilon \\ & \le m^* (A) + \epsilon + \epsilon \\ & = m^* (A) + 2 \epsilon  \end{align*} Since $\epsilon$ was arbitrary, $m^*(A) = \mathrm{vol}(A)$.
\end{proof}
\begin{proof}
    (ii)  If $m^* (E_n) = \infty$ for any $n \in \mathbb{N}$, then we are done. Assume not. Let $\epsilon > 0$. For every $n \in \mathbb{N}$, $\exists$ open cover of $E_n$, $\{ A_{n, k} \}_{k = 1}^\infty$ such that $$\sum_{k = 1}^\infty \mathrm{vol}(A_{n, k}) \le m^* (E_n) + \epsilon / 2^n$$Then $E \subseteq \bigcup_{n = 1}^\infty \bigcup_{k = 1}^\infty A_{n, k}$ and so \begin{align*}
            m^*(E) & \le \sum_{n = 1}^\infty \sum_{k = 1}^\infty \mathrm{vol}(A_{n, k}) \\
            & \le \sum_{n = 1}^\infty m^*(E_n) + \epsilon / 2^n \\ & = \sum_{n = 1}^\infty m^*(E_n) + \sum_1^\infty \epsilon / 2^n \\
            & = \sum_1^\infty m^*(E_n) + \epsilon \qedhere
      \end{align*}
\end{proof}

\begin{definition} Let $A, B \subseteq \bbR^{n}$.
  \begin{enumerate} [(i), nosep, left=0pt]
  \item $A \triangle B = (A \setminus B) \cup (B \setminus A)$.
  \item $d(A, B) = m^{*}(A \triangle B)$.
  \item We say $A_{n} \to A$ if $\displaystyle \lim_{{n \to \infty}} d(A_{n}, A) = 0$.
  \end{enumerate}
\end{definition}

\begin{definition}
  If there is a sequence of elementary sets $\{ A_{n} \}$ such that $A_{n} \to A$ then we say $A$ is \emph{finitely $m$-measurable} and we write $A \in \mathfrak{M}_{F}(m)$.
\end{definition}

\begin{definition}
  If $A$ is the countable union of finitely $m$-measurable sets, we say that $A$ is \emph{$m$-measurable} (Lebesgue measurable) and we write $A \in \mathfrak{M}(m)$.
\end{definition}

\begin{theorem}
  $\mathfrak{M}(m)$ is a $\sigma$-ring and $m^{*}$ is countably additive on $\mathfrak{M}(m)$.
\end{theorem}

\begin{definition}
  The \emph{Lebesgue measure} is the set function defined on $\mathfrak{M}(m)$ by $$m(A) = m^{*}(A), \quad \forall A \in \mathfrak{M}(m).$$
\end{definition}

\normalfont

To summarize,
\begin{table}[ht]
  \centering
  \begin{tabularx}{\textwidth}{|l|l|Y|}
    \hline
    \textbf{set function} & \textbf{domain} & \textbf{properties} \\ 
    \hline
    $\mathrm{vol}$ 
      & $\mathcal{E}$ 
      & $\ge0$, additive, $\mathcal{E}$‑regular \\
    \hline
    $m^*$ 
      & $\subseteq \mathbb{R}^n$ 
      & \shortstack[l]{%
          $\ge0,\ m^*(A)=\mathrm{vol}(A)\ \forall A\in\mathcal{E},$\\%
          countably subadditive} \\
    \hline
    $m$ 
      & $\mathcal{M}(m)$ 
      & \shortstack[l]{%
          $\ge0,\ m(E)=m^*(E)\ \forall E\in\mathcal{M}(m),$\\%
          countable additivity(!)} \\
    \hline
  \end{tabularx}
\end{table}

\begin{example} Fix $n \in \bbN$.
  \begin{enumerate} [(i), nosep, left=0pt]
  \item If $A \in \sE$ then $A \in \mathfrak{M}(m)$ since $m^{*}(A \triangle A) = m^{*}( \emptyset) = 0 \implies A \to A$.
    \item $\bbR^n \in \mathfrak{M}(m)$ since $\bbR^n = \bigcup_{N \in \bbN} [-N, N]^n \implies m(\bbR^n) = \infty$.
    \item If $A \in \mathfrak{M}(m)$ then $A^c \in \mathfrak{M}(m)$.
    \item $\forall x \in \bbR^n$, $\{ x \} \in \mathfrak{M}(m)$ and $m(\{ x \}) = 0$.
    \item $\forall x_1, \dots, x_n \in \mathbb{R}^n$, $\{ x_1, \dots, x_n \} \in \mathfrak{M} (m)$ and $m(\{ x_1, \dots, x_n \}) = 0 \implies m(\mathbb{Q}^n) = 0$.
    \end{enumerate}
\end{example}

\begin{definition}
    $f: \mathbb{R}^n \to \mathbb{R} \cup \{ - \infty, \infty \}$ is \emph{measurable} if $\{ x \in \mathbb{R}^n : f(x) > a \} \in \mathfrak{M}$, $\forall a \in \mathbb{R}$, i.e. $f^{-1} ((a, \infty]) \in \mathfrak{M}$, $\forall a \in \mathbb{R}$.
\end{definition}

\begin{example}
    $f$ continuous $\implies$ $f$ measurable. 
\end{example}

\begin{theorem}
    The following are equivalent, \begin{enumerate} [(i), nosep, left=0pt]
        \item $\{ x : f(x) > a \}$ is measurable $\forall a \in \mathbb{R}$. 
        \item $\{ x : f(x) \ge a \}$ is measurable $\forall a \in \mathbb{R}$. 
        \item $\{ x : f(x) < a \}$ is measurable $\forall a \in \mathbb{R}$. 
        \item $\{ x : f(x) \le a \}$ is measurable $\forall a \in \mathbb{R}$. 
    \end{enumerate}
\end{theorem}
\begin{proof} (i) $\implies$ (ii).
     $$\{ x : f(x) \ge a \} = \bigcap_{n = 1}^\infty \left \{ x : f(x) > a - \frac{1}{n} \right \}$$
\end{proof}

\begin{theorem}
    If $f$ is measurable then $|f|$ is measurable. 
\end{theorem}
\begin{proof}
    It suffices to show that $\{ x : |f(x)| < a \} \in \mathfrak{M}$, $\forall a \in \mathbb{R}$. $$\{ x : |f(x)  | < a \} = \{ x : f(x) < a \} \cap \{ x : f(x) > -a \}$$
\end{proof}

\begin{theorem}
     Suppose $\{ f_n \}$ is a sequence of measurable functions. Define $$g = \sup_n f_n \quad \text{ and } \quad h = \limsup_{n \to \infty} f_n$$ Then $g, h$ are measurable. 
\end{theorem}
\begin{proof}
    $\{ x : g(x) > a \} = \bigcup_{n = 1}^\infty \{ x : f_n(x) > a \} $ implies $ g$ is measurable. Similarly, $\inf_n f_n$ is measurable. Define $g_n = \sup_{n \ge m} f_n$ and note that $g_n$ is measurable for all $m$. Since $h = \inf_m g_m$, $h$ is measurable. 
\end{proof}

\begin{corollary}
     If $f, g$ are measurable then $\max\{ f, g \}$ and $\min \{ f, g \}$ are also measurable.
\end{corollary}

\begin{corollary}
    Define $f^+ = \max \{ f, 0 \}$, $f^- = - \min \{ f, 0 \}$. Then if $f$ is measurable, $f^+, f^-$ are also measurable.
\end{corollary}

\begin{corollary}
    If $\{ f_n \}$ is a sequence of measurable functions such that $f_n$ converges to $f$ pointwise, then $f$ is measurable. 
\end{corollary}

\begin{theorem}
    $f, g: \mathbb{R}^n \to \mathbb{R}$ measurable, $F: \mathbb{R}^2 \to \mathbb{R}$ continuous, and $h(x) = F(f(x), g(x))$. Then $h$ is measurable. In particular, this tells us that $f + g$ and $fg$ are measurable. 
\end{theorem}

\begin{definition}
    A function $f: \mathbb{R}^n \to \mathbb{R}$ is \emph{simple} if $\mathrm{range}(f)$ is a finite set.
\end{definition}

\begin{example}
    Let $E \subseteq \mathbb{R}^n$. The characteristic function of $E$ is $$\chi_E (x) = \begin{cases} 1 & \text{if } x \in E \\ 0 & \text{otherwise} \end{cases}$$Suppose $f$ is simple, so $\mathrm{range}(f) = \{ c_1, \dots, c_m \}$. Let $E_i = \{ x : f(x) = c_i \}$. Then $$f = \sum_1^m \chi_{E_i} c_i$$
\end{example}

\begin{theorem}
    $f: \mathbb{R}^n \to \mathbb{R}$. There exists a sequence $\{ f_n \}$ of simple functions such that $f_n \to f$ pointwise. \begin{enumerate}
        \item If $f$ is measurable, $\{ f_n \}$ can be chosen to be measurable.
        \item If $f\ge 0$ then $\{ f_n \}$ can be chosen to be monotonically increasing.
    \end{enumerate} 
\end{theorem}
\begin{proof}
  \renewcommand{\qedsymbol}{\blacksquare}
    If $f \ge 0$, define the sets $$E_{n, i} = \left \{ x : \frac{i - 1}{2^n} \le f(x) < \frac{i}{2^n} \right \}, \quad n \ge 1, i = 1, \dots, n2^n$$ $$F_n = \{ x \mid f(x) \ge n \}, \quad n \ge 1$$Define $$f_n = \sum_{i = 1}^{n2^n} \frac{i - 1}{2^n} \chi_{E_{n, i}} + n \chi_{F_n}$$ We see that $f_n$ is measurable. Fix $x \in \bbR^n$, let $\eps > 0$, and let $N \in \bbN$ such that $N > f(x)$ and $2^{-N} < \eps$. Let $n \ge N$. Note that $x \in E_{n, i}$ for some $i$. Since $f_n(x) = \frac{i - 1}{2^n}$ and $f(x) \ge f_n(x)$, $f(x) - f_n(x) \le \frac{1}{2^n} < \eps$. Thus $f_n \to f$ pointwise. We now show $\{ f_n \} $ is monotonically increasing. 
    \begin{enumerate}
        \item Case 1: $x \in F_n$. Then $f(x) \ge n$ and $f_n(x) = n$. If $x \in F_{n + 1}$, then $f_{n + 1}(x) = n + 1 > n = f_n(x)$. If $x \notin F_{n + 1}$ then $x \in$ some $E_{n+1, i}$. Then $\frac{i - 1}{2^{n + 1}} \ge n \implies f_{n + 1}(x) \ge n = f_n(x)$.
        \item Case 2: $x \in E_{n, i}$ for some $i$. Then $f_n(x) = \frac{i - 1}{2^n}$. Then there is some $j$ such that $x \in E_{n + 1, j} = \{ x : \frac{j - 1}{2^{n + 1}} \le f(x) \le \frac{j}{2^{n + 1}} \} $. Because $\frac{i - 1}{2^n} \le f(x)$, we have $\frac{j - 1}{2^{n + 1}} \ge \frac{i - 1}{2^n}$ so $f_{n + 1}(x) = \frac{j - 1}{2^{n + 1}} \ge \frac{i - 1}{2^n} = f_n (x)$.
    \end{enumerate}
    Thus in both cases, $\{ f_n \}$ is monotonically increasing. We next consider the general case. Given $f$, we write $f^+(x) = \max \{ f(x), 0 \}$ and $f^- = - \min \{ f(x), 0 \}$ so that $f = f^+ - f^-$ and $f^+, f^- \ge 0$. By the previous part, there exist two sequences of nonnegative measurable simple functions $f_n^+ \to f^+$ and $f_n^- \to f^-$ each converging pointwise. Define $f_n(x) = f_n^+(x) - f_n^-(x)$. Then $f_n$ is simple and measurable since it is the difference of two simple measurable functions, and converges pointwise.
  \end{proof}

  \begin{definition} (Lebesgue Integration)
    Suppose $g = \sum_{i = 1}^k c_i\chi_{E_i}$, $c_i > 0$ is measurable and $E \in \mathfrak{M}$. Define $$I_E(g) = \sum_1^k c_i m(E_i \cap E)$$ Let $f$ be a nonnegative measurable function, $E \in \mathfrak{M}$. Define $$\int_E f dm = \sup I_E(g)$$ where $\sup$ is taken over all measurable simple functions $g$ such that $0 \le g \le f$.
  \end{definition}

  \begin{remark} $\text{}$
    \begin{enumerate}
    \item $\int_E f dm$ is the Lebesgue integral of $f$ over $E$.
    \item It can take value $\infty$.
    \item If $f$ is measurable, simple, and nonnegative, then $$\int_E f dm = I_{E}(f)$$
   \end{enumerate}
 \end{remark}

  \begin{proof} of remark (iii).
        Suppose for the sake of contradiction that there exists $g$ simple, nonnegative, and measurable such that $0 \le g \le f$ and $I_E(g) > I_E(f)$. Then $$g = \sum_1^{k} c_i \chi_{E_{i}}, \quad f = \sum_1^k d_j \chi_{F_j}$$ and $$I_E(g) = \sum_1^k c_i m(E_i \cap E) > I_E(f) = \sum_1^k d_j m(F_j \cap E)$$ Let $H_{i, j} = E_i \cap F_j$. Since $g \le f$, $\forall i$, $E_i \subseteq \bigcup F_j$. Hence, \begin{align*} g & = \sum_{i = 1}^k \sum_{j = 1}^k c_i \chi_{E_i \cap F_j} \\ & = \sum_{n = 1}^M c_n \chi_{H_n} \end{align*} Note that for every $n$, $\exists$ unique $F_j \supseteq H_n$. This implies $c_n\le d_j$, contradiction.
      \end{proof}

  \begin{definition}
    Let $f$ be measurable, and consider $\int_E f^+ dm$ and $\int_E f^- dm$. If at least one is finite, define $$\int_E f dm = \int_E f^+ dm - \int_E f^- dm$$ If both $\int_E f^+ dm$ and $\int_E f^- dm$ are finite, we say that $f$ is \emph{integrable} on $E$ and write $f \in \sL$ on $E$.
  \end{definition}

  \begin{remark} $\text{}$
    \begin{enumerate}
    \item If $a \le f(x) \le b$ for all $x \in E \in \mathfrak{M}$ and $m(E) < \infty$, then $am(E) \le \int_E f dm \le bm(E)$.
    \item If $f$ is bounded on $E \in \mathfrak{M}$ and $m(E) < \infty$, then $f \in \sL$ on $E$.
    \item If $f, g \in \sL$ on $E$ and $f(x) \le g(x)$ for all $x \in E$, then $\int_E f dm \le \int_E g dm$.
    \item If $f \in \sL$ on $E \in \mathfrak{M}$ and $c \in \bbR$ then $cf \in \sL$ on $E$ and $\int_E cf dm = c \int_E f dm$.
    \item If $m(E) = 0$ then $\int_E f dm = 0$.
    \item If $f \in \sL$ on $E$, $A \in \mathfrak{M}$, $A \subseteq E$, then $f \in \sL$ on $A$.
      \item If $f$ is Riemann integrable on $[a, b]$ then $f \in \sL$ on $[a, b]$ and the values of the integrals agree.
    \end{enumerate}
  \end{remark}

  \begin{proof} of remark (i). Assume $a \ge 0$. $\int_E f dm = \sup \int_E g dm$ where $\sup$ is taken over all simple measurable $g$ such that $0 \le g \le f$. Let $g = a$ on $E$. Then $\int_E f dm \ge \int_E g dm = a m(E)$. Let $g$ be a measurable simple function such that $0 \le g \le f$. Then $g = \sum_1^k c_i \chi_{E_i}$ for distinct $c_i$'s and measurable $E_i$ that are disjoint. Since $g \le f \le b$, $c_i \le b$ for all $i$. So \begin{align*} \int_E g dm & = \sum_1^k c_i m (E_i \cap E) \\ & \le b \sum_1^k m (E_i \cap E) \\ & \le b m(E) \end{align*}Hence, $\int_E f dm \le b m(E)$.
  \end{proof}

  \begin{theorem} $\text{}$
    \begin{enumerate}
    \item Suppose $f$ is nonnegative and measurable. For $A \in \mathfrak{M}$ define $$\phi(A) = \int_A f dm$$
      Then $\phi$ is countably additive on $\mathfrak{M}$.
    \item The same conclusion holds if $f \in \sL$.
    \end{enumerate}
  \end{theorem}

  \begin{proof} To prove (ii), it suffices to apply (i) to $f^+$ and $f^-$. Suppose $\{ A_n \}$ is a sequence of measurable sets which are pairwise disjoint. Let $A = \bigcup A_n$. \\

    Step 1 (Characteristic functions). Suppose $f = \chi_E$ for some $E \in \mathfrak{M}$. Then \begin{align*} \phi(A) & = \int_A f dm \\ & = m(A \cap E) \\ & = m \left( \left( \bigcup_1^{\infty} A_n \right) \cap E \right) \\ & = m\left( \bigcup_1^\infty (A_n \cap E) \right) \\ & = \sum_1^\infty m(A_n \cap E) \\ & = \sum_1^\infty \int_{A_n} f dm \\ & = \sum_1^\infty \phi(A_n) \end{align*}

    Step 2 (Simple functions). Suppose $f$ is simple, measurable, and nonnegative, i.e., $f = \sum_1^k c_{i} \chi_{E_i}$ for disjoint $E_i$'s in $\mathfrak{M}$. Then \begin{align*} \phi(A) & = \int_A f dm \\ & = \sum_1^k c_i m(E_i \cap A) \\ & = \sum_1^k c_i \int_A \chi_{E_i} dm \\ & = \sum_1^k c_i \sum_1^\infty \int_{A_n} \chi_{E_i} dm \\ & = \sum_1^\infty \sum_1^k \int_{A_n} c_i \chi_{E_i} dm \\ & = \sum_1^\infty \int_{A_n} f dm \\ & = \sum_1^\infty \phi(A_n) \end{align*}

    Step 3. Let $g$ be a measurable simple function such that $0 \le g \le f$. Then \begin{align*} \int_A g dm & = \sum_1^\infty \int_{A_n} g dm \\ & \le \sum_1^\infty \int_{A_n} f dm \\ & = \sum_1^\infty \phi(A_n) \end{align*} Hence $\phi(A) = \int_A f dm \le \sum_1^{\infty} \phi(A_n)$.

    If $\phi(A_n) = \infty$ for any $n$, then we are done. Thus assume $\phi(A_n) < \infty$ for every $n$. Let $\epsilon > 0$, and choose measurable simple $g$ such that $0 \le g \le f$ and $\int_{A_1} g dm \ge \int_{A_1} f dm - \epsilon, \dots, \int_{A_n} g dm \ge \int_{A_n} f dm - \epsilon$. Hence $$\phi(A_1 \cup \dots \cup A_n) \ge \phi(A_1) + \dots + \phi(A_n) - n \epsilon$$Since $\epsilon$ was arbitrary, $\forall n$, $\phi(A_1 \cup \dots \cup A_n) \ge \phi(A_1) + \dots + \phi(A_n)$.
  \end{proof}

  \begin{corollary} If $A, B \in \mathfrak{M}$, $m(A \setminus B) = 0$, and $B \subseteq A$, then $$\int_A f dm = \int_B f dm$$ for every $f \in \sL$.
  \end{corollary}

  \begin{theorem} If $f \in \mathscr{L}$ on $E$, then $|f| \in \mathscr{L}$ on $E$ and $|\int_E f dm| \le \int_E |f| dm$.
  \end{theorem}
  \begin{proof}
    Let $A = \{ x \in E \mid f(x) \ge 0 \}$ and $B = \{ x \in E \mid f(x) < 0 \}$. Note that $E = A \sqcup B$ and $A, B \in \mathfrak{M}$. Then $$\int_E |f| dm = \int_A |f| dm + \int_B |f| dm = \int_E f^+ dm + \int_E f^- dm < \infty$$Thus $|f| \in \mathscr{L}$. Since $f \le |f|$ and $-f \le |f|$, $\int_E f dm \le \int_E f dm \le \int_E |f| dm$, and $\int_E -f dm = - \int_E f dm \le \int_E |f| dm$ so $$\left| \int_E f dm \right| \le \int_E |f| dm$$
  \end{proof}

  \begin{theorem} (Lebesgue's Monotone Convergence Theorem). Let $E \in \mathfrak{M}$ and $\{ f_n \}$ a sequence of measurable functions such that $$0 \le f_1 (x) \le f_2(x) \le \dots \quad \forall (x \in E)$$Define $f(x) = \lim_{n \to \infty} f_n(x)$ for all $x \in E$. Then $$\int_E  f_n dm \to \int_E f dm \quad (n \to \infty)$$
  \end{theorem}
  \begin{proof}
    Since $\{ f_n \}$ is a monotone sequence of nonnegative measurable functions,  $\{ \int_E f_n dm \}$ is a monotone sequence of extended real numbers. Thus there must exist $\alpha \in \mathbb{R} \cup \{ \pm \infty \}$ such that $\alpha = \lim_{n \to \infty} \int_E f_n dm$. Since $f_n \le f$ for every $n$, $\alpha \le \int_E f dm$. 
    Let $0 < c < 1$ and $g$ be a simple, measurable function such that $0 \le g \le f$. For every $n \ge 1$, define $$E_n = \{ x \in E \mid f_n(x) \ge cg(x) \}$$Since $\{ f_n \}$ is increasing, $E_1 \subseteq E_2 \subseteq \dots$. Since $f_n \to f$ pointwise, $E = \bigcup_1^\infty E_n$. For every $n$, $cg \le f_n$ on $E_n$, so $$c \int_{E_n} g dm = \int_{E_n} cg dm \le \int_{E_n} f_n dm$$As $n \to \infty$, $$\int_{E_n} g dm \to \int_E g dm$$Therefore, $\alpha \ge c \int_{E} g dm$. Since $c < 1$ was arbitrary, $\alpha \ge \int_E g dm$. By definition of integration, $\alpha \ge \int_E  f dm$.
  \end{proof}
  
\end{document}
