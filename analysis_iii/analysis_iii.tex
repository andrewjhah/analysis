\documentclass[11pt]{article}
\usepackage{master}

\usepackage{titling}

\fancypagestyle{plain}{%
  \fancyhf{}            % clear all header and footer fields
  \renewcommand\headrulewidth{0pt}%
  \renewcommand\footrulewidth{0pt}%
}

\renewcommand{\abstractname}{}

% \title{Accelerated Analysis 3 Lecture Notes}
% \author{Andrew Hah}

\title{MATH 20510. Analysis in \(\mathbb{R}^n\) III  (accelerated)}
\author{%
  Based on lectures by Prof.\ Donald Stull\\
  {\large Notes taken by Andrew Hah}\\[2ex]
  {\normalsize The University of Chicago – Spring 2025}
}
\date{}  % no date

% adjust spacing before/after title
\setlength{\droptitle}{-1cm}
\pretitle{\begin{center}\LARGE}
\posttitle{\end{center}\par\vskip 1ex}
\preauthor{\begin{center}\large}
\postauthor{\end{center}\par\vskip 2ex}
\predate{}
\postdate{}

\begin{document}

\maketitle
\thispagestyle{empty}

\vspace*{-1em}
\noindent
\begin{abstract} Any proof or argument that has been filled in, expanded, or written out in detail by me is marked with a \(\blacksquare\).  
  All other material follows the lectures and any errors or omissions are entirely my own.
  \end{abstract}

\vspace{2em}
\noindent
\tableofcontents

\newpage


\section{Measure and Integration}

\begin{definition}
    A family of sets $\mathscr{A}$ is called a \emph{ring} if, for every $A, B \in \mathscr{A}$, \begin{enumerate} [(i), nosep, left=0pt]
        \item $A \cup B \in \mathscr{A}$
        \item $A \setminus B \in \mathscr{A}$
    \end{enumerate}
\end{definition}

\begin{definition}
    A ring $\mathscr{A}$ is called a \emph{$\sigma$-ring} if for any $\{ A_n \}_1^\infty \subseteq \mathscr{A}$, $$\bigcup_1^\infty A_n \in \mathscr{A}.$$
\end{definition}

\begin{definition}
    $\phi$ is a \emph{set function} on a ring $\mathscr{A}$ if for every $A \in \mathscr{A}$, $$\phi(A) \in [-\infty, \infty].$$
\end{definition}

\begin{definition}
    A set function $\phi$ is \emph{additive} if for any $A, B \in \mathscr{A}$ such that $A \cap B = \emptyset$, $$\phi(A \cup B) = \phi(A) + \phi(B).$$
\end{definition}

\begin{definition}
    A set function $\phi$ is \emph{countably additive} if for any $\{ A_n \} \subseteq \mathscr{A}$ such that $A_i \cap A_j = \emptyset$, $\forall i \neq j$, $$\phi \left( \bigcup_1^n A_n \right) = \sum_1^n \phi(A_n).$$
\end{definition}
In the last two we assume that there are no $A, B \in \mathscr{A}$ such that $\phi(A) = -\infty, \phi(B) = \infty$. 

\begin{remark}
    If $\phi$ is an additive set function,
    \begin{enumerate} [(i), nosep, left=0pt]
        \item $\phi(\emptyset) = 0$.
        \item If $A_1, \dots, A_n$ are pairwise disjoint then $\phi(\bigcup_1^n A_n) = \sum_1^n \phi(A_n)$.
        \item $\phi(A_1 \cup A_2) + \phi(A_1 \cap A_2) = \phi(A_1) + \phi(A_2)$.
        \item If $\phi$ is nonnegative and $A_1 \subseteq A_2$ then $\phi(A_1) \le \phi(A_2)$.
        \item If $B \subseteq A$ and $|\phi(B)| < \infty$ then $\phi(A \setminus B) = \phi(A) - \phi(B)$. 
    \end{enumerate}
\end{remark}

\begin{theorem}
    Let $\phi$ be a countably additive set function on a ring $\mathscr{A}$. Suppose $\{ A_n \} \subseteq \mathscr{A}$ such that $A_1 \subseteq A_2 \subseteq \dots$ and $A = \bigcup_1^\infty A_n \in \mathscr{A}$. Then $\phi(A_n) \to \phi(A)$ as $n \to \infty$. 
\end{theorem}
\begin{proof}
    Set $B_1 = A_1$ and $B_n = A_n \setminus A_{n-1}$. Note \begin{enumerate} [(i), nosep, left=0pt]
        \item $\{ B_n \}$ is pairwise disjoint.
        \item $A_n = B_1 \cup B_2 \cup \dots \cup B_n$.
        \item $A = \bigcup_1^\infty B_n$.
    \end{enumerate}
    Hence $\phi(A_n) = \sum_1^\infty \phi(B_j)$, $\phi(A) = \sum_1^\infty \phi(B_j)$ and the conclusion follows.
\end{proof}

\begin{definition}
    An \emph{interval} $I = \{ (a_i, b_i) \}_1^n$ of $\mathbb{R}^n$ is the set of points $x = (x_1, \dots, x_n)$ such that $a_i \le x_i \le b_i$ or $a_i < x_i \le b_i$, etc. where $a_i \le b_i$. 
\end{definition}
Note. $\emptyset$ is an interval. 

\begin{definition}
    If $A$ is the union of a finite number of intervals, we say $A$ is \emph{elementary}.
\end{definition}
We denote the set of elementary sets by $\mathscr{E}$. 

\begin{definition}
    If $I$ is an interval of $\mathbb{R}^n$, we define the volume of $I$ by $$\mathrm{vol}(I) = \prod_i^n (b_i - a_i).$$
    If $A = I_1 \cup I_2 \cup \dots \cup I_k$ is elementary, and the intervals are disjoint, then $$\mathrm{vol}(A) = \sum_1^k \mathrm{vol}(I_j).$$
\end{definition}

\begin{remark} $\text{}$
  \begin{enumerate} [(i), nosep, left=0pt]
  \item $\mathscr{E}$ is a ring, but not a $\sigma$-ring.
  \item If $A \in \mathscr{E}$, then $A$ can be written as a finite union of disjoint intervals.
  \item If $A \in \sE$, then $\mathrm{vol}(A)$ is well-defined.
  \item $\mathrm{vol}$ is an additive set function on $\mathscr{E}$, and $\mathrm{vol} \ge 0$.
  \end{enumerate}
\end{remark}

\begin{definition}
  A nonnegative set function $\phi$ on $\sE$ is \emph{regular} if $\forall A \in \sE$, $\forall \eps > 0$, $\exists$ open $G \in \sE$, $G \supseteq A$ and closed $F \in \sE$, $F \subseteq A$, such that $$\phi(G) \le \phi(A) + \eps, \qquad \phi(A) \le \phi(F) + \eps.$$
\end{definition}

Note. $\mathrm{vol}$ is regular.

\begin{definition}
  A \emph{countable open cover} of $E \subseteq \bbR^{n}$ is a collection of open elementary sets $\{ A_{n} \}$ such that $E \subseteq \bigcup_{1}^{\infty} A_{n}$.
\end{definition}

\begin{definition}
  The \emph{Lebesgue outer measure} of $E \subseteq \bbR^{n}$ is defined as $$m^{*} (E) = \inf \sum_{1}^{\infty} \mathrm{vol}(A_{n}).$$ where $\inf$ is taken over all countable open covers of $E$.
\end{definition}

\begin{remark} $\text{}$
  \begin{enumerate} [(i), nosep, left=0pt]
       \item  m$^{*}(E)$ is well-defined.
       \item  $m^{*}(E) \ge 0$.
       \item  If $E_{1} \subseteq E_{2}$ then $m^{*}(E_{1}) \le m^{*}(E_{2})$.
  \end{enumerate}
\end{remark}

\begin{theorem} $\text{}$
  \begin{enumerate} [(i), nosep, left=0pt]
  \item If $A \in \sE$, then $m^{*}(A) = \mathrm{vol}(A)$.
  \item If $E = \bigcup_{1}^{\infty} E_{n}$ then $m^{*}(E) \le \sum_{1}^{\infty} m^{*}(E_{n})$.
  \end{enumerate}
\end{theorem}

\begin{proof}
  (i) Let $A \in \mathscr{E}$ and $\epsilon > 0$. Since $\mathrm{vol}$ is regular, $\exists$ open $G \in \mathscr{E}$ such that $A \subseteq G$ and $\mathrm{vol}(G) \le \mathrm{vol}(A) + \epsilon$. Since $G \supseteq A$ and $G \in \mathscr{E}$ is open, $m^*(A) \le \mathrm{vol}(G) \le \mathrm{vol}(A) + \epsilon$. There also $\exists$ closed $F \in \mathscr{E}$ such that $F \subseteq A$ and $\mathrm{vol}(A) \le \mathrm{vol}(F) + \epsilon$. By definition, $\exists$ collection $\{ A_n \}$ of open elementary sets such that $A \subseteq \bigcup A_n$ and $\sum_1^\infty \mathrm{vol}(A_n) \le m^*(A) + \epsilon$. Since $F \subseteq \bigcup A_n$ and $F$ is compact, $F\subseteq A_1 \cup \dots \cup A_N$ from some $N$. 
  \begin{align*} \mathrm{vol}(A) & \le \mathrm{vol}(F) + \epsilon \\ & \le \mathrm{vol}(A_1 \cup \dots \cup A_N) + \epsilon \\ & \le \sum_1^N \mathrm{vol}(A_n) + \epsilon \\ & \le \sum_1^\infty \mathrm{vol}(A_n) + \epsilon \\ & \le m^* (A) + \epsilon + \epsilon \\ & = m^* (A) + 2 \epsilon  \end{align*} Since $\epsilon$ was arbitrary, $m^*(A) = \mathrm{vol}(A)$.
\end{proof}
\begin{proof}
    (ii)  If $m^* (E_n) = \infty$ for any $n \in \mathbb{N}$, then we are done. Assume not. Let $\epsilon > 0$. For every $n \in \mathbb{N}$, $\exists$ open cover of $E_n$, $\{ A_{n, k} \}_{k = 1}^\infty$ such that $$\sum_{k = 1}^\infty \mathrm{vol}(A_{n, k}) \le m^* (E_n) + \epsilon / 2^n$$Then $E \subseteq \bigcup_{n = 1}^\infty \bigcup_{k = 1}^\infty A_{n, k}$ and so \begin{align*}
            m^*(E) & \le \sum_{n = 1}^\infty \sum_{k = 1}^\infty \mathrm{vol}(A_{n, k}) \\
            & \le \sum_{n = 1}^\infty m^*(E_n) + \epsilon / 2^n \\ & = \sum_{n = 1}^\infty m^*(E_n) + \sum_1^\infty \epsilon / 2^n \\
            & = \sum_1^\infty m^*(E_n) + \epsilon \qedhere
      \end{align*}
\end{proof}

\begin{definition} Let $A, B \subseteq \bbR^{n}$.
  \begin{enumerate} [(i), nosep, left=0pt]
  \item $A \triangle B = (A \setminus B) \cup (B \setminus A)$.
  \item $d(A, B) = m^{*}(A \triangle B)$.
  \item We say $A_{n} \to A$ if $\displaystyle \lim_{{n \to \infty}} d(A_{n}, A) = 0$.
  \end{enumerate}
\end{definition}

\begin{definition}
  If there is a sequence of elementary sets $\{ A_{n} \}$ such that $A_{n} \to A$ then we say $A$ is \emph{finitely $m$-measurable} and we write $A \in \mathfrak{M}_{F}(m)$.
\end{definition}

\begin{definition}
  If $A$ is the countable union of finitely $m$-measurable sets, we say that $A$ is \emph{$m$-measurable} (Lebesgue measurable) and we write $A \in \mathfrak{M}(m)$.
\end{definition}

\begin{theorem}
  $\mathfrak{M}(m)$ is a $\sigma$-ring and $m^{*}$ is countably additive on $\mathfrak{M}(m)$.
\end{theorem}

\begin{definition}
  The \emph{Lebesgue measure} is the set function defined on $\mathfrak{M}(m)$ by $$m(A) = m^{*}(A), \quad \forall A \in \mathfrak{M}(m).$$
\end{definition}

\normalfont

To summarize,
\begin{table}[ht]
  \centering
  \begin{tabularx}{\textwidth}{|l|l|Y|}
    \hline
    \textbf{set function} & \textbf{domain} & \textbf{properties} \\ 
    \hline
    $\mathrm{vol}$ 
      & $\mathcal{E}$ 
      & $\ge0$, additive, $\mathcal{E}$‑regular \\
    \hline
    $m^*$ 
      & $\subseteq \mathbb{R}^n$ 
      & \shortstack[l]{%
          $\ge0,\ m^*(A)=\mathrm{vol}(A)\ \forall A\in\mathcal{E},$\\%
          countably subadditive} \\
    \hline
    $m$ 
      & $\mathcal{M}(m)$ 
      & \shortstack[l]{%
          $\ge0,\ m(E)=m^*(E)\ \forall E\in\mathcal{M}(m),$\\%
          countable additivity(!)} \\
    \hline
  \end{tabularx}
\end{table}

\begin{example} Fix $n \in \bbN$.
  \begin{enumerate} [(i), nosep, left=0pt]
  \item If $A \in \sE$ then $A \in \mathfrak{M}(m)$ since $m^{*}(A \triangle A) = m^{*}( \emptyset) = 0 \implies A \to A$.
    \item $\bbR^n \in \mathfrak{M}(m)$ since $\bbR^n = \bigcup_{N \in \bbN} [-N, N]^n \implies m(\bbR^n) = \infty$.
    \item If $A \in \mathfrak{M}(m)$ then $A^c \in \mathfrak{M}(m)$.
    \item $\forall x \in \bbR^n$, $\{ x \} \in \mathfrak{M}(m)$ and $m(\{ x \}) = 0$.
    \item $\forall x_1, \dots, x_n \in \mathbb{R}^n$, $\{ x_1, \dots, x_n \} \in \mathfrak{M} (m)$ and $m(\{ x_1, \dots, x_n \}) = 0 \implies m(\mathbb{Q}^n) = 0$.
    \end{enumerate}
\end{example}

\begin{definition}
    $f: \mathbb{R}^n \to \mathbb{R} \cup \{ - \infty, \infty \}$ is \emph{measurable} if $\{ x \in \mathbb{R}^n : f(x) > a \} \in \mathfrak{M}$, $\forall a \in \mathbb{R}$, i.e. $f^{-1} ((a, \infty]) \in \mathfrak{M}$, $\forall a \in \mathbb{R}$.
\end{definition}

\begin{example}
    $f$ continuous $\implies$ $f$ measurable. 
\end{example}

\begin{theorem}
    The following are equivalent, \begin{enumerate} [(i), nosep, left=0pt]
        \item $\{ x : f(x) > a \}$ is measurable $\forall a \in \mathbb{R}$. 
        \item $\{ x : f(x) \ge a \}$ is measurable $\forall a \in \mathbb{R}$. 
        \item $\{ x : f(x) < a \}$ is measurable $\forall a \in \mathbb{R}$. 
        \item $\{ x : f(x) \le a \}$ is measurable $\forall a \in \mathbb{R}$. 
    \end{enumerate}
\end{theorem}
\begin{proof} (i) $\implies$ (ii).
     $$\{ x : f(x) \ge a \} = \bigcap_{n = 1}^\infty \left \{ x : f(x) > a - \frac{1}{n} \right \}$$
\end{proof}

\begin{theorem}
    If $f$ is measurable then $|f|$ is measurable. 
\end{theorem}
\begin{proof}
    It suffices to show that $\{ x : |f(x)| < a \} \in \mathfrak{M}$, $\forall a \in \mathbb{R}$. $$\{ x : |f(x)  | < a \} = \{ x : f(x) < a \} \cap \{ x : f(x) > -a \}$$
\end{proof}

\begin{theorem}
     Suppose $\{ f_n \}$ is a sequence of measurable functions. Define $$g = \sup_n f_n \quad \text{ and } \quad h = \limsup_{n \to \infty} f_n$$ Then $g, h$ are measurable. 
\end{theorem}
\begin{proof}
    $\{ x : g(x) > a \} = \bigcup_{n = 1}^\infty \{ x : f_n(x) > a \} $ implies $ g$ is measurable. Similarly, $\inf_n f_n$ is measurable. Define $g_n = \sup_{n \ge m} f_n$ and note that $g_n$ is measurable for all $m$. Since $h = \inf_m g_m$, $h$ is measurable. 
\end{proof}

\begin{corollary}
     If $f, g$ are measurable then $\max\{ f, g \}$ and $\min \{ f, g \}$ are also measurable.
\end{corollary}

\begin{corollary}
    Define $f^+ = \max \{ f, 0 \}$, $f^- = - \min \{ f, 0 \}$. Then if $f$ is measurable, $f^+, f^-$ are also measurable.
\end{corollary}

\begin{corollary}
    If $\{ f_n \}$ is a sequence of measurable functions such that $f_n$ converges to $f$ pointwise, then $f$ is measurable. 
\end{corollary}

\begin{theorem}
    $f, g: \mathbb{R}^n \to \mathbb{R}$ measurable, $F: \mathbb{R}^2 \to \mathbb{R}$ continuous, and $h(x) = F(f(x), g(x))$. Then $h$ is measurable. In particular, this tells us that $f + g$ and $fg$ are measurable. 
\end{theorem}

\begin{definition}
    A function $f: \mathbb{R}^n \to \mathbb{R}$ is \emph{simple} if $\mathrm{range}(f)$ is a finite set.
\end{definition}

\begin{example}
    Let $E \subseteq \mathbb{R}^n$. The characteristic function of $E$ is $$\chi_E (x) = \begin{cases} 1 & \text{if } x \in E \\ 0 & \text{otherwise} \end{cases}$$Suppose $f$ is simple, so $\mathrm{range}(f) = \{ c_1, \dots, c_m \}$. Let $E_i = \{ x : f(x) = c_i \}$. Then $$f = \sum_1^m \chi_{E_i} c_i$$
\end{example}

\begin{theorem}
    $f: \mathbb{R}^n \to \mathbb{R}$. There exists a sequence $\{ f_n \}$ of simple functions such that $f_n \to f$ pointwise. \begin{enumerate}
        \item If $f$ is measurable, $\{ f_n \}$ can be chosen to be measurable.
        \item If $f\ge 0$ then $\{ f_n \}$ can be chosen to be monotonically increasing.
    \end{enumerate} 
\end{theorem}
\begin{proof}
  \renewcommand{\qedsymbol}{\ensuremath{\blacksquare}}
    If $f \ge 0$, define the sets $$E_{n, i} = \left \{ x : \frac{i - 1}{2^n} \le f(x) < \frac{i}{2^n} \right \}, \quad n \ge 1, i = 1, \dots, n2^n$$ $$F_n = \{ x \mid f(x) \ge n \}, \quad n \ge 1$$Define $$f_n = \sum_{i = 1}^{n2^n} \frac{i - 1}{2^n} \chi_{E_{n, i}} + n \chi_{F_n}$$ We see that $f_n$ is measurable. Fix $x \in \bbR^n$, let $\eps > 0$, and let $N \in \bbN$ such that $N > f(x)$ and $2^{-N} < \eps$. Let $n \ge N$. Note that $x \in E_{n, i}$ for some $i$. Since $f_n(x) = \frac{i - 1}{2^n}$ and $f(x) \ge f_n(x)$, $f(x) - f_n(x) \le \frac{1}{2^n} < \eps$. Thus $f_n \to f$ pointwise. We now show $\{ f_n \} $ is monotonically increasing. 
    \begin{enumerate}
        \item Case 1: $x \in F_n$. Then $f(x) \ge n$ and $f_n(x) = n$. If $x \in F_{n + 1}$, then $f_{n + 1}(x) = n + 1 > n = f_n(x)$. If $x \notin F_{n + 1}$ then $x \in$ some $E_{n+1, i}$. Then $\frac{i - 1}{2^{n + 1}} \ge n \implies f_{n + 1}(x) \ge n = f_n(x)$.
        \item Case 2: $x \in E_{n, i}$ for some $i$. Then $f_n(x) = \frac{i - 1}{2^n}$. Then there is some $j$ such that $x \in E_{n + 1, j} = \{ x : \frac{j - 1}{2^{n + 1}} \le f(x) \le \frac{j}{2^{n + 1}} \} $. Because $\frac{i - 1}{2^n} \le f(x)$, we have $\frac{j - 1}{2^{n + 1}} \ge \frac{i - 1}{2^n}$ so $f_{n + 1}(x) = \frac{j - 1}{2^{n + 1}} \ge \frac{i - 1}{2^n} = f_n (x)$.
    \end{enumerate}
    Thus in both cases, $\{ f_n \}$ is monotonically increasing. We next consider the general case. Given $f$, we write $f^+(x) = \max \{ f(x), 0 \}$ and $f^- = - \min \{ f(x), 0 \}$ so that $f = f^+ - f^-$ and $f^+, f^- \ge 0$. By the previous part, there exist two sequences of nonnegative measurable simple functions $f_n^+ \to f^+$ and $f_n^- \to f^-$ each converging pointwise. Define $f_n(x) = f_n^+(x) - f_n^-(x)$. Then $f_n$ is simple and measurable since it is the difference of two simple measurable functions, and converges pointwise.
  \end{proof}

  \begin{definition} (Lebesgue Integration)
    Suppose $g = \sum_{i = 1}^k c_i\chi_{E_i}$, $c_i > 0$ is measurable and $E \in \mathfrak{M}$. Define $$I_E(g) = \sum_1^k c_i m(E_i \cap E)$$ Let $f$ be a nonnegative measurable function, $E \in \mathfrak{M}$. Define $$\int_E f dm = \sup I_E(g)$$ where $\sup$ is taken over all measurable simple functions $g$ such that $0 \le g \le f$.
  \end{definition}

  \begin{remark} $\text{}$
    \begin{enumerate}
    \item $\int_E f dm$ is the Lebesgue integral of $f$ over $E$.
    \item It can take value $\infty$.
    \item If $f$ is measurable, simple, and nonnegative, then $$\int_E f dm = I_{E}(f)$$
   \end{enumerate}
 \end{remark}

  \begin{proof} of remark (iii).
        Suppose for the sake of contradiction that there exists $g$ simple, nonnegative, and measurable such that $0 \le g \le f$ and $I_E(g) > I_E(f)$. Then $$g = \sum_1^{k} c_i \chi_{E_{i}}, \quad f = \sum_1^k d_j \chi_{F_j}$$ and $$I_E(g) = \sum_1^k c_i m(E_i \cap E) > I_E(f) = \sum_1^k d_j m(F_j \cap E)$$ Let $H_{i, j} = E_i \cap F_j$. Since $g \le f$, $\forall i$, $E_i \subseteq \bigcup F_j$. Hence, \begin{align*} g & = \sum_{i = 1}^k \sum_{j = 1}^k c_i \chi_{E_i \cap F_j} \\ & = \sum_{n = 1}^M c_n \chi_{H_n} \end{align*} Note that for every $n$, $\exists$ unique $F_j \supseteq H_n$. This implies $c_n\le d_j$, contradiction.
      \end{proof}

  \begin{definition}
    Let $f$ be measurable, and consider $\int_E f^+ dm$ and $\int_E f^- dm$. If at least one is finite, define $$\int_E f dm = \int_E f^+ dm - \int_E f^- dm$$ If both $\int_E f^+ dm$ and $\int_E f^- dm$ are finite, we say that $f$ is \emph{integrable} on $E$ and write $f \in \sL$ on $E$.
  \end{definition}

  \begin{remark} $\text{}$
    \begin{enumerate}
    \item If $a \le f(x) \le b$ for all $x \in E \in \mathfrak{M}$ and $m(E) < \infty$, then $am(E) \le \int_E f dm \le bm(E)$.
    \item If $f$ is bounded on $E \in \mathfrak{M}$ and $m(E) < \infty$, then $f \in \sL$ on $E$.
    \item If $f, g \in \sL$ on $E$ and $f(x) \le g(x)$ for all $x \in E$, then $\int_E f dm \le \int_E g dm$.
    \item If $f \in \sL$ on $E \in \mathfrak{M}$ and $c \in \bbR$ then $cf \in \sL$ on $E$ and $\int_E cf dm = c \int_E f dm$.
    \item If $m(E) = 0$ then $\int_E f dm = 0$.
    \item If $f \in \sL$ on $E$, $A \in \mathfrak{M}$, $A \subseteq E$, then $f \in \sL$ on $A$.
      \item If $f$ is Riemann integrable on $[a, b]$ then $f \in \sL$ on $[a, b]$ and the values of the integrals agree.
    \end{enumerate}
  \end{remark}

  \begin{proof} of remark (i). Assume $a \ge 0$. $\int_E f dm = \sup \int_E g dm$ where $\sup$ is taken over all simple measurable $g$ such that $0 \le g \le f$. Let $g = a$ on $E$. Then $\int_E f dm \ge \int_E g dm = a m(E)$. Let $g$ be a measurable simple function such that $0 \le g \le f$. Then $g = \sum_1^k c_i \chi_{E_i}$ for distinct $c_i$'s and measurable $E_i$ that are disjoint. Since $g \le f \le b$, $c_i \le b$ for all $i$. So \begin{align*} \int_E g dm & = \sum_1^k c_i m (E_i \cap E) \\ & \le b \sum_1^k m (E_i \cap E) \\ & \le b m(E) \end{align*}Hence, $\int_E f dm \le b m(E)$.
  \end{proof}

  \begin{theorem} $\text{}$
    \begin{enumerate}
    \item Suppose $f$ is nonnegative and measurable. For $A \in \mathfrak{M}$ define $$\phi(A) = \int_A f dm$$
      Then $\phi$ is countably additive on $\mathfrak{M}$.
    \item The same conclusion holds if $f \in \sL$.
    \end{enumerate}
  \end{theorem}

  \begin{proof} To prove (ii), it suffices to apply (i) to $f^+$ and $f^-$. Suppose $\{ A_n \}$ is a sequence of measurable sets which are pairwise disjoint. Let $A = \bigcup A_n$. \\

    Step 1 (Characteristic functions). Suppose $f = \chi_E$ for some $E \in \mathfrak{M}$. Then \begin{align*} \phi(A) & = \int_A f dm \\ & = m(A \cap E) \\ & = m \left( \left( \bigcup_1^{\infty} A_n \right) \cap E \right) \\ & = m\left( \bigcup_1^\infty (A_n \cap E) \right) \\ & = \sum_1^\infty m(A_n \cap E) \\ & = \sum_1^\infty \int_{A_n} f dm \\ & = \sum_1^\infty \phi(A_n) \end{align*}

    Step 2 (Simple functions). Suppose $f$ is simple, measurable, and nonnegative, i.e., $f = \sum_1^k c_{i} \chi_{E_i}$ for disjoint $E_i$'s in $\mathfrak{M}$. Then \begin{align*} \phi(A) & = \int_A f dm \\ & = \sum_1^k c_i m(E_i \cap A) \\ & = \sum_1^k c_i \int_A \chi_{E_i} dm \\ & = \sum_1^k c_i \sum_1^\infty \int_{A_n} \chi_{E_i} dm \\ & = \sum_1^\infty \sum_1^k \int_{A_n} c_i \chi_{E_i} dm \\ & = \sum_1^\infty \int_{A_n} f dm \\ & = \sum_1^\infty \phi(A_n) \end{align*}

    Step 3. Let $g$ be a measurable simple function such that $0 \le g \le f$. Then \begin{align*} \int_A g dm & = \sum_1^\infty \int_{A_n} g dm \\ & \le \sum_1^\infty \int_{A_n} f dm \\ & = \sum_1^\infty \phi(A_n) \end{align*} Hence $\phi(A) = \int_A f dm \le \sum_1^{\infty} \phi(A_n)$.

    If $\phi(A_n) = \infty$ for any $n$, then we are done. Thus assume $\phi(A_n) < \infty$ for every $n$. Let $\epsilon > 0$, and choose measurable simple $g$ such that $0 \le g \le f$ and $\int_{A_1} g dm \ge \int_{A_1} f dm - \epsilon, \dots, \int_{A_n} g dm \ge \int_{A_n} f dm - \epsilon$. Hence $$\phi(A_1 \cup \dots \cup A_n) \ge \phi(A_1) + \dots + \phi(A_n) - n \epsilon$$Since $\epsilon$ was arbitrary, $\forall n$, $\phi(A_1 \cup \dots \cup A_n) \ge \phi(A_1) + \dots + \phi(A_n)$.
  \end{proof}

  \begin{corollary} If $A, B \in \mathfrak{M}$, $m(A \setminus B) = 0$, and $B \subseteq A$, then $$\int_A f dm = \int_B f dm$$ for every $f \in \sL$.
  \end{corollary}

  \begin{theorem} If $f \in \mathscr{L}$ on $E$, then $|f| \in \mathscr{L}$ on $E$ and $|\int_E f dm| \le \int_E |f| dm$.
  \end{theorem}
  \begin{proof}
    Let $A = \{ x \in E \mid f(x) \ge 0 \}$ and $B = \{ x \in E \mid f(x) < 0 \}$. Note that $E = A \sqcup B$ and $A, B \in \mathfrak{M}$. Then $$\int_E |f| dm = \int_A |f| dm + \int_B |f| dm = \int_E f^+ dm + \int_E f^- dm < \infty$$Thus $|f| \in \mathscr{L}$. Since $f \le |f|$ and $-f \le |f|$, $\int_E f dm \le \int_E f dm \le \int_E |f| dm$, and $\int_E -f dm = - \int_E f dm \le \int_E |f| dm$ so $$\left| \int_E f dm \right| \le \int_E |f| dm$$
  \end{proof}

  \begin{theorem} (Lebesgue's monotone convergence theorem). Let $E \in \mathfrak{M}$ and $\{ f_n \}$ a sequence of measurable functions such that $$0 \le f_1 (x) \le f_2(x) \le \dots \quad \forall (x \in E)$$Define $f(x) = \lim_{n \to \infty} f_n(x)$ for all $x \in E$. Then $$\int_E  f_n dm \to \int_E f dm \quad (n \to \infty)$$
  \end{theorem}
  \begin{proof}
    Since $\{ f_n \}$ is a monotone sequence of nonnegative measurable functions,  $\{ \int_E f_n dm \}$ is a monotone sequence of extended real numbers. Thus there must exist $\alpha \in \mathbb{R} \cup \{ \pm \infty \}$ such that $\alpha = \lim_{n \to \infty} \int_E f_n dm$. Since $f_n \le f$ for every $n$, $\alpha \le \int_E f dm$. 
    Let $0 < c < 1$ and $g$ be a simple, measurable function such that $0 \le g \le f$. For every $n \ge 1$, define $$E_n = \{ x \in E \mid f_n(x) \ge cg(x) \}$$Since $\{ f_n \}$ is increasing, $E_1 \subseteq E_2 \subseteq \dots$. Since $f_n \to f$ pointwise, $E = \bigcup_1^\infty E_n$. For every $n$, $cg \le f_n$ on $E_n$, so $$c \int_{E_n} g dm = \int_{E_n} cg dm \le \int_{E_n} f_n dm$$As $n \to \infty$, $$\int_{E_n} g dm \to \int_E g dm$$Therefore, $\alpha \ge c \int_{E} g dm$. Since $c < 1$ was arbitrary, $\alpha \ge \int_E g dm$. By definition of integration, $\alpha \ge \int_E  f dm$.
  \end{proof}

  \begin{theorem} Let $f = f_1 + f_2$, $f_1, f_2 \in \mathscr{L}$ on $E \in \mathfrak{M}$. Then $f \in \mathscr{L}$ on $E$ and $\int_E f dm = \int_E f_1 dm + \int_E f_2 dm$.
  \end{theorem}
  \begin{proof} If $f_1, f_2$ are simple measurable functions, then the conclusion is immediate. Assume that $f_1, f_2 \ge 0$. Choose a monotonically increasing sequence of nonnegative measurable simple functions $\{ g_n \}$ and $\{ h_n \}$ converging to $f_1$ and $f_2$ respectively. Let $s_n = g_n + h_n$. Then $\forall n$, $$\int_E s_n dm = \int_E g_n dm + \int_E h_n dm$$Note. $\{ s_n \}$ is a monotonically increasing sequence of simple nonnegative measurable functions converging to $f$. By the monotone convergence theorem, $$\int_E f dm = \lim_{n \to \infty} \int_E s_n dm = \lim_{n \to \infty} \int_E g_n dm + \int_E h_n dm = \int_E f_1 dm + \int_E f_2 dm$$ Now assume $f_1 \ge 0, f_2 < 0$. Define $$A = \{ x \in E \mid f(x) \ge 0 \} \quad \text{and} \quad B = \{ x \in E \mid f(x) < 0 \}$$ Note that both $A$ and $B$ are measurable. Since $f, f_1, -f_2 \ge 0$ on $A$ and $f_1 = f + (-f_2)$, $$\int_A f_1 dm = \int_A f dm + \int_A - f_2 dm = \int_A f dm - \int_A f_2 dm$$i.e., $\int_A f dm = \int_A f_1 dm + \int_A f_2 dm$.
Since $-f, f_1, -f_2 \ge 0$ on $B$, $$\int_B -f_2 dm = \int_B -f dm + \int_B f_1 dm$$ i.e., $\int_B f dm = \int_B f_1 dm + \int_B f_2 dm$. 
Hence \begin{align*} \int_{E = A \sqcup B}  f dm & = \int_A f dm + \int_B f dm \\ & = \int_A f_1 dm + \int_A f_2 dm + \int_B f_1 dm + \int_B f_2 dm \\ & = \int_E f_1 dm + \int_E f_2 dm  \end{align*} Let \begin{align*} E_1 & = \{ x \in E \mid f_1(x) \ge 0, f_2(x) \ge 0 \} \\ E_2 & = \{ x \in E \mid f_1(x) \ge 0, f_2(x) < 0 \} \\ E_3 & = \{ x \in E \mid f_1(x) < 0, f_2(x) \ge 0 \} \\ E_4 & = \{ x \in E \mid f_1(x) < 0, f_2(x) < 0 \} \end{align*} Apply what we've proven to all four sets and then we get the generalized conclusion.
\end{proof}

\begin{lemma} (Fatou's lemma) $E \in \mathfrak{M}$, $\{ f_n \}$ nonnegative measurable functions. Let $f = \liminf_{n \to \infty} f_n$. Then \begin{align*} \int_E f dm \le \liminf_{n \to \infty} \int_E f_n dm \end{align*}
\end{lemma}
\begin{proof} For every $n \ge 1$, define $$g_n = \inf_{m \ge n} f_n$$ Note. the $g_n$'s are measurable on $E$ and \begin{enumerate}
  \item $0 \le g_1 \le g_2 \le \dots$.
  \item $g_n \le f_n$, $\forall n$.
  \item  $\lim_{n \to \infty} g_n(x) = f(x)$, $\forall x \in E$.
  \end{enumerate}
  By the monotone convergence theorem, $$\lim_{n \to \infty} \int_E g_n dm = \int_E f dm$$ By property (ii), $$\int_E g_n dm \le \int_E f_n dm \quad \forall n$$ Together, these two imply the conclusion.
\end{proof}

\begin{theorem} (Dominated convergence theorem) Suppose $E \in \mathfrak{M}$, $\{ f_n \}$ measurable on $E$ such that $f_n \to f$ pointwise on $E$. Suppose $\exists g \in \mathscr{L}$ on $E$ such that $|f_n(x)| \le g(x)$ for all $x \in E$. Then \begin{align*} \int_E f dm = \lim_{n \to \infty} \int_E f_n dm \end{align*}
\end{theorem}
\begin{proof}
  Note $f_n \in \mathscr{L}$ on $E$ for all $n$ and $f \in \mathscr{L}$ on $E$. Since $f_n + g \ge 0$ for all $n$, applying Fatou's Lemma gives $$\int_E (f + g) dm \le \liminf_{n \to \infty} \int_E (f_n + g) dm$$ Then \begin{align*} \int_E f dm + \int_E g dm & \le \liminf_{n \to \infty} \left( \int_E f_n dm + \int_E g dm \right) \\ & = \left( \liminf_{n \to \infty} \int_E f_n dm \right) + \int_E g dm \end{align*} Thus $$\int_E f dm \le \liminf_{n \to \infty} \int_E f_n dm$$ Since $g - f_n \ge 0$, apply Fatou's Lemma to get $$\int_E (g - f) dm \le \liminf_{n \to \infty} \left( \int_E (g - f_n) dm \right)$$ By the same logic as above, we see that $$- \int_E f dm \le \liminf_{n \to \infty} - \int_E f_n dm$$ We conclude that $$\int_E f \ge \limsup_{n \to \infty} \int_E f_n dm$$ Thus $$\int_E f dm = \lim_{n \to \infty} \int_E f_n dm$$
\end{proof}

\begin{lemma} Nonmeasurable sets exist (assuming Axiom of Choice).
\end{lemma}
\begin{proof} For every $a \in [-1, 1]$ define $\tilde{a} = \{ c \in [-1, 1] : a - c \in \mathbb{Q} \}$. \\

  Claim 1. If $\tilde{a} \cap \tilde{b} \neq \emptyset$ then $\tilde{a} = \tilde{b}$. 

  Suppose $c \in \tilde{a} \cap \tilde{b}$. Then $a - c \in \mathbb{Q}$, $b - c \in \mathbb{Q}$, and therefore $a - b, b-a \in \mathbb{Q}$. Let $d \in \tilde{a}$, so $a - d \in \mathbb{Q}$. Then $a - d = (a - b) + (b - d)$ so $b -d \in \mathbb{Q}$, i.e., $d \in \tilde{b}$ and the claim follows. 

  Note. $[-1, 1] = \bigcup_{a \in [-1, 1]} \tilde{a}$. 
  Let $V$ be a set that contains exactly one element from every distinct $\tilde{a}$ (Axiom of Choice). Let $r_1, r_2, \dots$ be an enumeration of $\mathbb{Q} \cap [-2, 2]$. \\

  Claim 2. $[-1, 1] \subseteq \bigcup_{k = 1}^\infty V + r_k$. 

  Let $d \in [-1, 1]$, so $d \in \tilde{a}$ for some $a$. Let $c \in V$ s.t. $c \in \tilde{a}$. Then $c - d \in \mathbb{Q} \cap [-2, 2]$ so $c - d = r_k$ for some $k$. Hence, $d \in V + r_k$. 

  By Claim 2, $$2 = m^*([-1, 1]) \le m^* \left( \bigcup_1^\infty V + r_k \right) \le \sum_1^\infty m^*(V + r_k)= \sum_1^\infty m^*(V)$$ Thus $m^*(V) > 0$. \\

  Claim 3. $V+ r_1, V+ r_2, \dots$ are disjoint.

  Suppose ftsoc that $d \in (V + r_k) \cap (V + r_\ell)$. Then $d = v + r_k$, $v \in V$ and $d = v' + r_\ell$, $v' \in V$. In particular, $v - v' \in \mathbb{Q}$. By Claim 1, $v, v' \in \tilde{a}$. Contradiction. 

  For any $n \in \mathbb{N}$, $$\bigcup_{k = 1}^n V + r_k \subseteq [-3, 3]$$Hence, $$m^* \left(\bigcup_1^\infty V + r_k\right) \le 6$$Let $n \in \mathbb{N}$ such that $nm^*(V) > 6$. Then $$m^* \left(\bigcup_1^n V + r_k \right) < \sum_1^n m^*(V + r_k)$$Which implies that $V+r_1, V+r_2, \dots$ cannot all be measurable. Hence, $V$ is not measurable.
\end{proof}

\begin{definition} Let $E \in \mathfrak{M}$, $f$ measurable. We write $f \in \mathscr{L}^2$ on $E$ if $$\int_E |f|^2 dm < \infty$$
\end{definition}

\begin{remark} $f \in \mathscr{L}$ on $E$ $(\mathscr{L}^1)$ if $\int_E |f| dm < \infty$.
\end{remark}

\begin{example} $\text{}$
  \begin{enumerate}
  \item $E = (0, 1]$, $f(x) = x^{-1/2}$. $f \in \mathscr{L}^1, f \notin \mathscr{L}^2$.
  \item $E = (1, \infty)$, $f(x) = \frac{1}{x}$. $f \notin \mathscr{L}^1, f \in \mathscr{L}^2$.
  \end{enumerate}
\end{example}

\begin{theorem} If $m(E) < \infty$, then $f \in \mathscr{L}^2 \implies f \in \mathscr{L}^1$.
\end{theorem}

\newpage 

\section{Fourier Analysis}

Recall. Let $f: \mathbb{R} \to \mathbb{C}$. We can decompose $f$ into its real and imaginary components, $$f = f_{RE} + i f_{IM}$$where $f_{RE}, f_{IM}: \mathbb{R} \to \mathbb{R}$.

We say $f \in \sR$ (Riemann integrable) if $f_{RE}, f_{IM} \in \mathscr{R}$ and $$\int_{- \infty}^\infty f dx = \int_{- \infty}^\infty f_{RE} dx + i \int_{-\infty}^\infty f_{IM} dx$$

\begin{definition} A \emph{trigonometric polynomial} is a function $$f(x) = a_0 + \sum_1^N a_n \cos (nx) + b_n \sin (nx)$$ where $a_0, \dots, a_N, b_1, \dots, b_N \in \mathbb{C}$.
\end{definition}

Note. Using Euler's formula, we can equivalently write this as $$f(x) = \sum_{-N}^N c_n e^{inx}$$ where $c_{-N}, \dots, c_N \in \mathbb{C}$.

We discuss $2\pi$-periodic functions defined on intervals $[a, b]$ of length $2\pi$.

\begin{definition} Let $f \in \mathscr{R}$ on $[a, a + 2\pi]$, $n \in \mathbb{Z}$. The \emph{$n$-th Fourier coefficient of $f$} is \begin{align*} \hat{f} (n) = \frac{1}{2\pi} \int_a^{a + 2\pi} f(x) e^{-inx} dx.
\end{align*}
\end{definition}

\begin{definition} The \emph{Fourier series of $f$} is given (formally) by \begin{align*} f \sim \sum_{- \infty}^\infty \hat{f}(n) e^{inx}.
\end{align*}
\end{definition}

\begin{definition} The \emph{$N$-th partial sum of $f$} is \begin{align*}  s_N(f) = \sum_{-N}^N \hat{f}(n) e^{inx}.
\end{align*}
\end{definition}

Note.  If $n \in \mathbb{Z} - \{ 0 \}$, $e^{inx}$ is the derivative of $\frac{e^{inx}}{in}$ (which is $2\pi$-periodic). Therefore, \begin{align*} \frac{1}{2\pi} \int_a^{a+2\pi} e^{inx} dx = \begin{cases} 1 & \text{if } n = 0 \\ 0 & \text{if } n \neq 0 \end{cases}
\end{align*}

\begin{example} Suppose $f(x) = \sum_{-N}^N c_n e^{inx}$. Let $|m| \le N$. Then \begin{align*} \hat{f}(m) & = \frac{1}{2\pi} \int_{-\pi}^\pi f(x) e^{-imx} dx \\ & = \frac{1}{2\pi} \int_{-\pi}^\pi \left( \sum_{-N}^N c_n e^{inx} \right) e^{-imx} dx \\ & = \frac{1}{2\pi} \int_{-\pi}^\pi \left( \sum_{-N}^N c_n e^{ix(n - m)} \right) dx\\ & = \frac{1}{2\pi} \sum_{-N}^N c_n \int_{-\pi}^\pi e^{ix(n - m)} dx \\ & = \frac{1}{2\pi} (c_m 2\pi) \\ & = c_m.  \end{align*}

Note. If $|m| > N$ then $\hat{f}(m) = 0$.

Hence, $f(x) = \sum_{-\infty}^\infty \hat{f}(n) e^{inx} = \sum_{-N}^N \hat{f}(n) e^{inx} = s_N (f)$.
\end{example}

Question. In what sense does $s_N(f) \to f$ as $N \to \infty$?

\begin{example} Let $f(x) = x$ on $[-\pi, \pi]$. \begin{align*} \hat{f} (0) = 0 = \frac{1}{2\pi} \int_{-\pi}^\pi x dx.
\end{align*}
For $n \neq 0$, \begin{align*} \hat{f}(n) & = \frac{1}{2\pi} \int_{-\pi}^\pi xe^{-inx} \\ & = \frac{1}{2\pi} \left[ \frac{xe^{inx}}{in} \right]_{-\pi}^\pi + \frac{1}{2\pi in} \int_{-\pi}^\pi e^{-inx} dx \\ & = \frac{(-1)^{n + 1}}{in}.
\end{align*}
Fourier series of $f$ is \begin{align*} \sum_{n \neq 0} \frac{(-1)^{n + 1}}{in} e^{inx} = 2 \sum_1^\infty (-1)^{n + 1} \frac{\sin (nx)}{n}.
\end{align*}
In this example, $s_N(f) \to f$ uniformly.
\end{example}

Overview. \begin{enumerate}
\item Let $f$ be (Riemann) integrable in $[a, a + 2\pi]$. Does $s_N(f) \to f$ pointwise? NO
\item What about if $f$ is continuous (and periodic)? NO
\item What if $f \in C^1$ (and periodic)? YES
  
\end{enumerate}

Motivating question. If $f$ is $2\pi$-periodic, when can we prove that $s_N(f) \to f$ pointwise (uniformly)?

\begin{theorem} Suppose $f \in \mathscr{R}$ on $[0, 2\pi]$, $f$ is $2\pi$-periodic, $\hat{f}(n) = 0$ for all $n \in \mathbb{Z}$. Then $f(x) = 0$ $\forall x$ at which $f$ is continuous. 
\end{theorem}

\begin{corollary} If $f$ is continuous, $2\pi$-periodic, and $\hat{f}(n) =0$ $\forall n \in \mathbb{Z}$, then $f =0$.
\end{corollary}

\begin{corollary} If $f, g$ are continuous, $2\pi$-periodic, and $\hat{f}(n) = \hat{g}(n)$ $\forall n \in \mathbb{Z}$, then $f = g$.
\end{corollary}

\begin{corollary} Suppose $f$ is continuous, $2\pi$-periodic, and the Fourier series of $f$ converges absolutely, i.e., $$\sum_{-\infty}^\infty |\hat{f}(n) | < \infty.$$ Then $\lim_{N \to \infty} S_N(f) (x) = f(x)$ uniformly.
\end{corollary}
\begin{proof} Since $\sum_{-\infty}^\infty |\hat{f}(n) | < \infty$, the partial sums $S_N(f)$ converge uniformly. Define \begin{align*} g(x) = \sum_{-\infty}^\infty \hat{f}(n) e^{inx} = \lim_{N \to \infty} \sum_{-N}^N \hat{f}(n) e^{inx}.
\end{align*} Since $g$ is the uniform limit of continuous functions, $g$ is continuous. Moreover, $\forall n \in \mathbb{Z}$, \begin{align*} \hat{g}(n) & = \frac{1}{2\pi} \int_0^{2\pi} g(x) e^{-inx} dx \\ & = \frac{1}{2\pi} \int_0^{2\pi} \left( \sum_{-\infty}^\infty \hat{f} (m) e^{imx} \right) e^{-inx} dx \\ &  = \frac{1}{2\pi} \int_0^{2\pi} \left( \sum_{-\infty}^\infty \hat{f}(m) e^{ix(m - n)} \right) dx \\ & = \sum_{m = -\infty}^\infty \frac{1}{2\pi} \int_0^{2\pi} \hat{f} (m) e^{ix(m-n)} dx \\ & = \sum_{m = -\infty}^\infty \frac{1}{2\pi} \hat{f}(m) \int_0^{2\pi} e^{ix(m-n)} dx \\ & = \hat{f}(n).
\end{align*}
Hence $f = g$.
\end{proof}

\begin{lemma} Suppose $f$ is $C^2$ and $2\pi$-periodic. Then $\exists c > 0$ such that for all sufficiently large $|n|$, $$|\hat{f}(n)| \le \frac{c}{|n|^2},$$ i.e., $|\hat{f}(n)| = O\left(\frac{1}{n^2}\right)$.
\end{lemma}
\begin{proof} By integration by parts (twice), \begin{align*} 2 \pi \hat{f}(n) & = \int_0^{2\pi} f(x) e^{-inx} dx \\ & = f(x) \left[ \frac{e^{inx}}{in} \right]_{0}^{2\pi} + \frac{1}{in} \int_0^{2\pi} f'(x) e^{-inx} dx \\ & = \frac{1}{in} \left[ -f'(x) \frac{e^{-inx}}{in} \right]_0^{2\pi} + \frac{1}{(in)^2} \int_0^{2\pi} f''(x) e^{-inx} dx \\ & = - \frac{1}{n^2} \int_0^{2\pi} f''(x) e^{-inx} dx.
\end{align*} Hence, \begin{align*} 2 \pi \hat{f}(n) = \frac{1}{|n|^2} \left| \int_0^{2\pi} f''(x) e^{-inx} dx \right|. \tag{1}
\end{align*} Then \begin{align*} \mathrm{RHS} \text{ of } (1) & \le \frac{1}{|n|^2} \int_0^{2\pi} |f''(x)| |e^{-inx} dx \\ & = \frac{1}{|n|^2} \int_0^{2\pi} |f''(x)| dx \\ & \le \frac{1}{|n|^2} 2\pi c
\end{align*}
where $c = \max_{x \in [0, 2\pi]} |f''(x)|$. Therefore, $|\hat{f}(n)| \le \frac{c}{|n|^2}$.
\end{proof}

Note. We showed within the above proof that if $f$ is $C^1$, $\hat{f}'(n) = in \hat{f}(n)$.

Next question. If $f$ is  $2\pi$-periodic and $\int_0^{2\pi} |f|^2 dx$ exists, under what type of convergence does $s_N(f) \to f$?

\begin{theorem} Let $f$ be a complex valued, $2 \pi$-periodic, (Riemann) integrable function. Then \begin{align*} \lim_{N \to \infty} \int_0^{2\pi} |f(x) - s_N (f)(x) |^2 dx = 0.
\end{align*}
\end{theorem}

\begin{definition} A \emph{vector space} over $\mathbb{C}$ is a set $V$ of vectors, operations $\cdot, +$ such that $\forall x, y, z \in V$, $\forall \lambda_1, \lambda_2 \in \mathbb{C}$,
  \begin{enumerate}
  \item $x + y \in V$.
  \item $x + y = y + x$.
  \item $x + (y + z) = (x + y) + z$.
  \item $\lambda_1 x \in V$.
  \item $\lambda_1 (x + y) = \lambda_1 x + \lambda_1 y$.
  \item $(\lambda_1 + \lambda_2)x = \lambda_1 x + \lambda_2 x$.
  \item $\lambda_1(\lambda_2 x) = (\lambda_1 \lambda_2)x$.
  \end{enumerate}
  In addition, $\exists 0 \in V$ such that $x + 0 = x$ $\forall x$. $\forall x \in V, \exists (-x) \in V$ such that $x + (-x) = 0$. $\exists 1 \in V$ such that $1 \cdot x = x$.
\end{definition}

\begin{definition} An \emph{inner product} of a vector space $V$ is a map $(\cdot, \cdot): V \times V \to \mathbb{C}$ satisfying
  \begin{enumerate}
  \item $(x, y) = \overline{(y, x)}$.
  \item $(\alpha x + \beta y, z) = \alpha(x, z) + \beta(y, z)$.
  \item $(x, \alpha y + \beta z) = \overline{\alpha} (x, y) + \overline{\beta} (x, z)$.
  \item $(x, x) \ge 0$.
  \end{enumerate}
\end{definition}

\begin{definition} Given an inner product $(\cdot, \cdot)$, we can define a \emph{norm} on $V$, $$\| x \| = (x, x)^{\frac{1}{2}}.$$
\end{definition}

\begin{definition} We say that $x, y$ are \emph{orthogonal} if $(x, y) = 0$, and we write $x \perp y$.
\end{definition}

\begin{example} $V = \mathbb{C}$, $(x, y) = x \overline{y}$. 
\end{example}

\begin{example} $V = \mathbb{R}^n$, $(x, y) = x \cdot y$.
\end{example}

\begin{example} Let $\mathcal{R}$ be the set of complex-valued, $2\pi$-periodic (Riemann) integrable functions. This is a vector space over $\mathbb{C}$. \begin{enumerate}
  \item $(f + g)(x) = f(x) + g(x)$.
  \item $(\lambda f)(x) = \lambda f(x)$.
  \end{enumerate}
Define the inner product \begin{align*} (f, g) = \frac{1}{2\pi} \int_0^{2\pi} f(x) \overline{g(x)} dx
\end{align*}
so the norm is \begin{align*} \| f \| = \left( \frac{1}{2\pi} \int_0^{2\pi} |f(x)|^2 dx \right)^{\frac{1}{2}}.
\end{align*}
\end{example}

Three properties. Let $V$ be an inner product space. \begin{enumerate}
  \item Pythagorean Theorem. If $x \perp y$ then \begin{align*} \| x + y \|^2 = \| x\|^2 + \| y \|^2.
  \end{align*}
  \item Cauchy-Schwarz. For any $x, y \in V$, \begin{align*} |(x, y)| \le \| x \|  \| y \|.
  \end{align*}
  \item Triangle inequality. For any $x, y \in V$, \begin{align*} \| x + y \| \le \| x \|  + \| y \|.
  \end{align*}
\end{enumerate}

Notation. We will write $e_n(x) = e^{inx}$.

Observation. The family $\{ e_n \}_{n \in \mathbb{Z}}$ is \emph{orthonormal}, i.e., \begin{align*} (e_n, e_m) = \begin{cases} 1 & \text{if } n = m \\ 0 & \text{otherwise} \end{cases}
\end{align*} So $e_n \perp e_m$ if $n \neq m$ and $\| e_n \| = 1$ $\forall n \in \mathbb{Z}$. Moreover, $\forall f \in \mathcal{R}$, $n \in \mathbb{Z}$, \begin{align*} (f, e_n) & = \frac{1}{2\pi} \int_0^{2\pi} f(x) \overline{e_n(x)} dx \\ & = \frac{1}{2\pi} \int_0^{2\pi} f(x) e^{-inx} dx \\ & = \hat{f}(n).
\end{align*} Then \begin{align*} s_N (f) & = \sum_{-N}^N \hat{f}(n) e_n \\ & = \sum_{-N}^N (f, e_n)e_n.
\end{align*}

Note. $\forall |m| \le N$, \begin{align*} (f - s_N(f)) \perp e_m.
\end{align*} We see this since \begin{align*} (f - s_N(f), e_m) & = (f, e_m) - (s_N(f), e_m) \\ & = (f, e_m) - \sum_{-N}^N ((f, e_n)e_n, e_m) \\ & = (f, e_m) - \sum_{-N}^N (f, e_m)(e_n, e_m) \\ & = (f, e_m) - (f, e_m) \\ & = 0.
\end{align*}

\begin{corollary} For every $\{ e_n \}_{-N}^N$, \begin{align*} (f - s_N(f)) \perp \sum_{-N}^N c_n e_n.
\end{align*}
\end{corollary}

Then, $f = f - s_N(f) + s_N(f)$, so by the Pythagorean theorem, \begin{align*} \| f \|^2 = \| f - s_N(f) \|^2 + \| s_N(f) \|^2.
\end{align*}
$s_N(f) = \sum_{-N}^N \hat{f}(n) e_n$, so \begin{align*} \| s_N(f) \|^2 & = \sum_{-N}^N \| \hat{f}(n) e_n \|^2 \\ & = \sum_{-N}^N \| \hat{f}(n) \|^2
\end{align*} and thus \begin{align*} \| f \|^2 = \| f - s_N(f) \|^2 + \sum_{-N}^N \| \hat{f}(n) \|^2. \tag{2}
\end{align*}

\begin{lemma} (Best approximation) $f \in \mathcal{R}$. Then \begin{align*} \| f - s_N(f) \| \le \left \| f - \sum_{-N}^N c_n e_n  \right\|
\end{align*}
for any complex numbers $\{ c_n \}_{-N}^N$.
\end{lemma}

\begin{theorem} If $f \in \mathcal{R}$ then \begin{align*} \lim_{N \to \infty} \int_0^{2\pi} |f - s_N(f) |^2 dx = 0.
\end{align*}
\end{theorem}
\begin{proof} Let $f \in \mathcal{R}$ be continuous. By (a version of) the Stone-Weierstrass theorem, $\forall \epsilon > 0$, $\exists$ trigonometric polynomial $P$ such that $|f(x) - P(x)| < \epsilon$, $\forall x \in [0, 2 \pi]$. \begin{align*} \| f - P \| & = \left( \frac{1}{2\pi} \int_0^{2\pi} |f(x) - P(x) |^2 dx  \right)^{1/2} \\ & <  \left( \frac{1}{2\pi} \int_0^{2\pi} \epsilon^2 dx  \right)^{1/2} \\ & = \epsilon.
\end{align*} Let $M$ be the degree of $P$, i.e., $P = \sum_{-M}^M c_n e_n$. By the best approximation lemma, $\forall N \ge M$, \begin{align*} \| f - s_N(f) \| \le \| f - P \| < \epsilon.
\end{align*} Hence, $\forall \epsilon > 0$, $\exists M$ such that $\forall N \ge M$, $\| f - s_N(f) \| < \epsilon$.
Now we drop the condition that $f$ is continuous. For every $\epsilon > 0$, $\exists$ continuous $g$ such that \begin{enumerate}
\item $\sup_{x \in [0, 2 \pi]} |g(x)| \le \sup_{x \in [0, 2\pi]} |f(x)| = B$.
\item $\int_0^{2\pi} |f(x) - g(x) | dx < \epsilon^2$.
\end{enumerate} Then \begin{align*} \| f - g \| & = \left( \frac{1}{2\pi} \int_0^{2\pi} |f(x) - g(x) |^2 dx  \right)^{1/2} \\ & = \left( \frac{1}{2\pi} \int_0^{2\pi} |f(x) - g(x) | |f(x) - g(x) |dx  \right)^{1/2} \\ & \le \left( \frac{B}{\pi} \int_0^{2\pi} |f(x) - g(x) | dx \right)^{1/2} \\ & < \left( \frac{B}{\pi} \epsilon^2 \right)^{1/2} \\ & = \sqrt{\frac{B}{\pi}} \epsilon.
\end{align*} Since $g$ is continuous, $\exists$ trigonometric polynomial $P$ such that $\| g - P \| < \epsilon$. Therefore, \begin{align*} \| f - P \| & \le \| f - g \| + \| g - P \| \\ & < \epsilon \sqrt{ \frac{B}{\pi}} + \epsilon \\ & = \epsilon \left( 1 + \sqrt{ \frac{B}{\pi}} \right).
\end{align*} By the best approximation lemma, $\forall N \ge \mathrm{deg}(P)$, \begin{align*} \| f - s_N(f) \| < \epsilon \left( 1 + \sqrt{ \frac{B}{\pi}} \right).
\end{align*}
\end{proof}

\begin{corollary} (Parseval's Identity) $f \in \mathcal{R}$. Then \begin{align*} \sum_{-\infty}^\infty | \hat{f}(n) |^2 = \| f \|^2.
\end{align*}
\end{corollary}
\begin{proof} For every $n$, $\| f \|^2 \ge \sum_{-N}^N | \hat{f}(n) |^2$ by $(2)$. By the previous theorem, $\forall \epsilon > 0$, $\exists M$ such that $\forall N \ge M$, $\| f - s_N(f) \| < \epsilon$, so by $(2)$ again, \begin{align*} \sum_{-N}^N | \hat{f}(n) |^2 \ge \| f \|^2 - \epsilon.
\end{align*}
\end{proof}

\begin{corollary} (Riemann-Lebesgue) $f \in \mathcal{R}$. Then \begin{align*} \lim_{|n| \to \infty} | \hat{f}(n) | = 0.
\end{align*}
\end{corollary}

\newpage

\section{Differential Forms}

Recall. $f: E \to \bbR$, $E \subseteq \bbR^n$ open, partials $D_1f, \dots, D_nf$. If the partials are themselves differentiable then the second order derivatives of $f$ are defined by $$D_{ij}f = D_iD_jf, \quad (i, j = 1, \dots, n).$$ If these functions are continuous in $E$, we say $f$ is $C^2$ in $E$.

\begin{theorem} If $f \in C^2$ in $E$ then $$D_{ij}f = D_{ji}f, \quad \forall i, j.$$
\end{theorem}

\begin{definition}
  If $f: E \to \bbR^n$, $E \subseteq \bbR^n$ open, $f$ is differentiable at $x \in E$, the determinant of (the linear operator) $f'(x)$ is called the \emph{Jacobian of $f$ at $x$} $$J_f(x) = \mathrm{det} f'(x)$$
\end{definition}

Notation. We may also use $\frac{\partial (y_1, \dots, y_n)}{\partial (x_1, \dots, x_n)}$; $f(x_1, \dots, x_n) = y_1, \dots, y_n$.

\begin{definition} Let $k \in \bbN$. A \emph{$k$-cell} in $\bbR^k$ is the set of points $I^k = \{ x = (x_1, \dots, x_k) \}$ such that $a_i \le x_i \le b_i$, $\forall i = 1, \dots, k$.
\end{definition}

Suppose $I^k$ is a $k$-cell in $\bbR^k$ and $f: I^k \to \bbR$ is continuous. For every $j \le k$, let $I^j$ be the restriction of $I^k$ to the first $j$ components.\\

Define $g_k: I^k \to \bbR$ by $g_k= f$. Define $g_{k - 1}: I^{k - 1} \to \bbR$ by $$g_{k - 1}(x_1, \dots, x_{k-1}) = \int_{a_k}^{b_k}g_k(x_1, \dots, x_k) d x_k$$

Since $g_k$ is uniformly continuous on $I^k$, $g_{k - 1}$ is (uniformly) continuous on $I^{k - 1}$. Define $g_{k - 2}: I^{k - 2} \to \bbR$ by $$g_{k - 2}(x_1, \dots, x_{k - 2}) = \int_{a_{k - 1}}^{b_{k - 1}}g_{k - 1}(x_1, \dots, x_{k - 1}) dx_{k - 1}$$

We can repeat this process, ultimately arriving at a number $$g_0 = \int_{a_1}^{b_1} g_1(x_1) dx_1$$

We say $g_0$ is the integral of $f$ over $I^k$ and we write $$\int_{I^k} f(x) dx = g_0.$$

\begin{example}
  Let $I^2 = [1, 2] \times [0, 1]$, $f(x_1, x_2) = 2x_1x_2^2$. What is $\int_{I^2} f dx$? \\

  $$g_1(x_1) = \int_0^1 2x_1x_2^2 dx = \left[ \frac{2}{3} x_1x_2^3 \right]_0^1 = \frac{2}{3} x_1 $$

  $$\int_{I^2} f dx = g_0 = \int_1^2 g_1(x_1) dx_1 = \int_1^2 \frac{2}{3} x_1 dx_1 = \left[ \frac{1}{3} x_1^2 \right]_1^2 = 1$$
\end{example}

Question. Does this depend on the order of integration?

Answer. No (try the other direction in the example above).

\begin{definition}
  If $f : \bbR^k \to \bbR$, the \emph{support} of $f$ is the closure of the set $\{ x \in \bbR^k : f(x) \neq 0 \}$.
\end{definition}

If $f: \bbR^k \to \bbR$ is continuous with compact support, let $I^k$ be any $k$-cell containing $\mathrm{supp}(f)$. We define $$\int_{\bbR^k} f dx = \int_{I^k} f dx$$

\begin{theorem} (Change of variables) Let $T$ be a 1-1, $C^1$ mapping of $E \subseteq \bbR^n$ open to $\bbR^n$. Also assume $J_T(x) \neq 0$ for all $x \in E$. If $f$ is continuous on $\bbR^n$ with compact support that is contained in $T(E)$, then $$\int_{\bbR^n} f(y) dy = \int_{\bbR^n} f(T(x))|J_T(x)|dx.$$
\end{theorem}

\begin{definition} (Informal) A \emph{differential $1$-form} on $\bbR^n$ is \begin{enumerate}
  \item An object which can be integrated on any curve in $\bbR^n$.
  \item A rule assigning a real number to every oriented line segment in $\bbR^n$ in a ``suitable'' way.
  \end{enumerate}
\end{definition}

\begin{definition} Let $p \in \bbR^n$. The \emph{tangent space} to $\bbR^n$ at $p$ is $T_p \bbR^n = \{ (p, v) : v \in \bbR^n \}$.
\end{definition}

Notation. If $\alpha$ is a 1-form, $p \in \bbR^n$, write $\alpha_p$ to denote the restriction of $\alpha$ to $T_p \bbR^n$.- $\alpha_p(v)$ is the value $\alpha$ assigns to the (oriented) line segment from $p$ to $p + v$.

We require that $\alpha_p$ is a linear functional $\forall p \in \bbR^n$, that is \begin{enumerate}
\item $\alpha_p(tv) = t \cdot \alpha_p(v)$, $\forall t \in \bbR, \forall p, v \in \bbR^n$.
\item $\alpha_p(v + w) = \alpha_p(v) + \alpha_p(w)$, $\forall p, v, w \in \bbR^n$.
\end{enumerate}

We denote the projection maps in $\bbR^n$ by $dx_1, \dots, dx_n$, where $$dx_i(v) = dx_i(v_1, \dots, v_n) = v_i, \quad \forall i = 1, \dots, n$$

These form a basis for the set of linear functionals. Therefore, for any $1$-form $\alpha$, its restriction $\alpha_p$ can be written as \begin{align*} \alpha_p & = A_1dx_1 + A_2dx_2 + \dots + A_ndx_n \\ & = A_1(p) dx_1 + \dots + A_n(p)dx_n \end{align*}

Last requirement: $A_i(p)$ must be sufficiently continuous with respect to $p$.

\begin{definition} A \emph{differential $1$-form} $\alpha$ on $\bbR^n$ is a map from every tangent vector $(p, v)$ in $\bbR^n$ which can be expressed in the form $$\alpha = f_1dx_1 + \dots + f_ndx_n$$ where $f_i: \bbR^n \to \bbR$ is $C^2$.
\end{definition}

\begin{example} $\alpha = ydx + dz$ on $\bbR^3$. Let $p = \begin{pmatrix} 1 \\ 2 \\ 3 \end{pmatrix}$ and $v = \begin{pmatrix} 4 \\ 5 \\ 6 \end{pmatrix}$. Then \begin{align*} \alpha((p, v)) & = \alpha_p(v) \\ & = f_1(p) dx_1(v) + f_2(p) dx_2(v) + f_3(p) dx_3(v) \\ & = 2 \cdot 4 + 0 + 1 \cdot 6 \\ & = 14 \end{align*}
\end{example}

\begin{definition} A \emph{curve} (1-surface) in $\bbR^n$ is a $C^1$-mapping $\gamma: [a, b] \to \bbR^n$.
\end{definition}

\begin{definition} Let $\alpha = f_1dx_1 + \dots + f_ndx_n$ be a $1$-form in $\bbR^n$ and let  $\gamma: [a, b] \to \bbR^{n}$ be $C^1$. \begin{align*} \int_{\gamma} \alpha = \int_a^b (f_1 (\gamma(t))\gamma_1'(t) + \dots + f_n (\gamma(t)) \gamma_n'(t)) dt \end{align*}
\end{definition}

\begin{example} $\alpha = x^2dx_1 + dx_2$ on $\bbR^2$. $\gamma(t) = (t, t^2)$, $t \in [0, 1]$. Then $\gamma_1' (t) = 1$, $\gamma_2'(t) = 2t$. \begin{align*} \int_{\gamma} \alpha & = \int_0^1 (f_1(\gamma(t)) \gamma_1' (t) + f_2 (\gamma(t)) \gamma_2'(t)) \\ & =\int_a^b (t^2 \cdot 1 +1  \cdot 2t) dt \\ & = \frac{4}{3} \end{align*}
\end{example}

\begin{definition} A \emph{$2$-surface} is a $C^1$ map $\gamma : I^2 \to \bbR^n$.
\end{definition}

\begin{definition} (Informal) A \emph{$2$-form} on $\bbR^n$ is
  \begin{enumerate}
  \item An object which can be integrated over any $2$-surface.
  \item A rule which assigns a real number to every oriented parallelogram in $\bbR^n$ in a ``suitable'' way.
  \end{enumerate}
\end{definition}

Specify an oriented parallelogram in $\bbR^n$ based at $p \in \bbR^n$ by giving $(v, w)$. We want every $2$-form $\omega$ to satisfy the following for every $p \in \bbR^n$
\begin{enumerate}
\item $\omega_p(tv_1, v_2) = \omega_p(v_1, tv_2) = t \omega_p(v_1, v_2)$.
\item $\omega_p(v_1, v_2 + v_3) = \omega_p (v_1, v_2) + \omega_p(v_1, v_3)$ and $\omega_p(v_1 + v_2, v_3) = \omega_p(v_1, v_3) + \omega_p(v_2, v_3)$.
\item $\omega_p(v_1, v_2) = - \omega_p(v_2, v_1)$.
\end{enumerate}

Basic $2$-forms on $\bbR^n$. $\forall v, w \in \bbR^n$,
\begin{enumerate}
\item $(dx_1 \wedge dx_2)(v, w) = \det \begin{pmatrix} v_1 & w_1 \\ v_2 & w_2 \end{pmatrix}$.
\item $(dx_1 \wedge dx_3)(v, w) = \det \begin{pmatrix} v_1 & w_1 \\ v_3 & w_3 \end{pmatrix}$.
\item $(dx_i \wedge dx_j)(v, w) = \det \begin{pmatrix} v_i & w_i \\ v_j & w_j \end{pmatrix}$.
\end{enumerate}

\begin{remark} If $\omega_p$ satisfies $(i) - (iii)$ then $\omega_p$ can be expressed as $$\omega_p = \sum_{i, j} A_{i, j}(p) (dx_i \wedge dx_j)$$ for constant $A_{i, j}$.
\end{remark}

\begin{definition} A \emph{$2$-form} in $\bbR^n$ is a rule assigning a real number to each oriented parallelogram in $\bbR^n$ that can be written as $$\omega = \sum_{i, j} f_{i, j} (dx_i \wedge dx_j)$$ where $f_{i, j}: \bbR^n \to \bbR$ is $C^2$. \\

For any $p \in \bbR^n$, $v, w \in \bbR^n$, $$\omega_p(v, w) = \sum_{i, j} f_{i, j}(p) (dx_i \wedge dx_j)(v, w).$$
\end{definition}

\begin{example} $\omega$ is a $2$-form in $\bbR^2$, \begin{align*} \omega & = f_{1,1} \underbrace{(dx_1 \wedge dx_1)}_{= 0} + f_{1, 2} (dx_1 \wedge dx_2) + f_{2, 1} \underbrace{(dx_2 \wedge dx_1)}_{= - (dx_1 \wedge dx_2)} + f_{2, 2} \underbrace{(dx_2 \wedge dx_2)}_{= 0} \\ & = (f_{1, 2} - f_{2, 1})(dx_1 \wedge dx_2) \end{align*}
  This implies that every $2$-form in $\bbR^2$ can be written as $\omega = f(dx_1 \wedge dx_2)$ where $f$ is $C^2$.
\end{example}

\begin{example} $\omega$ is a $2$-form in $\bbR^3$, \begin{align*} \omega = f_1 (dx_1 \wedge dx_2) + f_2 (dx_1 \wedge dx_3) + f_3 (dx_2 \wedge dx_3). \end{align*}
\end{example}

\begin{definition} Let $\gamma : I^2 \to \bbR^3$ be $C^1$, and $\omega = f_1 (dx_1 \wedge dx_2) + f_2 (dx_1 \wedge dx_3) + f_3 (dx_2 \wedge dx_3)$ be a $2$-form. Then \begin{align*} \int_{\gamma} \omega & = \int_{I^2} \omega_{\gamma(z)} \left( \frac{\partial \gamma}{\partial x_1}(z), \frac{\partial \gamma}{\partial x_2} (z) \right) dz \\ & = \int_{I^2}  f_1(\gamma (z)) (dx_1 \wedge dx_2) \left( \frac{\partial \gamma}{\partial x_1}(z), \frac{\partial \gamma}{\partial x_2} (z) \right) \\ & \qquad + f_2(\gamma(z))(dx_1 \wedge dx_3) \left( \frac{\partial \gamma}{\partial x_1}(z), \frac{\partial \gamma}{\partial x_2} (z) \right) + f_3 (\gamma (z)) (dx_2 \wedge dx_3) \left( \frac{\partial \gamma}{\partial x_1}(z), \frac{\partial \gamma}{\partial x_2} (z) \right) dz \\ & = \int_{I^2} f_1(\gamma(z)) \det \begin{pmatrix} D_1\gamma_1(z) & D_2 \gamma_1 (z) \\ D_1 \gamma_2(z) & D_2 \gamma_2(z) \end{pmatrix}\\ & \qquad +  f_2 (\gamma(z)) \det \begin{pmatrix} D_1\gamma_1(z) & D_2 \gamma_1 (z) \\ D_1 \gamma_3(z) & D_2 \gamma_3(z) \end{pmatrix} +  f_3(\gamma(z)) \det \begin{pmatrix} D_1\gamma_2(z) & D_2 \gamma_2 (z) \\ D_1 \gamma_3(z) & D_2 \gamma_3(z) \end{pmatrix} \end{align*}
\end{definition}

\begin{definition} The integral of a $2$-form $\omega = \sum_{i, j} f_{i, j} (dx_i \wedge dx_j)$ over a $2$-surface $\gamma : [a, b] \times [c, d] \to \bbR^n$ (which is $C^1$) is \begin{align*} \int_{\gamma} \omega = \int_a^b \left( \int_c^d \omega_{\gamma(t_1, t_2)} \left( \frac{\partial \gamma}{\partial t_1}, \frac{\partial \gamma}{\partial t_2} \right)  dt_2 \right) dt_1 \end{align*}
\end{definition}

\begin{definition} A \emph{$k$-surface} in $\bbR^n$ is a $C^1$ map $\gamma: D \to \bbR^n$ where $D$ is a $k$-cell.
\end{definition}

\begin{definition} (Informal) A $k$-form in $\bbR^n$, $\omega$, is a rule that assigns a real number to every oriented $k$-dimensional parallelepiped in $\bbR^n$ in a ``suitable'' way.
\end{definition}

Specify a $k$-dimensional oriented parallelepiped in $\bbR^n$ based at $p \in \bbR^n$ by giving an ordered list of vectors $v_1, \dots, v_k \in T_p\bbR^n$. We require that for any $p \in \bbR^n$, a $k$-form $\omega$ satisfies
\begin{enumerate}
\item $\omega_p(v_1, \dots, tv_i, \dots,  v_k) = t \omega_p (v_1, \dots, v_i, \dots,  v_k)$.
\item $\omega_p(v_1, \dots, v_i + w_i, \dots, v_k) = \omega(v_1, \dots, v_i, \dots, v_k) + \omega_p (v_1, \dots, w_i, \dots v_k)$.
\item $\omega_p(v_1, \dots, v_i, \dots, v_j, \dots, v_k) = - \omega_p(v_1, \dots, v_j, \dots, v_i, \dots, v_k)$.
\end{enumerate}

\begin{definition} A \emph{multi-index} of length $k$ in $\bbR^n$ is a list $I = (i_1, \dots, i_k)$ of $k$ integers between $1$ and $n$.
\end{definition}

\begin{definition} Let $I = (i_1, \dots, i_k)$ be a multi-index. Then $dx_I = dx_{i_1} \wedge \dots \wedge dx_{i_k}$ is the $k$-form in $\bbR^n$ defined by \begin{align*} dx_I (v^1 , \dots, v^k) & = \det \begin{pmatrix} v_{i_1}^1 & v_{i_1}^2 & \hdots & v_{i_1}^k \\ v_{i_2}^1 & v_{i_2}^2 & \hdots & v_{i_2}^k \\ \vdots & \vdots & \ddots & \vdots \\ v_{i_k}^1 & v_{i_k}^2 & \hdots & v_{i_k}^k \end{pmatrix} \end{align*}
\end{definition}

\begin{remark} $\text{}$
  \begin{enumerate}
  \item If $I$ contains a repeated index, then $dx_I (v^1, \dots, v^k) = 0$.
  \item For any $I$, if $v^1, \dots, v^k$ contains a repeated vector, then $dx_I (v^1, \dots, v^k) = 0$. 
  \item If $J$ is obtained from $I$ by swapping a single pair of indices, then $dx_I (v^1, \dots, v^k) = - dx_J (v^1, \dots, v^k)$.
  \end{enumerate}
\end{remark}

\begin{definition} A \emph{differential $k$-form} in $\bbR^n$, $\omega$, is a rule assigning a real number to each oriented parallelepiped of the form \begin{align*} \omega = \sum_I f_I dx_I \end{align*} where the sum is taken over all multi-indices $I$ of length $k$ and $f_I: \bbR^n \to \bbR$ is $C^2$. If $p \in \bbR^n$, $v^1, \dots, v^k \in \bbR^n$, \begin{align*} \omega_p (v^1, \dots, v^k) = \sum_I f_I (p) dx_I (v^1, \dots, v^k) \end{align*}
\end{definition}

\begin{definition} Let $\phi: D \to \bbR^n$ be a $k$-surface and $\omega = \sum_I f_I dx_I$ be a $k$-form. \begin{align*} \int_{\phi} \omega & = \int_D \omega_{\phi(u)} \left( \frac{\partial \phi}{\partial u_1} , \dots, \frac{\partial \phi}{\partial u_k} \right) du \\ & = \int_D \sum_I f_I(\phi(u)) dx_I \left( \frac{\partial \phi}{\partial u_1} , \dots, \frac{\partial \phi}{\partial u_k} \right) du \\ & = \int_D \sum_I f_I (\phi(u)) \frac{\partial(x_{i_1}, \dots, x_{i_k})}{\partial (u_1, \dots, u_k)} du \end{align*} where $\frac{\partial(x_{i_1}, \dots, x_{i_k})}{\partial (u_1, \dots, u_k)}$ is the Jacobian of the map $u_1, \dots, u_k \mapsto \phi_{i_1}(u), \dots, \phi_{i_k}(u)$.
\end{definition}

\begin{example} $\omega = xdy \wedge dz - y dx \wedge dz + z dx \wedge dy$ is a $2$-form in $\bbR^3$. $\phi: [0, 3] \times [0, 2 \pi] \to \bbR^3$, $\phi(r, \theta) = (r \cos \theta, r \sin \theta, 5)$. \begin{align*}
\end{example}
  
\begin{definition} If $I = (i_1, \dots, i_k)$ is a multi-index and $i_1 < \dots < i_k$, we say $I$ is an \emph{increasing multi-index}. We say that $dx_I$ is a basic $k$-form.
\end{definition}

\begin{remark} Every $k$-form can be represented in terms of basic $k$-forms.
\end{remark}

\begin{example} $dx_1 \wedge dx_5 \wedge dx_3 \wedge dx_2 = - dx_1 \wedge dx_2 \wedge dx_3 \wedge dx_5$.
\end{example}

\begin{example} $dx_1 \wedge dx_3 \wedge dx_5 \wedge dx_2 = dx_1 \wedge dx_2 \wedge dx_3 \wedge dx_5$.
\end{example}

\begin{definition} If $\omega = \sum_I a_I dx_I$ is a $k$-form, we can convert each multi-index $I$ into an increasing multi-index $J$, and we say that \begin{align*} \omega = \sum_J b_J dx_J
\end{align*} is in \emph{standard presentation}.
\end{definition}

\begin{example} \begin{align*} \omega & = x_1 dx_2 \wedge dx_1 - x_2dx_3 \wedge dx_2 + x_3 dx_2 \wedge dx_3 + dx_1 \wedge dx_2 \\ & = -x_1 dx_1 \wedge dx_2 + x_2 dx_2 \wedge dx_3 + x_3 dx_2 \wedge dx_3 + dx_1 \wedge dx_2 \\ & = (1 - x_1) dx_1 \wedge dx_2 + (x_2 + x_3) dx_2 \wedge dx_3.
\end{align*} The last line is in standard presentation.
\end{example}

\begin{definition} Suppose $I = (i_1, \dots, i_p)$ and $J = (j_1, \dots, j_q)$ are increasing multi-indices. The \emph{product} of $dx_I$ and $dx_J$ is the $(p + q)$-form \begin{align*} dx_I \wedge dx_J & = dx_{i_1} \wedge dx_{i_2} \wedge \dots \wedge dx_{i_p} \wedge dx_{j_1} \wedge \dots \wedge dx_{j_q}.
\end{align*}
\end{definition}

Note. If $I$ and $J$ have an element in common, $dx_I \wedge dx_J = 0$.

Notation. If $I$ and $J$ have no elements in common, we denote the increasing $(p + q)$ length multi-index obtained from rearranging the members of $I \cup J$ in increasing order by $[I, J]$. \begin{align*} dx_I \wedge dx_J = (-1)^{\alpha} dx_{[I, J]}
\end{align*} where $\alpha$ is the number of swaps needed to convert $I \cup J$ into an increasing multi-index.

Suppose $\omega, \lambda$ are $p$ and $q$-forms respectively in $\bbR^n$ with standard representations \begin{align*} \omega = \sum_I b_I dx_I \quad \lambda = \sum_J c_J dx_J.
\end{align*} The product of $\omega$ and $\lambda$ is the $(p + q)$-form \begin{align*} \omega \wedge \lambda & = \sum_{I, J} b_I c_I (dx_I \wedge dx_J).
\end{align*}

\begin{remark} $\text{}$ \begin{enumerate}
  \item $(\omega_1 + \omega_2) \wedge \lambda = (\omega_1 \wedge \lambda) + (\omega_2 \wedge \lambda)$
  \item $\omega \wedge (\lambda_1 + \lambda_2) = (\omega \wedge \lambda_1) + (\omega \wedge \lambda_2)$
    \item $(\omega \wedge \lambda) \wedge \sigma = \omega \wedge (\lambda \wedge \sigma)$
  \end{enumerate}
\end{remark}

\begin{definition} A $0$-form is a $C^1$ function.
\end{definition}

Notation. The product of a $0$-form $f$ with a $k$-form $\omega = \sum_I b_I dx_I$ is \begin{align*} f \omega = \omega f = \sum_I (fb_I) dx_I.
\end{align*}

\begin{remark} $f ( \omega \wedge \lambda) = f \omega \wedge \lambda = \omega \wedge f \lambda $.
\end{remark}

\begin{definition} (Differentiation of $k$-forms) Operator which associates a $(k + 1)$-form, $d \omega$, to each $k$-form, $\omega$.
\begin{enumerate}
  \item $0$-forms in $\bbR^n$. $f: E \to \bbR$, $E \subseteq \bbR^n$. \begin{align*} df & = D_1 fdx_1 + \dots + D_n f dx_n  \\ & = \frac{\partial f}{\partial x_1} dx_1 + \dots + \frac{\partial f}{\partial x_n} dx_n.
  \end{align*}
  \item $k$-forms in $\bbR^n$. Let $\omega = \sum_I b_I dx_I$ be given in standard presentation. \begin{align*} d \omega & = \sum_I (db_I) \wedge dx_I.
  \end{align*}
\end{enumerate}
\end{definition}

\begin{example} Let $\omega = \underbrace{x}_{f_1} dx + \underbrace{y^2}_{f_2} dz$ be a $1$-form in $\bbR^3$. \begin{align*} d \omega & = (df_1) \wedge dx + (df_2) \wedge dz \\ & = (1 dx + 0 dy + 0 dz) \wedge dx + (0dx + 2y dy + 0 dz) \wedge dz \\ & = dx \wedge dx + 2y dy \wedge dz \\ & = 2y dy \wedge dz.
\end{align*} Further, \begin{align*} d(d\omega) & = d (2y dy \wedge dz) \\ & = (df) \wedge (dy \wedge dz) \\ & = (2dy) \wedge (dy \wedge dz) \\ & = 0.
\end{align*}
\end{example}
  
\end{document}
