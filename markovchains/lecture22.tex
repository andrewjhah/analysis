\documentclass[11pt]{article}
\usepackage{master}
\title{MATH 235 Lec 22}
\author{Andrew Hah}

\begin{document}

\pagestyle{plain}
\begin{center}
{\Large MATH 23500} \\
{\Large Lecture 22} \\
\vspace{.2in}
May 14, 2025
\end{center}

\begin{remark} Suppose $\{ X_n \}_{n \in \bbN}$ is a discrete stochastic process, and $A \subseteq S$ has $\bbP (X_n \in A) = 0$ for all $n$. Then $\bbP (\exists n \text{ such that } X_n \in A) \le \sum_{n = 0}^{\infty} \bbP (X_n \in A) = 0$. This is not true for Brownian motion. $\bbP (B_t = 1) = 0$, while $\bbP (\exists t \text{ such that } B_t = 1) = 1$.
\end{remark}

\begin{example} Suppose $B$ is a standard Brownian motion. \begin{enumerate}
    \item Find $\bbE [B_4 \mid B_2 = 6]$. Note $B_4 = (B_4 - B_2) + B_2$. Thus we seek \begin{align*} \bbE [(B_4 - B_2) + B_2 \mid B_2 = 6] & = \bbE [B_4 - B_2 \mid B_2 = 6] + \bbE [B_2 \mid B_2 = 6] \\ & = 0 + 6 \\ & = 6.
    \end{align*}
    \item Find $\bbE [B_2^2 B_t^2]$ for $s \le t$. Note \begin{align*} B_t^2 & = ((B_t - B_s) + B_s)^2 \\ & = B_{t - s}^2 + 2B_sB_{t - s} + B_s^2.
    \end{align*} Thus, \begin{align*} \bbE [B_s^2 B_t^2] & = \bbE [B_s^2 B_{t - s}^2] + \bbE [2B_s^3B_{t - s}] + \bbE [B_s^4] \\ & = s(t - s) + 3s^2.
    \end{align*}
  \end{enumerate}
\end{example}

Assume $X \sim \mathcal{N}(0, \sigma^2)$. The moment generating function of $X$ is given by \begin{align*} M_X(t) & \coloneqq \bbE [ e^{tX}] \\ & = e^{\frac{\sigma^2 t^2}{2}}.
\end{align*} The left hand side is \begin{align*} \bbE [e^{tX}] & = \sum_{k = 0}^{\infty} \frac{\bbE [(tx)^k]}{k!} \\ & = \sum_{k = 0}^{\infty} \frac{\bbE [X^k]}{k!} t^k.
\end{align*} On the other hand, the right hand side is \begin{align*} e^{\frac{\sigma^2 t^2}{2}} & = \sum_{k = 0}^{\infty} \frac{\left( \frac{\sigma^2 t^2}{2} \right)^k}{k!} \\ & = \sum_{k = 0}^{\infty} \frac{\sigma^{2k}}{2^kk!} t^{2k}.
\end{align*} Comparing the coefficients of $t^{2n}$, we find that \begin{align*} \frac{\bbE[X^{2n}]}{2n!} = \frac{\sigma^{2n}}{2^nn!} \implies \bbE [X^{2n}] = \frac{\sigma^{2n}(2n)!}{2^n n!} = (2n-1)!! \sigma^{2n}.
\end{align*}

\begin{example} Suppose $B$ is a standard Brownian motion. Find $\bbP (B_2 > B_1 > B_3)$. Let $X = B_1, Y = B_2 - B_1, Z = B_3 - B_2$. We seek \begin{align*} \bbP (X + Y > X > X + Y + Z) & = \bbP (Y > 0 > Y + Z) \\ & = \bbP (B_1 > 0, B_2 < 0) \\ & = \int_0^{\infty} \bbP (B_2 < 0 \mid B_1 = x) \frac{1}{\sqrt{2 \pi}} e^{-x^2/2} dx.
\end{align*} Note \begin{align*} \bbP (B_2 < 0 \mid B_1 = x) & = \bbP (B_2 < -x \mid B_1 = 0) \\ & = \bbP (B_2 > x \mid B_1 = 0).
\end{align*} Thus, the integral becomes \begin{align*} \int_0^{\infty} \left( \int_x^{\infty} \frac{1}{\sqrt{2\pi}} e^{-y^2/2} dy \right) \frac{1}{\sqrt{2\pi}} e^{-x^2/2} dx &  = \frac{1}{2\pi} \int_0^{\infty} \int_x^{\infty} e^{- \left( \frac{x^2 + y^2}{2} \right)} dy dx \\ & = \int_0^{\infty} \int_{\pi / 4}^{\pi / 2} re^{-r^2 /2} d \theta d r \\ & = \frac{1}{8}.
\end{align*}
\end{example}

\end{document}
