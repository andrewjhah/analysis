\documentclass[11pt]{article}
\usepackage{master}

\title{Math Proseminar Notes}
\author{Andrew Hah}
\begin{document}

\pagestyle{plain}
\begin{center}
{\Large MATH 297. Proseminar in Mathematics} \\
\vspace{.2in}  
Andrew Hah \\
Spring 2025
\end{center}

\begin{theorem} (Multicolor triangle Ramsey theorem) For every positive integer $r$, there is some integer $N = N(r)$ such that if each edge of $K_N$ is colored using on of $r$ colors, then there is a monochromatic triangle.
\end{theorem}
\begin{proof} Zhao, GTAC, pg 3.
\end{proof}

\begin{theorem} (Graph Ramsey theorem) For every $k$ and $r$ there exists some $N = N(k, r)$ such that if each edge of $K_N$ is colored using one of $r$ colors, then there is a monochromatic $K_k$.
\end{theorem}

Question. What is the maximum number of edges in a triangle-free $n$-vertex graph?

\begin{definition} (Turán number) We write $\mathrm{ex} (n, H)$ for the maximum number of edges in an $n$-vertex $H$-free graph, where a graph is $H$-free if it does not contain $H$ as a subgraph.
\end{definition}

\begin{theorem} (Mantel's theorem) Every $n$-vertex triangle-free graph has at most $\lfloor n^2 /4 \rfloor$, i.e., $\mathrm{ex}(n, K_3) = \lfloor n^2 /4 \rfloor$.
\end{theorem}
\begin{proof} Proof ii, Zhao, GTAC, pg 13.
\end{proof}

\begin{exercise} Let $X$ and $Y$ be independent and identically distributed random vectors in $\bbR^d$ according to some arbitrary probability distribution. Prove that \begin{align*} \bbP (| X + Y | \ge 1) \ge \frac{1}{2} \bbP (|X| \ge 1)^2.
\end{align*}
\end{exercise}

\begin{definition} The Turán graph $T_{n, r}$ is defined to be the complete $n$-vertex $r$-partite graph with part sizes differing by at most $1$ (so each part has size $\lfloor n/ r \rfloor$ or $\lceil n / r \rceil$.
\end{definition}

\begin{example} $T_{10, 3} = K_{3, 3, 4}$.
\end{example}

\begin{theorem} (Turán's theorem) The Turán graph $T_{n, r}$ maximizes the number of edges among all $n$-vertex $K_{r+1}$-free graphs. It is also the unique maximizer.
\end{theorem}

\begin{corollary} $\mathrm{ex} (n, K_{r + 1}) \le \left( 1 - \frac{1}{r} \right) \frac{n^2}{2}$. 
\end{corollary}

\begin{definition} The edge density of a graph $G$ is \begin{align*} \frac{e(G)}{\binom{v(G)}{2}}.
\end{align*}
\end{definition}

\begin{proposition} (Monotonicity of Turán numbers) For every graph $H$ and positive integer $n$, \begin{align*} \frac{\mathrm{ex} (n + 1, H)}{\binom{n + 1}{2}} \le \frac{\mathrm{ex}(n, H)}{\binom{n}{2}}.
\end{align*}
\end{proposition}

For every fixed $H$, the sequence $\mathrm{ex}(n, H)} / \binom{n}{2}$ is nonincreasing and bounded between $0$ and $1$. It follows that it approaches a limit.

\begin{definition} The \emph{Turán density} of a graph $H$ is defined to be \begin{align*} \pi(H) = \lim_{n \to \infty} \frac{\mathrm{ex}(n, H)}{n \choose 2}.
\end{align*}
\end{definition}

\end{document}
