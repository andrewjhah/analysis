\documentclass[11pt]{article}
\usepackage{master}

\title{Entropy and Graphons}
\author{Andrew Hah}
\date{\vspace{-.75em}Spring 2025}

\begin{document}

\maketitle

\begin{abstract}
This paper began with my curiosity about how ideas from information theory and analysis can be used to understand the behavior of large combinatorial structures. In particular, I explored two tools in depth: entropy, which captures uncertainty and structure, and graphons, which describe the limiting behavior of dense graphs. This paper surveys what I’ve learned so far. Beginning with entropy in finite combinatorics, I then introduce the theory of graph limits and graphons, and finally explore how entropy extends to this setting and informs recent results.
\end{abstract}

\tableofcontents

\section{Extremal Graph Theory}

\subsection{Classical Results}

\subsection{Homomorphism Densities}
Let \( G \) be a finite simple graph with vertex set \( V(G) \) and edge set \( E(G) \). An important question in many areas of graph theory is: how often does a fixed small graph \( H \) appear inside \( G \)?

To quantify this, we define the \emph{homomorphism density} of a graph \( H \) in a graph \( G \), denoted \( t(H, G) \), as the probability that a uniformly random map \( \phi : V(H) \to V(G) \) is a graph homomorphism; that is, for every edge \( \{u, v\} \in E(H) \), the image \( \{\phi(u), \phi(v)\} \in E(G) \). Formally,
\[
t(H, G) := \frac{|\mathrm{Hom}(H, G)|}{|V(G)|^{|V(H)|}},
\]
where \( \mathrm{Hom}(H, G) \) is the set of all homomorphisms from \( H \) to \( G \).

This quantity reflects the “density” of \( H \) in \( G \). For example, if \( H \) is the triangle \( K_3 \), then \( t(K_3, G) \) captures how likely it is for three randomly chosen vertices in \( G \) to form a triangle, when accounting for all mappings.

The homomorphism density is closely related to induced subgraph counts and plays a fundamental role in extremal graph theory. Many classical results — such as Turán’s Theorem — can be rephrased in terms of maximizing or minimizing \( t(H, G) \) under certain constraints.

In what follows, we will use homomorphism densities as a lens for understanding graph sequences, graph parameters, and the eventual need for limiting objects such as graphons.

\section{Entropy in Combinatorics}

\section{Motivating Graph Limits}

\section{Graphons}

\section{Entropy of Graphons}

\section{Flag Algebras}

\section{Open Questions}

\section*{Acknowledgments}

\section*{References}


\end{document}
