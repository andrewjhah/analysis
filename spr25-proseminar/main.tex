\documentclass[11pt]{article}
\usepackage{master}
\usepackage[backend=biber,style=alphabetic]{biblatex}
\addbibresource{refs.bib}

\title{Entropy and Graphons}
\author{Andrew Hah}
\date{\vspace{-.75em}Spring 2025}

\begin{document}

\maketitle

\begin{abstract}
This paper began with my curiosity about how ideas from information theory and analysis can be used to understand the behavior of large combinatorial structures. In particular, I explored two tools in depth: entropy, which captures uncertainty and structure, and graphons, which describe the limiting behavior of dense graphs. This paper surveys what I’ve learned so far. Beginning with entropy in finite combinatorics, I then introduce the theory of graph limits and graphons, and finally explore how entropy extends to this setting and informs recent results.
\end{abstract}

\tableofcontents

\newpage

\section{Extremal Graph Theory}

\subsection{Classical Results}

A classical question in extremal graph theory is of the form, given a fixed forbidden subgraph \(H\), what is the maximum number of edges a graph on \(n\) vertices can have without containing \(H\) as a (not necessarily induced) subgraph? This number is called the Turán number, denoted $\mathrm{ex}(n, H)$. What is often considered the first result in extremal graph theory is Mantel's theorem, which answers this question for the special case when $H = K_3$ is a triangle.

\begin{theorem} (Mantel's theorem) Every $n$-vertex triangle-free graph has at most $\lfloor n^2 /4 \rfloor$ edges, i.e., $\mathrm{ex}(n, K_3) = \lfloor n^2 /4 \rfloor$.
\end{theorem}
\begin{proof} See \cite[Theorem~1.1.1]{gtac2023} for two different proofs, or the original argument in \cite{mantel1907}.
\end{proof}

We would like to generalize this theorem from triangles to arbitrary cliques. To do so, we first construct the Turán graph.

\begin{definition} The \emph{Turán graph} $T_{n, r}$ is defined to be the complete $n$-vertex $r$-partite graph with part sizes differing by at most $1$ (so each part has size $\lfloor n/ r \rfloor$ or $\lceil n / r \rceil$.
\end{definition}

For example, $T_{3, 1} = K_3$ and $T_{10, 3} = K_{3, 3, 4}$.

\begin{theorem} (Turán's theorem) The Turán graph $T_{n, r}$ maximizes the number of edges among all $n$-vertex $K_{r+1}$-free graphs. It is also the unique maximizer.
\end{theorem}
\begin{proof} Turán first showed this in \cite{turan1941}. See \cite[Theorem~1.2.4]{gtac2023} for four different proofs.
\end{proof}



\subsection{Homomorphism Densities}

\section{Entropy in Combinatorics}

\section{Motivating Graph Limits}

\section{Graphons}

\section{Entropy of Graphons}

\section{Flag Algebras}

\section{Open Questions}

\section*{Acknowledgments}

\printbibliography

\end{document}
