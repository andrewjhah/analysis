\documentclass[11pt]{article}
\usepackage{master}
\DeclareMathOperator{\diam}{diam}
\newcommand{\interior}[1]{%
  {\kern0pt#1}^{\mathrm{o}}%
}

\title{Accelerated Analysis HW6}
\author{Andrew Hah}
\begin{document}

\pagestyle{plain}
\begin{center}
{\Large MATH 20310. Accelerated Analysis Homework 6} \\ 
\vspace{.2in}  
Andrew Hah \\
Due November 13, 2024
\end{center}

\begin{exercise}{7}
    \begin{proof} 
        Suppose that \( \sum a_n \) converges and that \( a_n \ge 0 \) for all \( n \). We apply the Cauchy-Schwarz inequality to the sequence \( \frac{\sqrt{a_n}}{n} \), which says that for any two sequences of real numbers \( \{x_n\} \) and \( \{y_n\} \), \[ \left( \sum x_n y_n \right)^2 \le \left( \sum x_n^2 \right) \left( \sum y_n^2 \right). \] In our case, let \( x_n = \sqrt{a_n} \) and \( y_n = \frac{1}{n} \). Then \[ \sum \frac{\sqrt{a_n}}{n} = \sum x_n y_n. \] Applying the inequality, we get \[ \left( \sum \frac{\sqrt{a_n}}{n} \right)^2 \le \left( \sum a_n \right) \left( \sum \frac{1}{n^2} \right). \] Since \( \sum a_n \) converges by assumption, we know that \( \sum a_n \) is finite. For $\sum \frac{1}{n^2}$, we see that by the ratio test, $$\alpha = \limsup \frac{\frac{1}{(n+1)^2}}{\frac{1}{n^2}} = \limsup \frac{n^2}{(n+1)^2} = \frac{1}{2n + 1} = 0 < 1$$ so $\sum \frac{1}{n^2}$ is finite. Therefore, the product \( \left( \sum a_n \right) \left( \sum \frac{1}{n^2} \right) \) is finite, which implies that \( \sum \frac{\sqrt{a_n}}{n} \) converges. 
    \end{proof}
\end{exercise}

\begin{exercise}{8}
    \begin{proof}
        Since $\sum a_n$ converges, we know $a_n \to 0$. We also know that since $\{ b_n \}$ is monotonic and bounded, $b_n$ converges, i.e. $b_n \to B$. We rewrite $a_nb_n$ as $a_nb_n = a_n(b_n - B) + a_n B$. Then the series becomes $\sum a_n(b_n - B) + \sum a_n B$. We know the second term, $\sum a_n B = B \sum a_n$ converges because $\sum a_n$ converges. Thus, it suffices to show that $\sum a_n(b_n - B)$ converges. By definition, we see that for all $\epsilon > 0$, there exists $N$ such that for all $n \ge N$, $b_n - B < \sqrt{\epsilon}$. We also see that $\forall \epsilon > 0$, there exists $N'$ such that $\forall m > n \ge N'$, $\left| \sum_{k = n}^m a_k \right| \le \sum_{k = n}^m |a_k| < \sqrt{\epsilon}$. Taking $N'' = \max(N, N')$, we see that $\forall m > n \ge N''$, \begin{equation*}
            \begin{split}
                \left| \sum_{k = n}^m a_k(b_k - B) \right| & \le \sum_{k = n}^m |a_k||b_k - B| \\
                & < \sqrt{\epsilon} \cdot \sum_{k = n}^m |a_k| \\
                & < \sqrt{\epsilon} \cdot \sqrt{\epsilon} = \epsilon
            \end{split}
        \end{equation*}
        Thus, $\sum a_n (b_n - B)$ converges by Cauchy convergence, so the entire series $\sum a_nb_n$ converges. 
    \end{proof}
\end{exercise}

\begin{exercise}{9}
    \begin{enumerate} [(a)]
        \item \begin{solution}
            \begin{equation*}
                \begin{split}
                    \limsup \sqrt[n]{|n^3|} & = \limsup |n^3|^{\frac{1}{n}} \\ 
                    & = \limsup |n|^{\frac{3}{n}} \\
                    & = 1 \\
                    & \implies R = 1
                \end{split}
            \end{equation*}
        \end{solution}
        \item \begin{solution}
            \begin{equation*}
                \begin{split}
                    \limsup \left| \frac{\frac{2^{n+1}}{(n+1)!}}{\frac{2^n}{n!}} \right| & = \limsup \frac{2^{n+1} \cdot n!}{2^n \cdot (n+1)!} \\
                    & = \limsup \frac{2}{n+1} \\
                    & = 0 \\
                    & \implies R = \infty
                \end{split}
            \end{equation*}
        \end{solution}
        \item \begin{solution}
            \begin{equation*}
                \begin{split}
                    \limsup \sqrt[n]{\left| \frac{2^n}{n^2} \right|} & = \limsup \frac{2}{|n|^{\frac{2}{n}}} \\
                    & = 2 \\
                    & \implies R = \frac{1}{2}
                \end{split}
            \end{equation*}
        \end{solution}
        \item \begin{solution}
            \begin{equation*}
                \begin{split}
                    \limsup \sqrt[n]{\left| \frac{n^3}{3^n} \right|} & = \limsup \frac{|n|^{\frac{3}{n}}}{3} \\ 
                    & = \frac{1}{3} \\
                    & \implies R = 3
                \end{split}
            \end{equation*}
        \end{solution}
    \end{enumerate}
\end{exercise}

\begin{exercise}{10}
    \begin{proof}
        Suppose for the sake of contradiction that $R > 1$. Then $\sum a_nz^n$ converges for any $z$ such that $|z| < 1 + \delta$ with $\delta > 0$. Let $z = 1$. Then since $\sum a_nz^n$ converges, $a_nz^n \to 0$, i.e. $a_n \to 0$ since $z = 1$. However, clearly $a_n$ does not converge to 0, since there are infinitely many $a_n$'s that are distinct from 0. Thus, we have a contradiction, and $R \le 1$. 
    \end{proof}
\end{exercise}

\begin{exercise}{11}
    \begin{enumerate} [(a)]
        \item \begin{proof}
            We consider two cases: 
        \begin{enumerate} [(i)]
            \item Case 1: $0 < a_n \le 1$ for all $n \ge N$. Then we have $1 + a_n \le 1 + 1 = 2$, which implies that $\frac{a_n}{1 + a_n} \ge \frac{a_n}{2}$. Since $a_n > 0$ and $\sum a_n$ diverges, $$\sum_{n = N}^\infty \frac{a_n}{1 + a_n} \ge \frac{1}{2} \sum_{n = N}^\infty a_n = \infty$$ Thus, $\sum \frac{a_n}{1 + a_n}$ diverges.
            \item Case 2: $a_n > 1$ infinitely often. Consider $A = \{ n \in \mathbb{N} \mid a_n > 1 \}$. By construction, $A$ is an infinite set. For $a_n > 1$, we have $\frac{a_n}{1 + a_n} > \frac{1}{1+1} = \frac{1}{2}$. Therefore for all $n \in A$, $\sum_{n \in A} \frac{a_n}{1 + a_n} > \sum_{n \in A} \frac{1}{2}$. Since $A$ is infinite, $\sum_{n \in A} \frac{1}{2}$ diverges, implying that $\sum_{n \in A} \frac{a_n}{1 + a_n}$ diverges. 
        \end{enumerate}
        In both cases, $\sum \frac{a_n}{1 + a_n}$ diverges. 
        \end{proof}
        \item \begin{proof}
            We want to prove that $\frac{a_{N+1}}{s_{N+1}} + \dots + \frac{a_{N + k}}{s_{N + k}} \ge 1 - \frac{s_N}{s_{N+k}}$. Rewriting the left hand side, we get that $$\frac{a_{N+1}}{s_{N+1}} + \dots + \frac{a_{N + k}}{s_{N + k}} = \sum_{j = N + 1}^{N + k} \frac{a_j}{s_j}$$ Rewriting the right hand side, we see that
            \begin{equation*}
                \begin{split}
                    1 - \frac{s_N}{s_{N+k}} & = \frac{s_{N+k} - s_N}{s_{N+k}} \\
                    & = \frac{\sum_{j = N+1}^{N+k} a_j}{s_{N+k}} \\
                    & = \sum_{j = N+1}^{N+k} \frac{a_j}{s_{N+k}}
                \end{split}
            \end{equation*}
            Rewriting the given inequality, we see that it is equivalent to $$\sum_{j = N + 1}^{N + k} \frac{a_j}{s_j} \ge \sum_{j = N+1}^{N+k} \frac{a_j}{s_{N+k}}$$
            Since $a_n > 0$, we know that the partial sums are strictly increasing, i.e. that for $N + 1 \le j \le N + k$, we have $s_j \le s_{N + k}$. This implies that for $N + 1 \le j \le N + k$, $$\frac{a_j}{s_j} \ge \frac{a_j}{s_{N+k}}$$ Using this, we can sum over these terms from $N+1$ to $N+k$ to get that $$\sum_{j = N + 1}^{N + k} \frac{a_j}{s_j} \ge \sum_{j = N+1}^{N+k} \frac{a_j}{s_{N+k}}$$ which is exactly what we wanted to show. \\
            Since $\sum a_n$ diverges, we know that as $n \to \infty$, $s_n \to \infty$. For any fixed $N$, $s_N$ is finite, but taking the limit as $k \to \infty$, $s_{N +k} \to \infty$, i.e. $\lim_{k \to \infty} s_{N +k} = \infty$. Using this, we see that $$\lim_{k \to \infty} \frac{s_N}{s_{N+k}} = 0$$From the inequality we proved, we have that $\sum_{j = N+1}^{N + k} \frac{a_j}{s_j} \ge 1 - \frac{s_N}{s_{N+k}}$. Taking $k \to \infty$, we get $$\sum_{j = N+1}^{N+k} \frac{a_j}{s_j} \ge 1 - \epsilon$$ Since the inequality holds for any $N$ and large $k$, we can construct an increasing sequence of partial sums that grows without bound. For $N= 0$ we have $\sum_{j = 1}^k \frac{a_j}{s_j} \ge 1 - \frac{s_0}{s_k} = 1 - 0 = 1$. For $N = k$, $\sum_{j = k+1}^{2k} \frac{a_j}{s_j} \ge \frac{s_k}{s_{2k}} = 1 - \epsilon$. Continuing this process, we can find intervals where the sum over each interval is at least $1 - \epsilon$. By summing over these intervals, $$\sum_{j = 1}^\infty \frac{a_j}{s_j} = \sum_{j = 1}^k \frac{a_j}{s_j} + \sum_{j = k+1}^{2k} \frac{a_j}{s_j} + \dots$$ Each group sum is at least $1 - \epsilon$ and there are infinitely many such groups. Thus the series diverges since $$\sum_{j = 1}^\infty \frac{a_j}{s_j} \ge \sum_{n = 1}^\infty (1 - \epsilon) = \infty$$
        \end{proof}
        \item \begin{proof}
            We first compute the difference \( \dfrac{1}{s_{n-1}} - \dfrac{1}{s_n} \) for \( n \geq 1 \):
\[
\frac{1}{s_{n-1}} - \frac{1}{s_n} = \frac{s_n - s_{n-1}}{s_{n-1} s_n} = \frac{a_n}{s_{n-1} s_n}.
\] Since \( s_{n-1} < s_n \), it follows that:
\[
s_{n-1} s_n < s_n s_n = s_n^2.
\]
Therefore, 
\[
\frac{a_n}{s_{n-1} s_n} > \frac{a_n}{s_n^2}.
\]
which implies 
\[
\frac{1}{s_{n-1}} - \frac{1}{s_n} = \frac{a_n}{s_{n-1} s_n} > \frac{a_n}{s_n^2}.
\]
Thus:
\[
\frac{a_n}{s_n^2} \leq \frac{1}{s_{n-1}} - \frac{1}{s_n}.
\]
Then, we consider the partial sums up to \( N \):
\[
\sum_{n=1}^N \frac{a_n}{s_n^2} \leq \sum_{n=1}^N \left( \frac{1}{s_{n-1}} - \frac{1}{s_n} \right).
\]
We see that the right-hand side is a telescoping sum:
\[
\sum_{n=1}^N \left( \frac{1}{s_{n-1}} - \frac{1}{s_n} \right) = \frac{1}{s_0} - \frac{1}{s_N}.
\]
Since \( s_0 = 0 \), \( \frac{1}{s_0} \) is undefined, so we start the sum from \( n = 2 \). For \( n \geq 2 \):
\[
\sum_{n=2}^N \frac{a_n}{s_n^2} \leq \frac{1}{s_1} - \frac{1}{s_N}.
\]
As \( N \to \infty \), \( s_N \to \infty \), so \( \frac{1}{s_N} \to 0 \).

Therefore:
\[
\sum_{n=2}^\infty \frac{a_n}{s_n^2} \leq \frac{1}{s_1}.
\]

Adding \( \dfrac{a_1}{s_1^2} \) to both sides:
\[
\sum_{n=1}^\infty \frac{a_n}{s_n^2} = \frac{a_1}{s_1^2} + \sum_{n=2}^\infty \frac{a_n}{s_n^2} \leq \frac{a_1}{s_1^2} + \frac{1}{s_1}.
\]
Since \( \dfrac{a_1}{s_1^2} \) and \( \dfrac{1}{s_1} \) are finite constants, the entire sum is bounded above by a finite number. Because the partial sums of \( \sum\limits_{n=1}^\infty \dfrac{a_n}{s_n^2} \) are bounded above and increase monotonically (since all terms are positive), the series converges.
        \end{proof}
        \item \begin{solution}
            We first look at $\sum \frac{a_n}{1 + na_n}$. We can estimate this term based on how large $a_n$ is. If $a_n$ is small relative to $\frac{1}{n}$, then $na_n$ is small, and the term $\frac{a_n}{1 + na_n} \approx a_n$, and behaves similarly to $a_n$. Since $\sum a_n$ diverges, the series diverges. On the other hand, if $a_n$ is large relative to $\frac{1}{n}$, then $na_n \gg 1$, so $\frac{a_n}{1+na_n} \approx \frac{a_n}{na_n} = \frac{1}{n}$. This series $\sum \frac{1}{n}$ also diverges, so since both cases suggest divergence, the series diverges. \\

            Next, for $\sum \frac{a_n}{1 + n^2a_n}$, we can use the comparison test with $\sum \frac{1}{n^2}$. Since $a_n > 0$, we know that $1 + n^2a_n > 1$. This implies that $$\frac{a_n}{1 + n^2a_n} < \frac{a_n}{n^2a_n} = \frac{1}{n^2}$$ Therefore, $\frac{a_n}{1+n^2a_n} \le \frac{1}{n^2}$ for all $n$. Since $\sum \frac{1}{n^2}$ converges, by the comparison test $\sum \frac{a_n}{1+n^2a_n}$ also converges. 
        \end{solution}
    \end{enumerate}
\end{exercise}

\begin{exercise}{12}
    \begin{enumerate} [(a)]
        \item \begin{proof}
            We see that $r_m = a_m + a_{m+1} + \dots + a_n + r_{n+1}$. Dividing each term by $r_m$, we have that \begin{equation*}
                \begin{split}
                    1 & = \frac{a_m}{r_m} + \frac{a_{m+1}}{r_m} + \dots + \frac{a_n}{r_m} + \frac{r_{n+1}}{r_m} \\
                    1 - \frac{r_{n+1}}{r_m} & = \frac{a_m}{r_m} + \frac{a_{m+1}}{r_m} + \dots + \frac{a_n}{r_m}
                \end{split}
            \end{equation*}
            Since $r_{n+1} < r_n$ (because $r_{n+1} = r_n - a_n$, we have that $$1 - \frac{r_{n+1}}{r_m} > 1 - \frac{r_n}{r_m}$$ Putting this together, we have that $$1 - \frac{r_{n+1}}{r_m} = \frac{a_m}{r_m} + \frac{a_{m+1}}{r_m} + \dots + \frac{a_n}{r_m} > 1 - \frac{r_n}{r_m}$$ Since $m < n$, $r_n < r_m$, so changing the denominator values to the given inequality only makes the left hand side larger, namely $$\frac{a_m}{r_m} + \dots + \frac{a_n}{r_n} \ge  \frac{a_m}{r_m} + \frac{a_{m+1}}{r_m} + \dots + \frac{a_n}{r_m} > 1 - \frac{r_n}{r_m}$$

            Because $\sum a_n$ converges, as $n \to \infty$, $r_n \to 0$, which implies that for large $m$ and $n$, we have approximately that $\frac{a_m}{r_m} + \dots + \frac{a_n}{r_n} > 1$. This inequality suggests that we can repeatedly add terms $\frac{a_n}{r_n}$ to the partial sums of $\sum \frac{a_n}{r_n}$ in such a way taht each addition results in a positive contribution. Thus, the partial sums of $\sum \frac{a_n}{r_n}$ grow without bound, since we can always find segments of the series whose sum is greater than 1. This implies that $\sum \frac{a_n}{r_n}$ diverges, as its partial sums are unbounded. 
        \end{proof}
        \item \begin{proof}
            We first rewrite $\sqrt{r_n} - \sqrt{r_{n+1}}$ to get $$\sqrt{r_n} - \sqrt{r_{n+1}} = \frac{r_n - r_{n+1}}{\sqrt{r_n} - \sqrt{r_{n+1}}} = \frac{a_n}{\sqrt{r_n} - \sqrt{r_{n+1}}}$$ Now, we see that \begin{equation*}
                \begin{split}
                    \frac{a_n}{\sqrt{r_n}} & = \frac{a_n}{\sqrt{r_n} - \sqrt{r_{n+1}}} \cdot \frac{\sqrt{r_n} - \sqrt{r_{n+1}}}{\sqrt{r_n}} \\
                    & = (\sqrt{r_n} - \sqrt{r_{n+1}}) \cdot \frac{\sqrt{r_n} - \sqrt{r_{n+1}}}{\sqrt{r_n}}
                \end{split}
            \end{equation*}
            $\frac{\sqrt{r_n} - \sqrt{r_{n+1}}}{\sqrt{r_n}} < 2$ because $\sqrt{r_{n+1}} < \sqrt{r_n}$. Therefore, we the inequality, $$\frac{a_n}{\sqrt{r_n}} = (\sqrt{r_n} - \sqrt{r_{n+1}}) \cdot \frac{\sqrt{r_n} - \sqrt{r_{n+1}}}{\sqrt{r_n}} < 2 (\sqrt{r_n} - \sqrt{r_{n+1}})$$

            We sum both sides to see that $$\sum \frac{a_n}{\sqrt{r_n}} < 2 \sum (\sqrt{r_n} - \sqrt{r_{n+1}})$$ The series $\sum (\sqrt{r_n} - \sqrt{r_{n+1}})$ is telescoping, and we are left with $\sum (\sqrt{r_n} - \sqrt{r_{n+1}}) = \sqrt{r_1} - \lim_{n \to \infty} \sqrt{r_n}$. Since $r_n \to 0$, $\sum (\sqrt{r_n} - \sqrt{r_{n+1}}) = \sqrt{r_1}$. Therefore, since $\sum \frac{a_n}{\sqrt{r_n}}$ is bounded above by a convergent series, we conclude the it converges. 
        \end{proof}
    \end{enumerate}
\end{exercise}

\end{document}