\documentclass[11pt]{article}
\usepackage{master}
\DeclareMathOperator{\diam}{diam}
\newcommand{\interior}[1]{%
  {\kern0pt#1}^{\mathrm{o}}%
}

\let\eps\varepsilon

\title{Accelerated Analysis 3 HW3}
\author{Andrew Hah}
\begin{document}

\pagestyle{plain}
\begin{center}
{\Large MATH 20510. Accelerated Analysis III} \\
{\Large Homework 3} \\
\vspace{.2in}  
Andrew Hah \\
Due April 16, 2025
\end{center}

\begin{exercise}{11.4}
    \begin{proof}
    We know that $f$ and $g$ measurable on $E$ implies that $fg$ is measurable on $E$. Since $g$ is bounded, there exists a constant $M=\sup_{x\in E}|g(x)|<\infty$ with $|f(x)g(x)|\le M|f(x)|$ for all $x\in E$.  Integrating, we obtain
    \[
      \int_E |fg|\,d m \le M\int_E |f|\,d m < \infty,
    \]
    so $fg \in \mathscr{L}(m)$.
    \end{proof}
\end{exercise}

\begin{exercise}{1}
    \begin{enumerate} [(a)]
        \item \begin{proof}
            Fix $\eta>0$ and define the sets
            \[
              A_k=\bigcup_{n\ge k}\{x\in E:|f_n(x)-f(x)|\ge\eta\}, \qquad k\in\mathbb N.
            \]
            Since $f_n \to f$ pointwise, we have that $\bigcap_{k\ge1}A_k=\emptyset$.  Since $m(E)<\infty$, $m(A_k)\to0$.  Given $\delta>0$ choose $N$ with $m(A_N)<\delta$. Letting $A=E\setminus A_N$ gives us the conclusion. 
            \end{proof}
            \item \begin{proof}
Let $\varepsilon>0$ be given.
For each $k\ge1$ apply part a) with $\eta_k =\frac{1}{k}$ and $\delta_k =\frac{\varepsilon}{2^{k+1}}$ to obtain a set $A_k\subseteq E$ and an index $N_k$ such that
\begin{equation*}
 |f_n(x)-f(x)|<\frac1k \;\;(x\in A_k,\;n\ge N_k), \qquad m(E\setminus A_k)<\delta_k.
\end{equation*}
Define
\(
  F_0 =\bigcap_{k=1}^{\infty}A_k.
\)
Because $\sum_{k\ge1}\delta_k=\varepsilon/2$, 
\[
 m(E\setminus F_0)=m\Bigl(\bigcup_{k\ge1}(E\setminus A_k)\Bigr)\le\sum_{k\ge1}\delta_k<\frac{\varepsilon}{2}. \tag{1}
\]

Now fix $\eta>0$ and choose $k$ with $1/k<\eta$.  For every $x\in F_0\subseteq A_k$ and $n\ge N_k$ we have $|f_n(x)-f(x)|<1/k<\eta$.  Hence $f_n\to f$ uniformly on $F_0$. Then we know there exists a closed $F\subseteq F_0$ with
\[
 m(F_0\setminus F)<\frac{\varepsilon}{2}.
\]
Combining with $(1)$ gives $m(E\setminus F)<\varepsilon$.  Uniform convergence passes from $F_0$ to the subset $F$, so we are done.
\end{proof}
    \end{enumerate}
\end{exercise}

\begin{exercise}{2}
    \begin{enumerate} [(a)]
        \item \begin{proof} Write \( f = \sum_{k=1}^m a_k \chi_{A_k} \), where \( \{A_k\}_{k=1}^m \) are disjoint measurable sets with \( \bigcup_{k=1}^m A_k = E \). Given \( \eps > 0 \), for each $k$ there exists a closed set \( F_k \subseteq A_k \) such that
\[
m(A_k \setminus F_k) < \frac{\eps}{m}.
\]
Let \( F = \bigcup_{k=1}^m F_k \). Since finite unions of closed sets are closed, \( F \) is closed. Then
\[
m(E \setminus F) = \sum_{k=1}^m m(A_k \setminus F_k) < \sum_{k=1}^m \frac{\eps}{m} = \eps.
\]
On each \( F_k \), \( f = a_k \), which is continuous. Since the \( F_k \) are pairwise disjoint and closed, \( f \) is continuous on \( F \). By Lemma 1, there exists a continuous extension \( g: \mathbb{R} \to \mathbb{R} \) such that \( g(x) = f(x) \) for all \( x \in F \). \end{proof}
\item  \begin{proof} Let \( f: E \to \mathbb{R} \) be an arbitrary measurable function. We know there exists a sequence \( \{f_n\} \) of simple functions converging pointwise to \( f \). Applying Exercise 1, for \( \eps/2 > 0 \), there exists a closed set \( F_0 \subseteq E \) such that \( f_n \to f \) uniformly on \( F_0 \) with:
    \[
    m(E \setminus F_0) < \frac{\eps}{2}.
    \]
For each \( f_n \), by part (a), there exists a closed set \( F_n \subseteq E \) and a continuous function \( g_n: \mathbb{R} \to \mathbb{R} \) such that
    \[
    f_n = g_n \text{ on } F_n \quad \text{and} \quad m(E \setminus F_n) < \frac{\eps}{2^{n+1}}.
    \]
Define
    \[
    F = F_0 \cap \left( \bigcap_{n=1}^\infty F_n \right).
    \]
    This intersection is closed, and
    \[
    m(E \setminus F) \leq m(E \setminus F_0) + \sum_{n=1}^\infty m(E \setminus F_n) < \frac{\eps}{2} + \sum_{n=1}^\infty \frac{\eps}{2^{n+1}} = \eps.
    \]
    On \( F \), we have \( f_n = g_n \) for all \( n \), and \( f_n \to f \) uniformly. Since each \( g_n \) is continuous and uniform convergence preserves continuity, \( f|_F \) is continuous. By Lemma 1, extend \( f|_F \) to a continuous function \( g: \mathbb{R} \to \mathbb{R} \).
    Thus, \( f = g \) on the closed set \( F \) with \( m(E \setminus F) < \eps \), completing the proof .
\end{proof}
    \end{enumerate}
\end{exercise}

\begin{exercise}{3}
    \begin{proof} Because \(\{f_n\}\) is monotone increasing
and non‑negative, the pointwise limit  
\(
  f(x)=\sup_{n\ge1}f_n(x)=\lim_{n\to\infty}f_n(x)
\)
is measurable and still non‑negative. For every \(n\) we have \(0\le f_n\le f\).
By monotonicity of the Lebesgue integral,
\(
  \int_E f_n\,d m \;\le\;\int_E f\,d m 
\)
so
\[
  \displaystyle \limsup_{n\to\infty}\int_E f_n d m
  \le \int_E f d m.
\]

Fatou’s Lemma applied directly to the sequence \(\{f_n\}\) (which is non‑negative) gives
\[
  \int_E \liminf_{n\to\infty}f_n\,d m
  \;\le\;
  \liminf_{n\to\infty}\int_E f_n\,d m
\]
But \(\liminf_{n\to\infty} f_n = f\) almost everywhere, hence
\[
  \int_E f\,d m
  \;\le\;
  \liminf_{n\to\infty}\int_E f_n\,dm 
\]
Putting this together we obtain
\[
  \int_E f\,d m
  \;\le\;
  \liminf_{n\to\infty}\int_E f_n\,d m
  \;\le\;
  \limsup_{n\to\infty}\int_E f_n\,d m
  \;\le\;
  \int_X f\, dm
\]
All three quantities are therefore equal, and the limit
\(\displaystyle\lim_{n\to\infty}\int_E f_n\,d m\) exists and equals \(\int_E f\,d m\).
\end{proof}
\end{exercise}

\begin{exercise}{4}
    \begin{proof} For each \(n\ge1\) let  
\[
  N_{0} =\{x\in E : f_{1}(x)<0\},\qquad
  N_{n} =\{x\in E : f_{n}(x)>f_{n+1}(x)\}.
\]
By hypotheses, every \(N_{k}\) satisfies \(m(N_{k})=0\).
Let \(N =\bigcup_{k=0}^{\infty}N_{k}\).  
Because a countable union of measure zero sets has measure zero, \(m(N)=0\). On \(A = E \setminus N\) the sequence is pointwise increasing and non–negative:
\[
  0\le f_{1}(x)\le f_{2}(x)\le\cdots
  \qquad(\forall x\in A).
\]
For every \(n\) define the functions  
\[
  g_{n} =f_{n}\,\chi_{A},\qquad 
  g =\lim_{n\to\infty}g_{n}=\bigl(\sup_{n\ge1}g_{n}\bigr).
\]
All \(g_{n}\) are measurable, \(0\le g_{1}\le g_{2}\le\cdots\) on \emph{all}
of \(E\), and \(g_{n}=f_{n}\) on $A$.  
Because \(\{g_{n}\}\) is monotone everywhere, by the usual MCT,  
\[
  \lim_{n\to\infty}\int_{E}g_{n}\,dm
  =\int_{E}g\,dm.
\tag{2}
\]
Since \(g_{n}=f_{n}\) and \(g=f\) on \(A\) while
\(m(E\setminus A)=m(N)=0\), we have
\[
  \int_{E}g_{n}\,dm
  =\int_{A}f_{n}\,dm\quad\text{and}\quad
  \int_{E}g\,dm
  =\int_{A}f\,dm
  =\int_{E}f\,dm.
\]
Similarly, \(\int_{E}f_{n}\,dm=\int_{A}f_{n}\,dm\), so
\(
  \int_{E}g_{n}\,dm=\int_{E}f_{n}\,dm
\)
for every \(n\).
On \(A\) we have \(g_{n}(x)=f_{n}(x)\) and the pointwise limit of
\(\{f_{n}\}\) exists. On the measure zero set \(N\) we defined \(f= 0 \) while
\(g=0\) there as well, so \(g=f\) a.e.
Thus, we can replace the integrals of \(g_{n}\) and \(g\) in
(2) by  \(f_{n}\) and \(f\):
\[
  \lim_{n\to\infty}\int_{E}f_{n}\,dm
  =\lim_{n\to\infty}\int_{E}g_{n}\,dm
  =\int_{E}g\,dm
  =\int_{E}f\,dm.
\]
\end{proof}
\end{exercise}

\end{document}