\documentclass[11pt]{article}
\usepackage{master}
\DeclareMathOperator{\diam}{diam}
\newcommand{\interior}[1]{%
  {\kern0pt#1}^{\mathrm{o}}%
}

\title{Accelerated Analysis HW6}
\author{Andrew Hah}
\begin{document}

\pagestyle{plain}
\begin{center}
{\Large MATH 20310. Accelerated Analysis Homework 7} \\ 
\vspace{.2in}  
Andrew Hah \\
Due November 20, 2024
\end{center}

\begin{exercise}{2}
    \begin{proof}
        Let $y \in f(\overline{E})$. Then there exists some $x \in \overline{E}$ such that $y = f(x)$. If $x \in E$, then $f(x) \in f(E) \subseteq \overline{f(E)}$. If $x \notin E$ but is a limit point of $E$, then there exists $\{ x_n \} \subseteq E$ such that $x_n \to x$. Since $f$ is continuous, $f(x_n) \to f(x)$. Thus, $f(x)$ is a limit point of $f(E)$, and hence $f(x) \in \overline{f(E)}$. Thus $f(\overline{E}) \subseteq \overline{f(E)}$. \\
        Consider $X = [0, \infty)$ and $Y = [0, 1]$ and define $f(x) = \frac{x}{1 + x}$. We take $E = [0, \infty)$, so $\overline{E} = E$. Then \( f(\overline{E}) = f([0, \infty)) = \left\{ \frac{x}{1 + x} \mid x \in [0, \infty) \right\} = [0,1) \) since as $x \to \infty$, $f(x) \to 1$. On the other hand, $\overline{f(E)} = [0, 1]$, and we have $f(\overline{E}) \subsetneq \overline{f(E)}$. 
    \end{proof}
\end{exercise}

\begin{exercise}{3}
    \begin{proof}
        We see that the image of the zero set of $f$ is $\{ 0 \}$, which is closed. Then since $f$ is continuous, the preimage of a closed set is closed, i.e. $Z(f) = f^{-1}(\{ 0 \})$ is closed. 
    \end{proof}
\end{exercise}

\begin{exercise}{4}
    \begin{proof}
        Let $y \in f(X)$. Then there exists some $x \in X$ such that $y = f(x)$. Since $E$ is dense in $X$, for any neighborhood $U$ of $x$, $U$ contains a point from $E$. This implies that there exists a sequence $\{ x_n \} \subset E$ such that $x_n \to x$. Because $f$ is continuous, $f(x_n) \to f(x) = y$. Thus, every point $y \in f(X)$ is a limit of a sequence from $f(E)$, which means $y \in \overline{f(E)}$, hence $f(X) \subset \overline{f(E)}$. Since $f(E) \subset f(X)$, the closure $\overline{f(E)} \subset \overline{f(X)}$. But $\overline{f(X)} = f(X)$ because $f(X)$ is the image of $X$ under $f$, and no additional limit points exist beyond $f(X)$ in $Y$. Thus, we have $\overline{f(E)} = f(X)$.  \\
        Now assume $f(p) = g(p)$ for all $p \in E$. Define the function $h: X \to Y$ by $h(p) = f(p) - g(p)$. Since $f$ and $g$ are continuous, $h$ is continuous. We know that $h(p) = 0$ for all $p \in E$. Now suppose for a contradiction that there exists a point $p_0 \in X$ such that $h(p_0) \neq 0$. Since $h$ is continuous at $p_0$, there exists an open neighborhood $U$ of $p_0$ where $h$ remains nonzero, i.e. some $\epsilon > 0$ for which points $\delta$ apart in $X$ get mapped within $\epsilon$ close to $p_0$ in $Y$. However, since $E$ is dense in $X$, the open set $U$ must contain points from $E$. For any $q \in U \cap E$, we have $h(q) = 0$, which contradicts that $h(p) \neq 0 $ on $U$. Thus, $h(p) = 0$ for all $p \in X$, which implies that $f(p) = g(p)$ for all $p \in X$. 
    \end{proof}
\end{exercise}

\begin{exercise}{5}
    \begin{proof}
        Let $f: E \to \mathbb{R}$ be a continuous function defined on $E \subset \mathbb{R}^1$. Since $E$ is a closed subset of $\mathbb{R}$, its complement $D = \mathbb{R} \setminus E$ is open. In particular, because $D$ is open, it can be expressed as a countable union of disjoint open intervals, $D = \bigcup_{n} I_n$, where each $I_n = (a_n, b_n)$ is an open interval. Set $g(x) = f(x)$ for all $x \in E$. For each interval $I_n$, we define $g$ such that it connects the values of $f$ at the endpoints of $I_n$ (when finite) by a straight line. \\
        Case 1) Both endpoints of $I_n$ are finite. If $a_n, b_n$ are finite, define $g$ on $I_n$ by $g(x) = f(a_n) + \frac{(x- a_n)}{(b_n - a_n)} [f(b_n) - f(a_n)]$ for $x \in I_n$. This linearly interpolates between $f(a_n)$ and $f(b_n)$. \\
        Case 2) One endpoint is infinite. If $b_n = \infty$, define $g(x) = f(a_n)$ for all $x \in I_n$. If $a_n = - \infty$, define $g(x) = f(b_n)$ for all $x \in I_n$. \\
        Clearly on $E$, $g$ is continuous because $f$ is continuous on $E$. On each $I_n$, $g$ is continuous because it's either a linear function or a constant, both of which are continuous. At endpoints of $I_n$, for finite endpoints, since \( f \) is continuous on \( E \) and \( E \ni a_i, b_i \), \( f \) is continuous at \( a_i \) and \( b_i \). Thus, $g$ is continuous on $\mathbb{R}$ and extends $f$ from $E$ to $\mathbb{R}$. \\

        This result fails if $E$ is not closed. Consider $E = (0, 1)$, an open interval, and define $f: E \to \mathbb{R}$ by $f(x) = \frac{1}{x}$. As $x \to 0^+$, $f(x) \to \infty$ so $f$ cannot be continuously extended to $x = 0$. Similarly, $f(x)$ cannot be extended to $x = 1$. \\

        We now extend this result to vector-valued functions. Let $f: E \to \mathbb{R}^k$ be continuous. We use the exact same approach. Because vector addition and scalar multiplication are continuous operations in $\mathbb{R}^k$, $g$ is continuous. 
    \end{proof}
\end{exercise}

\begin{exercise}{6}
    \begin{proof}
        $(\implies)$ Since $E$ is compact, it is closed and bounded. Consider the function $\varphi: E \to \mathbb{R}^2$ defined by $\varphi(x) = (x, f(x))$. The function $\varphi$ is continuous because $f$ is continuous. Specifically, for each $x_0 \in E$, $\lim_{x \to x_0} \varphi(x) = \left( \lim_{x \to x_0} x, \lim_{x \to x_0} f(x) \right) = (x_0, f(x_0)) = \varphi(x_0)$. Since $\varphi$ is continuous and $E$ is compact, the image of $E$, $\varphi(E)$, is compact in $\mathbb{R}^2$. Therefore the graph of $f$, $\varphi(E)$, is compact. \\

        $(\impliedby)$ Suppose for a contradiction that $f$ is not continuous at some point $x_0 \in E$. Then there exists $\epsilon_0 > 0$ such that for every $\delta > 0$ there exists $x \in E$ satisfying $0 < |x - x_0| < \delta$ and $|f(x) - f(x_0)| \ge \epsilon_0$. Then, we can find a sequence $\{ x_n \}$ in $E$ such that $x_n \to x$ and $|f(x) - f(x_0)| \ge \epsilon_0$ for all $n$. Look at the sequence $\{ (x_n, f(x_n)) \}$ in the graph of $f$. Since the graph of $f$ is compact, this sequence has a convergent subsequence, $\{ (x_{n_k}, f(x_{n_k})) \}$ converging to some point $(x^*, y^*) \in \mathbb{R}^2$. Since $x_{n_k} \to x_0$, it follows that $x^* = x_0$. However, $f(x_{n_k})$ does not converge to $f(x_0)$ because $|f(x_{n_k}) - f(x_0)| \ge \epsilon_0$ for all $k$. Therefore, the limit $y^*$ satisfies, $|y^* - f(x_0)| \ge \epsilon_0$. Since $(x_{n_k}, f(x_{n_k})) \to (x_0, y^*)$ and the graph of $f$ is closed, the limit point $(x_0, y^*)$ must belong to the graph of $f$, i.e. $y^* = f(x_0)$. However, $|y^* - f(x_0)| \ge \epsilon_0$, which is a contradiction. 
    \end{proof}
\end{exercise}

\begin{exercise}{7}
    \begin{proof}
        For $(x, y) \neq (0, 0)$, consider $|f(x, y)| = \left| \frac{xy^2}{x^2 + y^4} \right|$. We see \begin{equation*}
            \begin{split}
                (x - y^2)^2 & \ge 0 \\
                x^2 - 2xy^2 + y^4 & \ge 0 \\
                x^2 + y^4 & \ge 2x^2 \\
                x^2 + y^4 & \ge 2|xy^2| \\
                1 & \ge \frac{2|xy^2|}{x^2 + y^4} \\
                \frac{1}{2} & \ge \left| \frac{xy^2}{x^2 + y^4} \right| = |f(x, y)|
            \end{split}
        \end{equation*}
        At $(0, 0)$, $f(0,0) = 0$. Thus, $\forall (x, y) \in \mathbb{R}^2$, $|f(x, y)| \le \frac{1}{2}$, so $f$ is bounded on $\mathbb{R}^2$. \\

        Consider the path $y = \sqrt[3]{x}$ for $x > 0$ approaching 0. Along this path, $x = x$, $y = \sqrt[3]{x}$, and as $x \to 0^+, y \to 0^+$. Computing $g$ along this path, we see $$g(x, y) = \frac{xy^2}{x^2 + y^6} = \frac{x(x^{\frac{1}{3}})^2}{x^2 + (x^{\frac{1}{3}})^6} = \frac{x^{\frac{5}{3}}}{2x^2} = \frac{x^{-\frac{1}{3}}}{2}$$ As $x \to 0^+$, $\frac{x^{- \frac{1}{3}}}{2} \to \infty$. Therefore $|g(x, y)| \to \infty$ along this path, which implies that in every neighborhood of $(0, 0)$, $g$ becomes unbounded. \\

        Finally, we show that $f$ is not continuous at $(0, 0)$. We consider two different paths. For path 1, let $y = x$, $x \neq 0$. Then $$f(x, y) = \frac{xy^2}{x^2 + y^4} = \frac{x^3}{x^2 + x^4} = \frac{x}{1 + x^2}$$ As $x \to 0$, $f(x, x) = \frac{x}{1 + x^2} \to 0$. For path 2, let $y = \sqrt{x}$, $x > 0$. Then $$f(x, y) = \frac{xy^2}{x^2 + y^4} = \frac{x^2}{2x^2} = \frac{1}{2}$$ As $x \to 0^+$, $f(x, \sqrt{x}) \to \frac{1}{2}$. Thus, the limit of $f(x, y)$ as $(x, y) \to (0, 0)$ depends on the path taken, so the overall limit does not exist. Thus, $f$ is not continuous at $(0, 0)$. 
    \end{proof}
\end{exercise}

\begin{exercise}{8}
    \begin{proof}
        Since $f$ is uniformly continuous on $E$, for $\epsilon = 1$, there exists $\delta > 0$ such that for all $x, y \in E$, $|x - y| < \delta \implies |f(x) - f(y)| < 1$. Because $E$ is bounded, it lies within some interval $[a, b]$ in $\mathbb{R}$. The length of $[a, b]$ is finite, say $L = b - a$. Divide $[a, b]$ into $N$ subintervals of length less than $\delta / 2$: $N = \lceil \frac{L}{\delta / 2} \rceil$. This means we have finitely many intervals $I_1, \dots, I_n$ covering $[a, b]$ and hence covering $E$, where each interval $I_k$ has length less than $\delta / 2$. For each interval $I_k$ that contains points of $E$, choose a point $x_k \in E \cap I_k$. This set of points is finite, $\{ x_1, \dots, x_N \} \subset E$. Then for any $x \in E$, there exists a $x_k$ such that $x$ and $x_k$ are in the same interval $I_k$, so $| x- x_k| < \delta / 2 < \delta$. By uniform continuity, $|f(x) - f(x_k)| < 1$. We have \begin{equation*}
            \begin{split}
                |f(x)| & = |f(x_k) + [f(x) - f(x_k)]| \\
                & \le |f(x_k)| + |f(x) - f(x_k)| \\
                & < |f(x_k)| + 1
            \end{split}
        \end{equation*}
        Let $M = \max_{1 \le k \le N} |f(x_k)| + 1$. Since the set $\{ x_1, \dots , x_N \}$ is finite, $M$ is finite. Thus, for all $x \in E$, $|f(x)| \le M$, so $f$ is bounded on $E$. \\

        Consider $E = \mathbb{R}$ and define $f: E \to \mathbb{R}$ by $f(x) = x$. $f$ is uniformly continuous on $\mathbb{R}$. To see this, given $\epsilon > 0$, let $\delta  = \epsilon$. Then for all $x, y \in \mathbb{R}$, if $|x - y| < \delta$ then $|f(x) - f(y)| = |x - y| < \delta = \epsilon$. However, $f$ is not bounded on $E$, since as $x \to \infty$, $f(x) = x \to \infty$. 
    \end{proof}
\end{exercise}

\begin{exercise}{9}
    \begin{proof}
        $(\implies)$ Given $\epsilon > 0$, there exists $\delta > 0$ such that for all $x, y \in X$, $d(x, y) < \delta \implies d(f(x), f(y)) < \epsilon$. Let $E \subset X$ with $\mathrm{diam}(E) < \delta$. Then by definition, since $\mathrm{diam}(E) = \sup \{ d(x, y) \mid x, y \in E \} < \delta$, $\forall x, y \in E$, $d(x, y) < \delta$. Since $d(x, y) < \delta$ for all $x, y \in E$, it follows that $d(f(x), f(y)) < \epsilon$ for all $x, y \in E$. Therefore the diameter of $f(E)$ satisfies $\mathrm{diam} (f(E)) = \sup \{ d(f(x), f(y)) \mid x, y \in E \} < \epsilon$. \\

        $(\impliedby)$ Given $\epsilon > 0$, there exists $\delta > 0$ such that for every subset $E \subset X $ with $\mathrm{diam}(E) < \delta$, $\mathrm{diam}(f(E)) < \epsilon$. Consider arbitrary $x, y \in X$ with $d(x, y) < \delta$. Let $E = \{ x, y \} \subset X$. Then $\mathrm{diam}(E) = d(x, y) < \delta$. Since $\mathrm{diam}(E) < \delta$, it follows that $\mathrm{diam}(f(E)) < \epsilon$. But $E = \{ x, y \}$ so $\mathrm{diam}(f(E)) = d(f(x), f(y)) < \epsilon$. Therefore, for all $x, y \in X$ such that $d(x, y) < \delta$, $d(f(x), f(y)) < \epsilon$. 
    \end{proof}
\end{exercise}

\begin{exercise}{B}
    \begin{proof}
        $(\implies)$ Since $f$ is uniformly continuous on $(0, 1]$, for every $\epsilon > 0$ there exists $\delta > 0$ such that for all $x, y \in (0, 1]$, $|x - y| < \delta \implies |f(x) - f(y)| < \epsilon$. Consider any sequence $\{ x_n \} \subset (0, 1]$ with $x_n \to 0^+$ as $n \to \infty$. Then for $n, m$ large enough, $x_n, x_m < \delta$. Then $| x_n - x_m | < \delta$ so $|f(x_n) - f(x_m)| < \epsilon$. This implies that $\{ f(x_n) \}$ is a Cauchy sequence. Since $\mathbb{R}$ is complete, $\{ f(x_n) \}$ converges to some limit $L \in \mathbb{R}$. The limit $L$ is independent of the sequence $\{ x_n \}$ because the same argument applies to any sequence converging to $0^+$. Define $f(0) = L$. Now, given $\epsilon > 0$, we know by uniform continuity that there exists $\delta > 0$ such that $|x - y| < \delta \implies |f(x) - f(y)| < \epsilon $ for all $x, y \in (0, 1]$. For $x \in (0, \delta)$, $|x - 0| < \delta \implies |f(x) - f(0)| = |f(x) - L| < \epsilon$. Therefore, $f$ is continuous at $x = 0$. Thus, we have extended $f$ to $[0, 1]$ and the extended function is continuous on $[0, 1]$. \\

        $(\impliedby)$ Assume there exists a continuous function $F: [0, 1] \to \mathbb{R}$ such that $F(x) = f(x)$ for all $x \in (0, 1]$. Since $F$ is continuous on the compact set $[0, 1]$, $F$ is uniformly continuous on $[0, 1]$. Then for any $\epsilon > 0$, there exists $\delta > 0$ such that for all $x, y \in [0, 1]$, $| x- y | < \delta \implies |F(x) - F(y)| < \epsilon$. Since $f(x) = F(x)$ for $x \in (0, 1]$, the above inequality holds for all $x, y \in (0, 1]$ with $|x - y| < \delta \implies |f(x) - f(y)| < \epsilon$. Therefore, $f$ is uniformly continuous on $(0, 1]$. 
    \end{proof}
\end{exercise}

\end{document}