\documentclass[11pt]{article}
\usepackage{master}
\DeclareMathOperator{\diam}{diam}
\newcommand{\interior}[1]{%
  {\kern0pt#1}^{\mathrm{o}}%
}

\title{Accelerated Analysis HW1}
\author{Andrew Hah}
\begin{document}

\pagestyle{plain}
\begin{center}
{\Large MATH 20410. Accelerated Analysis II Homework 1} \\ 
\vspace{.2in}  
Andrew Hah \\
Due January 15, 2025
\end{center}

\begin{exercise}{5.2}
    \begin{proof}
        Let $x, y \in (a, b)$ s.t. $x < y$. By MVT, $\exists \lambda \in (x, y)$ s.t. $f'(\lambda) = \frac{f(y) - f(x)}{y - x} > 0$. Since $y - x > 0$, it must be that $f(y) - f(x) > 0 \implies f(y) > f(x)$. \\

        Let $g$ be the inverse of $f$. Then $g(f(x)) = x$. Differentiating both sides we get $g'(f(x)) f'(x) = 1$. Rearranging, we see that $g'(f(x)) = \frac{1}{f'(x)}$, and since $f'(x)$ exists on $(a, b)$, $g$ is differentiable on $(a, b)$. 
    \end{proof}
\end{exercise}

\begin{exercise}{5.3}
    \begin{proof}
        Taking the derivative of $f$, we get $f'(x) = 1 + \epsilon g'(x)$. To ensure that $f$ is one-to-one, we want to find $\epsilon > 0$ s.t. $f'(x) > 0$, i.e. $f$ is strictly increasing. We know that $-M \le g'(x) \le M$, so we have that $f'(x) \ge 1 - \epsilon M$. For $f'(x) > 0$ to hold, $1 - \epsilon M$ must be $>0$, and rearranging, we see that this happens when $\epsilon < \frac{1}{M}$. Thus, if $\epsilon$ is small enough, in particular less than $\frac{1}{M}$, $f$ is one-to-one. 
    \end{proof}
\end{exercise}

\begin{exercise}{5.6}
    \begin{proof}
        Taking the derivative of $g$, we get $g'(x) = \frac{xf'(x) - f(x)}{x^2}$. We want to show that $g'(x) > 0$, $(x > 0)$. Thus it suffices to show that $xf'(x) - f(x) > 0$, or that $f'(x) > \frac{f(x)}{x}$. We know that $f$ is differentiable on $(0, x)$, so by MVT, $\exists \lambda \in (0, x)$ s.t. $f'(\lambda) = \frac{f(x) - f(0)}{x - 0} = \frac{f(x)}{x}$. Since $\lambda \in (0, x)$, $\lambda < x$, which means that $f'(\lambda) < f'(x)$ because $f'$ is monotonically increasing. Thus, $\frac{f(x)}{x} < f'(x)$, so $g$ is monotonically increasing. 
    \end{proof}
\end{exercise}

\begin{exercise}{5.7}
    \begin{proof}
        For $t \neq x$, we have \begin{equation*}
            \begin{split}
                \frac{f(t)}{g(t)} & = \frac{f(t) - f(x)}{g(t) - g(x)} \\
                & = \frac{\frac{f(t) - f(x)}{t - x}}{\frac{g(t) - g(x)}{t - x}} \\
            \end{split}
        \end{equation*}
        Taking the limit as $t \to x$, we get by definition that $\lim_{t \to x} \frac{f(t)}{g(t)} = \lim_{t \to x} \frac{\frac{f(t) - f(x)}{t - x}}{\frac{g(t) - g(x)}{t - x}} = \frac{f'(x)}{g'(x)}$. 
    \end{proof}
\end{exercise}

\begin{exercise}{5.8}
    \begin{proof}
        By MVT, for $x, t \in [a, b]$ with $x \neq t$, $\exists \lambda \in (x, t)$ s.t. $f'(\lambda) = \frac{f(t) - f(x)}{t - x}$. Since $f'$ is continuous on $[a, b]$ and $[a, b]$ is compact, $f'$ is uniformly continuous on the interval. In other words, given $\epsilon > 0$, there exists $\delta > 0$ s.t. $|u - v| < \delta \implies | f'(u) - f'(v)| < \epsilon$ for all $u, v \in [a , b]$. Now if $0 < |t - x | < \delta$, then $|\lambda - x| \le |t - x| < \delta$ so $|f'(\lambda) - f'(x)| < \epsilon$. Substituting, we get that $\left| \frac{f(t) - f(x)}{t - x} - f'(x) \right| < \epsilon$. \\

        Yes this holds for vector valued functions as well since we can make a coordinate-wise argument. 
    \end{proof}
\end{exercise}

\begin{exercise}{5.9}
    \begin{proof}
        Suppose $t > 0$ (the case for $t < 0$ is symmetric). Then by MVT, $\exists \lambda \in (0, t)$ s.t. $f'(\lambda) = \frac{f(t) - f(0)}{t - 0}$. As $t \to 0$, $\lambda \to 0$ so we see that $f'(\lambda) \to 3$ and it follows that $\lim_{t \to 0} \frac{f(t) - f(0)}{t} = 3$. Thus $f'(0)$ exists and is equal to 3. 
    \end{proof}
\end{exercise}

\begin{exercise}{5.13}
    \begin{enumerate} [(a)]
        \item \begin{proof}
            $(\implies)$ First, if $x \neq 0$, $f$ is a composition of continuous functions, so $f$ is continuous for all $x \neq 0$. To check continuity of $f$ then, it suffices to check if $\lim_{x \to 0} f(x) = f(0) = 0$, i.e. we want $\lim_{x \to 0} |x|^a \sin(|x|^{-c}) = 0$. Since $\sin(|x|^{-c})$ is bounded between $-1$ and $+1$, we have $||x|^a \sin(|x|^{-c})| \le |x|^a \cdot |\sin(|x|^{-c})| \le |x|^a$. Thus, the limit goes to 0 if and only if $|x|^a \to 0$ as $x \to 0$. If $a > 0$, then $|x|^a \to 0$ as $x \to 0$. Consequently, $f$ is continuous at $x = 0$ in this case. Now let us consider when $a \le 0$. If $a = 0$, then $f(x) = |x|^0 \sin(|x|^{-c}) = \sin(|x|^{-c})$. As $x \to 0$, $|x|^{-c} \to + \infty$. As $x \to 0$, $f$ does not have a limit, since $\sin$ keeps oscillating as $|x|^{-c} \to \infty$. Thus $f$ is not continuous at $x = 0$. If $a < 0$, then $|x|^a$ blows up to $+ \infty$, and once again $f$ is not continuous at $x = 0$. Thus, $f$ continuous $\iff$ $a > 0$. 
        \end{proof}
        \item \begin{proof}
            We want the limit $f'(0) = \lim_{h \to 0} \frac{f(h) - f(0)}{h - 0} = \lim_{h \to 0} \frac{f(h)}{h}$. For $h \neq 0$, $\frac{f(h)}{h} = \frac{|h|^a \sin(|h|^{-c})}{h}$. Consider when $h > 0$ (the argument for $h < 0$ is symmetric). Then $|h| = h$ so $\frac{f(h)}{h} = h^{a - 1} \sin(h^{-c})$. We see that $\sin(h^{-c})$ is bounded between $-1$ and $+1$ and $h^{a - 1} \to 0$ as $h \to 0$ if and only if $a > 1$. Thus, if $a > 1$, $h^{a - 1} \to 0$, so $\frac{f(h)}{h} \to 0$ which means $f'(0) = 0$ exists. If $a \le 1$, we have two cases. If $a = 1$, the limit becomes $\sin(h^{-c})$, for which the limit is not well-defined. If $a < 1$, we have $a - 1 < 0$ so $h^{a - 1} \to + \infty$ as $h \to 0$. In either case, $f'(0)$ fails to exist. Hence, $f'(0)$ exists $\iff$ $ a > 1$. 
        \end{proof}
        \item [(d)] \begin{proof}
            We first consider $f$ for $x > 0$, noting that the case for $ x < 0$ is symmetric. For $x > 0$, $f(x) = x^a \sin(x^{-c})$. Differentiating, we get $f'(x) = ax^{a - 1} \sin(x^{-c}) - cx^{a - c - 1} \cos(x^{-c})$. We want to see if $\lim_{x \to 0^+} f'(x)$ exists and equals $0$. For term 1, we see that for this product to go to 0, we need $x^{a - 1} \to 0$ as $x \to 0$, which requires $a - 1 > 0$, i.e. $a > 1$. For term 2, we need $x^{a - c- 1} \to 0$ as $x \to 0$. That requires $a - c- 1 > 0$, or $a > 1 + c$. Putting this together we have our desired conclusion. 
        \end{proof}
    \end{enumerate}
\end{exercise}

\begin{exercise}{5.26}
    \begin{proof}
        Fix $x_0 \in [a, b]$ and define $M_0 = \sup |f(x)|$, $M_1 = \sup |f'(x)|$. By assumption, we have $M_1 \le A M_0$. For any $x \in [a, x_0]$, $|f(x)| = |f(x) - f(a)| \le M_1(x - a)$ by MVT. Consequently, $|f(x)| \le M_1 (x - a) \le A M_0 (x - a)$. Taking the supremum in $|f(x)|$ on $[a, x_0]$ gives $M_0 \le AM_0 (x_0 - a)$. If $A(x_0 - a) < 1$, then from $M_0 \le A M_0(x_0 - a)$ we must have $M_0 = 0$. This implies $\sup |f(x)| = 0 \implies f(x) = 0$ for all $x \in [a, x_0]$. Thus, whenever $x_0 \in [a, b]$ and $x_0$ satisfies $A(x_0 - a) < 1$, it follows that $f = 0$ on $[a, x_0]$. If $(b -a)$ is small enough s.t. $A(b - a) < 1$, we apply the above argument directly with $x_0 = b$. Otherwise, split the interval into subintervals each of which has length less than $1/ A$. Then by repeating the argument on each of the intervals incrementally we have that $f = 0$ on the entire interval. 
    \end{proof}
\end{exercise}

\begin{exercise}{1}
    First, note $d(a)=0$, since when $x=a$, the first row equals the second row \((f(a), g(a), h(a))\), making the determinant zero. Similarly, $d(b)=0$, because at $x=b$, the first row equals the third row. Hence $d(x)$ is continuous on $[a,b]$, differentiable on $(a,b)$, and satisfies $d(a)=d(b)=0$. By Rolle's Theorem, there exists $\lambda \in (a,b)$ with $d'(\lambda)=0$. \\

    Differentiating $d(x)$ with respect to\ $x$,  we get
  \[
    d'(x) \;=\; \det\!\begin{pmatrix}
      f'(x) & g'(x) & h'(x)\\
      f(a)  & g(a)  & h(a)\\
      f(b)  & g(b)  & h(b)
    \end{pmatrix}
  \]
    We know there is a $\lambda\in (a,b)$ such that $d'(\lambda)=0$.  Therefore,
  \[
    0 
    \;=\;
    d'(\lambda)
    \;=\;
    \det\!\begin{pmatrix}
      f'(\lambda) & g'(\lambda) & h'(\lambda)\\
      f(a)        & g(a)        & h(a)\\
      f(b)        & g(b)        & h(b)
    \end{pmatrix}
  \]
\item Expanding this determinant yields a linear dependence among 
  $f'(\lambda), g'(\lambda), h'(\lambda)$ with coefficients from $\{f(a),f(b),g(a),g(b),\dots\}$.
\end{exercise}

\begin{exercise}{2}
    We claim $f$ is differentiable at $x=0$, with $f'(0)=0$, and $f$ is not differentiable at any $x \neq 0$. First, note $f(0)=0$ (since $0\in\mathbb{Q}$). If $x\in\mathbb{Q}$, $f(x)=x^2\to 0$ as $x\to 0$; if $x\notin\mathbb{Q}$, $f(x)=0\to 0$. Hence $\lim_{x\to 0} f(x)=0=f(0)$. So $f$ is continuous at $0$. Derivative at $0$: 
  \(
    f'(0)
    = \lim_{h\to 0}\frac{f(h)-f(0)}{h}
    = \lim_{h\to 0}\frac{f(h)}{h}.
  \)
  For $h\neq 0$:
  \[
    \frac{f(h)}{h} \;=\;
    \begin{cases}
       \frac{h^2}{h} = h, & \text{if }h\in\mathbb{Q},\\[4pt]
       \frac{0}{h} = 0,   & \text{if }h\notin\mathbb{Q}.
    \end{cases}
  \]
  As $h\to 0$, both possibilities go to $0$. Thus $\lim_{h\to 0} \tfrac{f(h)}{h}=0$, so $f'(0)=0$. \\
    
  Now for $x \neq 0$, if $x_0\in\mathbb{Q}$, then $f(x_0)=x_0^2 \neq 0$. 
  But we can choose a sequence of irrationals $x_n\to x_0$ with $f(x_n)=0$. Hence $\lim_{n\to\infty}f(x_n)=0\neq x_0^2$. 
  So $f$ is not continuous at $x_0$ if $x_0\neq 0$ is rational. If $x_0\notin\mathbb{Q}$, then $f(x_0)=0$. 
  But we can choose a sequence of rationals $x_n\to x_0$ with $f(x_n)=x_n^2\to x_0^2\neq 0$. 
  So again there's no continuity at $x_0\neq 0$. Since $f$ is not continuous at any $x_0\neq 0$, it cannot be differentiable there.
\end{exercise}

\end{document}