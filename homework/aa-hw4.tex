\documentclass[12pt]{article}
\usepackage{master}

\title{Accelerated Analysis HW4}
\author{Andrew Hah}
\begin{document}

\pagestyle{plain}
\begin{center}
{\Large MATH 20310. Accelerated Analysis Homework 4} \\ 
\vspace{.2in}  
Andrew Hah \\
Due October 30, 2024
\end{center}

\begin{exercise}{A}
    \begin{proof}
        $(\implies)$ Suppose $E$ is disconnected. Then we can write $E$ as $E = A \cup B$ where $A, B \neq \emptyset$, $A \cap B = \emptyset$, and $\overline{A} \cap B = A \cap \overline{B} = \emptyset$. It suffices to show that this implies that $A, B$ are open relative to $E$. Let $x \in A$. Then $x \notin \overline{B}$ by the fact that $A \cap \overline{B} = \emptyset$. This means that $x$ is not a limit point of $B$ since $x \notin B$. It follows that there exists $\epsilon > 0$ such that $B(x, \epsilon) \cap B = \emptyset$. Then, we see that \begin{equation*}
            \begin{split}
                B(x, \epsilon) \cap E & = B(x, \epsilon) \cap (A \cup B) \\
                & = (B(x, \epsilon) \cap A) \cup (B(x, \epsilon) \cap B) \\
                & = B(x, \epsilon) \cap A \\
                & \subset A
            \end{split}
        \end{equation*} 
        A symmetric argument applies to all $y \in B$, hence $A$ and $B$ are both open relative to $E$. \\
        $(\impliedby)$ It suffices to show that $\overline{A} \cap B = A \cap \overline{B} = \emptyset$, given $E = A \cup B$ with $A, B \neq \emptyset$, $A \cap B = \emptyset$, and $A, B$ open relative to $E$. Since $A$ is open relative to $E$, $\forall x \in A, \exists \epsilon > 0$ such that $B(x, \epsilon) \cap E \subset A$. This implies that $B(x, \epsilon) \cap B = \emptyset$ for all $x \in A$, i.e., $B$ contains no limit points of $A$ in $E$. Thus $\overline{A} \cap B = \emptyset$. This is symmetric for $B$. Thus, $E$ is disconnected. 
    \end{proof}
\end{exercise}

\begin{exercise}{17}
Each element in \( E \) corresponds to a unique infinite sequence of the digits 4 and 7, such as \( 0.444\ldots \), \( 0.474747\ldots \), etc. We can map each element in \( E \) to a sequence in \(\{4, 7\}^\mathbb{N}\), the set of all infinite sequences of 4s and 7s. Since each sequence corresponds to a real number in \( [0, 1] \), and the set of all binary sequences \(\{4, 7\}^\mathbb{N}\) is uncountable, \( E \) is uncountable. Thus, we conclude that
\(
E \text{ is uncountable}.
\) We could also establish a bijection between $E$ and $\{ 0, 1 \}^\bbN$ which is uncountable. 

A subset \( E \subset [0, 1] \) is dense in \( [0, 1] \) if every open interval \( (a, b) \subset [0, 1] \) contains at least one point from \( E \). For any number in \( E \), only the digits 4 and 7 appear in its decimal expansion, so no element of \( E \) can lie in intervals containing points with any other digits. For instance, the interval \( (0.5, 0.6) \) contains no points from \( E \), as no number with only 4s and 7s in its expansion can be within this interval. Thus,
\( E \text{ is not dense in } [0, 1].
\)

To determine compactness, we need to show that \( E \) is closed and bounded. Since \( E \subset [0,1] \), it is trivially bounded by \( [0, 1] \). To show that \( E \) is closed, we must show that it contains all of its limit points. Let \( x \in \overline{E} \), where \( \overline{E} \) denotes the closure of \( E \). This means that there exists a sequence \( \{x_n\} \subset E \) such that \( x_n \to x \) as \( n \to \infty \). Each \( x_n \) has a decimal expansion consisting only of the digits 4 and 7. Since decimal expansions are unique (except for numbers that can be represented with a repeating 9, which do not exist in \( E \) due to the restriction on digits), any limit of a sequence of numbers in \( E \) must also have only the digits 4 and 7 in its decimal expansion. Thus, \( x \in E \), and we conclude that \( E \) is closed.

A set is perfect if it is closed and contains no isolated points. We have already established that \( E \) is closed. To show that \( E \) has no isolated points, consider any point \( x \in E \). We can approximate \( x \) arbitrarily closely with other points in \( E \) by changing a finite number of digits in \( x \)'s decimal expansion. Thus, every point in \( E \) is a limit point of \( E \), so \( E \) contains no isolated points. Hence,
\(
E \text{ is perfect}.
\)
\end{exercise}

\begin{exercise}{18}
    \begin{solution}Yes. We can construct such a set by modifying the Cantor set construction to ensure only irrational numbers are included. We construct the set as follows: Start with an interval with irrational endpoints: For example, the interval \( [\sqrt{2} - 1, \sqrt{2}] \), which lies within \( [0, 2] \) but has irrational endpoints. Follow the usual Cantor set construction by removing the open middle third of each remaining interval at each step. Since an irrational number divided by a whole number is still irrational, the end points (the only points that remain in the set) are all irrational. After infinitely many steps, the remaining set will be a Cantor-like set that retains only irrational endpoints at every stage, ensuring that every point in the final set is irrational. Like the standard Cantor set, this modified set is closed and contains all its limit points, thus making it perfect. 
    \end{solution}
\end{exercise}

\begin{exercise}{19}
    \begin{enumerate} [(a)]
        \item \begin{proof}
            Since $A$ and $B$ are closed, $A = \overline{A}$ and $B = \overline{B}$. Thus, $\overline{A} \cap B = A \cap B = \emptyset$ and $A \cap \overline{B} = A \cap B = \emptyset$. Thus, $A$ and $B$ are separated. 
        \end{proof}
        \item \begin{proof}
            Since $A$ is open, $\forall x \in A$, $\exists \epsilon > 0$ such that $B(x, \epsilon) \subset A$. Since $A \cap B = \emptyset$, $B(x, \epsilon) \cap B = \emptyset$, i.e., $B$ contains no limit points of $A$. Thus, $\overline{A} \cap B = \emptyset$. This is symmetric for $B$. Thus $A$ and $B$ are separated. 
        \end{proof}
        \item \begin{proof}
            $A$ is defined as the open ball centered at $p$ with radius $\delta$, $A = B(p, \delta)$. The closure of $A$ is all of $A$ as well as all of its limit points. The limit points of $A$ that are not in $A$ are exactly those $q \in X$ such that $d(p, q) = \delta$. This is because every neighborhood of such $q$ has a point in $A$. In other words, $\overline{A} = \{ q \mid q \in X, d(p, q) \le \delta \}$. This is clearly $X \setminus B$, or in other words, $\overline{A} \cup B = X$ and $\overline{A} \cap B = \emptyset$. Similarly for $B$, we see that $\overline{B} = \{ q \mid q \in X, d(p, q) \ge \delta \}$ from which it follows that $A \cap \overline{B} = \emptyset$.  
        \end{proof}
        \item \begin{proof}
            Suppose not, i.e., \( X \) is a connected metric space with at least two points and that \( X \) is countable. Choose two distinct points \( p, q \in X \) such that \( d(p, q) = \delta > 0 \) (this distance is positive since \( p \neq q \)). For each \( n \in \mathbb{N} \), consider the sets
    \[
    A_n = \{ x \in X \mid d(p, x) < \delta - \frac{1}{n} \}
    \]
    and
    \[
    B_n = \{ x \in X \mid d(p, x) > \delta - \frac{1}{n} \}.
    \]

    \( A_n \) and \( B_n \) Form Separated Sets: Each pair \( (A_n, B_n) \) is separated, following the reasoning from the previous exercise. This is because:
    \begin{itemize}
        \item \( A_n \) is defined as the set of points within a certain distance from \( p \), and \( B_n \) consists of points beyond that distance.
        \item \( \overline{A_n} \) and \( \overline{B_n} \) do not intersect, which satisfies the definition of separation.
    \end{itemize}

    Since \( X \) is assumed to be countable, we can find some \( A_n \) and \( B_n \) such that \( A_n \) and \( B_n \) cover all points of \( X \). This separation implies that \( X \) is disconnected, contradicting the assumption that \( X \) is connected. Therefore, the assumption that \( X \) is countable must be false. Thus, every connected metric space with at least two points is uncountable.
        \end{proof} 
    \end{enumerate}
\end{exercise}

\begin{exercise}{20}
We want to show that the closure \( \overline{S} \) of a connected set \( S \) is also connected.

\begin{proof}
    Suppose, for contradiction, that \( \overline{S} \) is not connected. Then there exist two disjoint, non-empty open sets \( U \) and \( V \) in \( X \) such that
\[
\overline{S} \subset U \cup V, \quad \overline{S} \cap U \neq \emptyset, \quad \overline{S} \cap V \neq \emptyset, \quad \text{and} \quad \overline{S} \cap U \cap V = \emptyset.
\]
Since \( U \) and \( V \) are disjoint open sets that separate \( \overline{S} \), and \( S \subset \overline{S} \), it follows that
\[
S = (S \cap U) \cup (S \cap V).
\]
Because \( U \) and \( V \) are open in \( X \), the sets \( S \cap U \) and \( S \cap V \) are relatively open in \( S \) (i.e., they are open in the subspace topology on \( S \)). Additionally, \( S \cap U \) and \( S \cap V \) are disjoint and non-empty (since \( \overline{S} \cap U \neq \emptyset \) and \( \overline{S} \cap V \neq \emptyset \)).

This contradicts the assumption that \( S \) is connected, as \( S \) can be separated into two disjoint, non-empty, relatively open sets. Therefore, our assumption that \( \overline{S} \) is disconnected must be false, so \( \overline{S} \) is connected. Thus, we conclude that the closure of a connected set is connected.
\end{proof}

Now, we check whether the interior of a connected set is always connected. It turns out that this is not always true, as shown by the following counterexample.

Consider the set \( S \) in \( \mathbb{R}^2 \) defined as the closed unit disk with an open line segment removed from the center to the boundary:
\[
S = \overline{D} \setminus \left\{ (x, 0) \mid 0 < x < 1 \right\}
\]
where 
\[
\overline{D} = \left\{ (x, y) \in \mathbb{R}^2 \mid x^2 + y^2 \leq 1 \right\}
\]
is the closed unit disk.
The set \( S \) remains connected despite the removal of the open line segment, as any two points in \( S \) can be connected by a path that does not intersect the removed segment. The interior of \( S \) is given by:
    \[
    \text{int}(S) = D \setminus \left\{ (x, 0) \mid 0 < x < 1 \right\}
    \]
    where \( D \) is the open unit disk. The removal of the open line segment \( \left\{ (x, 0) \mid 0 < x < 1 \right\} \) disconnects the interior into two disjoint open sets: Points above the \( x \)-axis and points below the \( x \)-axis. Hence, \( \text{int}(S) \) is disconnected.
\end{exercise}

\end{document}