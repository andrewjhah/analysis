\documentclass[12pt]{article}


%----------Packages----------
\usepackage{amsmath}
\usepackage{amssymb}
\usepackage{amsthm}
%\usepackage{amsrefs}
\usepackage{dsfont}
\usepackage{mathrsfs}
\usepackage{stmaryrd}
\usepackage[all]{xy}
\usepackage[mathcal]{eucal}
\usepackage{verbatim}  %%includes comment environment
\usepackage{fullpage}  %%smaller margins
\usepackage{hyperref}
\usepackage{setspace}
\usepackage{graphicx}
\usepackage[shortlabels]{enumitem}
\usepackage[dvipsnames]{xcolor}
\usepackage{tikz-cd}
\usepackage{multicol}
\onehalfspacing
%----------Commands----------

%%penalizes orphans
\clubpenalty=9999
\widowpenalty=9999

%% bold math capitals
\newcommand{\bA}{\mathbf{A}}
\newcommand{\bB}{\mathbf{B}}
\newcommand{\bC}{\mathbf{C}}
\newcommand{\bD}{\mathbf{D}}
\newcommand{\bE}{\mathbf{E}}
\newcommand{\bF}{\mathbf{F}}
\newcommand{\bG}{\mathbf{G}}
\newcommand{\bH}{\mathbf{H}}
\newcommand{\bI}{\mathbf{I}}
\newcommand{\bJ}{\mathbf{J}}
\newcommand{\bK}{\mathbf{K}}
\newcommand{\bL}{\mathbf{L}}
\newcommand{\bM}{\mathbf{M}}
\newcommand{\bN}{\mathbf{N}}
\newcommand{\bO}{\mathbf{O}}
\newcommand{\bP}{\mathbf{P}}
\newcommand{\bQ}{\mathbf{Q}}
\newcommand{\bR}{\mathbf{R}}
\newcommand{\bS}{\mathbf{S}}
\newcommand{\bT}{\mathbf{T}}
\newcommand{\bU}{\mathbf{U}}
\newcommand{\bV}{\mathbf{V}}
\newcommand{\bW}{\mathbf{W}}
\newcommand{\bX}{\mathbf{X}}
\newcommand{\bY}{\mathbf{Y}}
\newcommand{\bZ}{\mathbf{Z}}

%% blackboard bold math capitals
\newcommand{\bbA}{\mathbb{A}}
\newcommand{\bbB}{\mathbb{B}}
\newcommand{\bbC}{\mathbb{C}}
\newcommand{\bbD}{\mathbb{D}}
\newcommand{\bbE}{\mathbb{E}}
\newcommand{\bbF}{\mathbb{F}}
\newcommand{\bbG}{\mathbb{G}}
\newcommand{\bbH}{\mathbb{H}}
\newcommand{\bbI}{\mathbb{I}}
\newcommand{\bbJ}{\mathbb{J}}
\newcommand{\bbK}{\mathbb{K}}
\newcommand{\bbL}{\mathbb{L}}
\newcommand{\bbM}{\mathbb{M}}
\newcommand{\bbN}{\mathbb{N}}
\newcommand{\bbO}{\mathbb{O}}
\newcommand{\bbP}{\mathbb{P}}
\newcommand{\bbQ}{\mathbb{Q}}
\newcommand{\bbR}{\mathbb{R}}
\newcommand{\bbS}{\mathbb{S}}
\newcommand{\bbT}{\mathbb{T}}
\newcommand{\bbU}{\mathbb{U}}
\newcommand{\bbV}{\mathbb{V}}
\newcommand{\bbW}{\mathbb{W}}
\newcommand{\bbX}{\mathbb{X}}
\newcommand{\bbY}{\mathbb{Y}}
\newcommand{\bbZ}{\mathbb{Z}}

%% script math capitals
\newcommand{\sA}{\mathscr{A}}
\newcommand{\sB}{\mathscr{B}}
\newcommand{\sC}{\mathscr{C}}
\newcommand{\sD}{\mathscr{D}}
\newcommand{\sE}{\mathscr{E}}
\newcommand{\sF}{\mathscr{F}}
\newcommand{\sG}{\mathscr{G}}
\newcommand{\sH}{\mathscr{H}}
\newcommand{\sI}{\mathscr{I}}
\newcommand{\sJ}{\mathscr{J}}
\newcommand{\sK}{\mathscr{K}}
\newcommand{\sL}{\mathscr{L}}
\newcommand{\sM}{\mathscr{M}}
\newcommand{\sN}{\mathscr{N}}
\newcommand{\sO}{\mathscr{O}}
\newcommand{\sP}{\mathscr{P}}
\newcommand{\sQ}{\mathscr{Q}}
\newcommand{\sR}{\mathscr{R}}
\newcommand{\sS}{\mathscr{S}}
\newcommand{\sT}{\mathscr{T}}
\newcommand{\sU}{\mathscr{U}}
\newcommand{\sV}{\mathscr{V}}
\newcommand{\sW}{\mathscr{W}}
\newcommand{\sX}{\mathscr{X}}
\newcommand{\sY}{\mathscr{Y}}
\newcommand{\sZ}{\mathscr{Z}}

\newcommand\cG{\mathcal G}
\newcommand{\BB}{\mathcal B}
\newcommand{\CC}{\mathcal C}

\newcommand{\simto}{\overset{\sim}{\rightarrow}}
\newcommand{\epi}{\twoheadrightarrow}
\newcommand{\mon}{\hookrightarrow}
\newcommand{\from}{\leftarrow}
\newcommand{\id}{\mathbf{1}}
\newcommand{\ev}{\operatorname{ev}}
\newcommand{\im}{\operatorname{im}}
\newcommand{\push}{\operatorname{push}}
\newcommand{\FF}{\mathbf{F}}
\newcommand{\ZZ}{\mathbf{Z}}
\newcommand{\Fun}{\operatorname{Fun}}
\newcommand{\Maps}{\operatorname{Maps}}
\newcommand{\Mat}{\operatorname{Mat}}
\newcommand{\LP}{\operatorname{LP}}
\newcommand{\coker}{\operatorname{coker}}
\newcommand{\spec}{\operatorname{spec}}

\renewcommand{\_}[1]{\underline{ #1 }}
\renewcommand{\emptyset}{\O}

\DeclareMathOperator{\ext}{ext}

\providecommand{\abs}[1]{\lvert #1 \rvert}
\providecommand{\norm}[1]{\lVert #1 \rVert}
\providecommand{\x}{\times}
\providecommand{\ar}{\rightarrow}
\providecommand{\arr}{\longrightarrow}


%----------Theorems----------

\newtheorem{theorem}{Theorem}[section]
\newtheorem{proposition}[theorem]{Proposition}
\newtheorem{lemma}[theorem]{Lemma}
\newtheorem{corollary}[theorem]{Corollary}

\theoremstyle{definition}
\newtheorem{definition}[theorem]{Definition}
\newtheorem{nondefinition}[theorem]{Non-Definition}
\newtheorem{note}[theorem]{Note}
\newtheorem{example}[theorem]{Example}
\newtheorem{innercustomthm}{Exercise}
\newenvironment{exercise}[1]
  {\renewcommand\theinnercustomthm{#1}\innercustomthm}
  {\endinnercustomthm}

\newenvironment{solution}{\par\noindent\textit{Solution.}\ }{\par}

\numberwithin{equation}{subsection}

\addtolength{\jot}{.05in}

\setlength{\parindent}{0pt}



%----------Title-------------
\title{Honors Analysis HW1}
\author{Andrew Hah}
\begin{document}

\pagestyle{plain}
\begin{center}
{\Large MATH 20700. Honors Analysis Homework 1} \\ 
\vspace{.2in}  
Andrew Hah \\
Due October 7, 2024
\end{center}

\begin{exercise}{9}
    Let $x = A \mid B, x' = A' \mid B'$ be cuts in $\bbQ$. We defined $$x + x' = (A + A') \mid \text{rest of } \bbQ.$$
    \begin{itemize}
        \item [(a)] Show that although $B + B'$ is disjoint from $A + A'$, it may happen in degenerate cases that $\bbQ$ is not the union of $A + A'$ and $B + B'$.
        \begin{proof}
            Clearly $B + B' \cap A + A' = \emptyset$ because $\forall a \in A, b \in B$, we know $a < b$, and $\forall a' \in A', b' \in B'$, we have $a' < b'$. Then $\forall \alpha = a + a' \in A + A'$ and $\forall \beta = b + b' \in B + B'$, $\alpha < \beta$. However, it is not necessarily true that $B + B' \cup A + A' = \bbQ$. For example, when both $x$ and $x'$ are the cut at irrational number $\sqrt{2}$, we find a gap in $\bbQ$. Specifically, there are rational numbers less than $2 \sqrt{2}$ that cannot be represented by the sum of two rational numbers less than $\sqrt{2}$.
        \end{proof}
        \item [(b)] Infer that the definition of $x + x'$ as $(A + A') \mid (B + B')$ would be incorrect. 
            \begin{solution}For addition to be defined on cuts, we want the result of $x + x' = (A + A') \mid (B + B')$ to also be a cut. However, in the case that $(A + A') \cup (B + B') \neq \bbQ$, $x + x'$ would not be a cut since the two disjoint sets do not cover all of $\bbQ$.
            \end{solution}
        \item [(c)] Why did we not define $x \cdot x' = (A \cdot A') \mid \text{rest of } \bbQ$?
        \begin{solution}Defining $x \cdot x' = (A \cdot A') \mid \text{rest of } \mathbb{Q}$ is problematic because the product $A \cdot A'$ results in nonsensical cuts. $A$ and $A'$ contain very small negative rationals, and multiplying them would lead to very large positive rationals, which do not belong in the first set of cut $x \cdot x'$. 
        \end{solution}
    \end{itemize}
\end{exercise}

\begin{exercise}{10}
    Prove that for each cut $x$ we have $x + (-x) = 0^*$.
\end{exercise}
\begin{proof}
    Let $x = A \mid B$. We know $a < b$ for all $a \in A$ and $b \in B$. The additive inverse of $x$ is defined as $-x = \{ r \in \bbQ : \text{for some } b \in B, \text{ not the smallest element of } B, r = -b \}$. Then, the first set of $x + (-x)$ is essentially $a - b$ for all $a \in A$ and $b \in B \setminus \{ \text{smallest element of } B \}$. From above, we know $a < b$, or $a - b < 0$ for all $a \in A, b \in B$. Thus, all elements of the first set of $x + (-x)$ are bounded above by 0 and the second set is the rest of $\bbQ$ (including 0), which is the definition of the cut $0^*$. 
\end{proof}

\begin{exercise}{11}
    A multiplicative inverse of a nonzero cut $x = A \mid B$ is a cut $y = C \mid D$ such that $x \cdot y = 1^*$. 
    \begin{itemize}
        \item [(a)] If $x > 0^*$, what are $C$ and $D$?
        \begin{solution}$C = \{ r \in \bbQ : r \le 0 \text{ or for some } q \in A, qr < 1 \}$, and $D$ is the rest of $\bbQ$. 
        \end{solution}
        \item [(b)] If $x < 0^*$, what are they?
        \begin{solution}$C = \{ r \in \bbQ : \text{for some } q \in A, qr > 1 \}$, and $D$ is the rest of $\bbQ$. 
        \end{solution}
        \item [(c)] Prove that $x$ uniquely determines $y$. 
        \begin{proof}
            For each $x \neq 0^*$, $y$ is as above. Then, since cuts are uniquely defined by the least upper bound of the first set, each cut $y$ is uniquely defined as the cut at $xy = 1$ for $x, y \in \bbR$.
        \end{proof}
    \end{itemize}
\end{exercise}

\begin{exercise}{13}
    Let $b = \sup S$, where $S$ is a bounded nonempty subset of $\bbR$. 
    \begin{itemize}
        \item [(a)] Given $\epsilon > 0$ show that there exists an $s \in S$ with $$b - \epsilon \le s \le b.$$
        \begin{proof}
            Since $b$ is the least upper bound for $S$, we know that $b - \epsilon$ is not an upper bound for $S$. Thus, there exists $s \in S$ such that $s > b - \epsilon$. Since $s \in S$, this implies $s \le b$, since if not, $b$ would not be the least upper bound of $S$. Thus we have found $s$ such that $b - \epsilon \le s \le b$. 
        \end{proof}
        \item [(b)] Can $s \in S$ always be found so that $b - \epsilon < s < b$?
        \begin{solution}
            No. $s \in S$ can be found so that $b - \epsilon < s < b$ only when $b \notin S$. Consider the set $S = \{ 1 \}$. In this case, $b = 1$, and $b - \epsilon < 1$. There are no elements in the open interval $(b - \epsilon, b)$, so we can't find such an $s$. 
        \end{solution}
        \item [(c)] If $x = A \mid B$ is a cut in $\bbQ$, show that $x = \sup A$. 
        \begin{proof}
            If $x = A \mid B$, then $\forall a \in A, a < x$. Thus $x$ is an upper bound for $A$. We want to show that it is the least upper bound. Suppose it is not. Let $y$ be another upper bound for $A$ and $y < x$. Then by the density of $\bbQ$ in $\bbR$, there exists rational $r$ such that $y < r < x$. Since $r \in \bbQ$ and $r < x$, $r \in A$, since $A$ is the set of all rationals less than $x$. However this is a contradiction as $y$ can not be an upper bound for $A$ if there exists $r \in A$ such that $r > y$. Thus, $x = \sup A$.  
        \end{proof}
    \end{itemize}
\end{exercise}

\begin{exercise}{14}
    Prove that $\sqrt{2} \in \bbR$ by showing that $x \cdot x = 2$ where $x = A \mid B$ is the cut in $\bbQ$ with $A = \{ r \in \bbQ : r \le 0 \text{ or } r^2 < 2 \}$.
\end{exercise}
\begin{proof}
    By definition, $x \cdot x = \{ r \in \bbQ : r \le 0 \text{ or } \exists a \in A \text{ s.t. } r = a^2 \} $. By construction of $A$, we know $\forall a \in A, a^2 < 2$, so every element of $x \cdot x$ is $< 2$. 
\end{proof}

\begin{exercise}{18}
    \begin{itemize}
        \item [(a)] Show that each $x_k$ is a digit between 0 and 9. 
        \begin{proof}
            We can write $x \in \bbR$ as $x = N + \sum_{k = 1}^\infty \frac{x_k}{10^k}$, where each $x_k$ is defined recursively as, $$x_k \le 10 \left( x - \left( N + \sum_{j = 1}^{k -1} \frac{x_j}{10^j} \right) \right).$$ Let $y_k$ be the remaining fractional part of $x$ after subtracting $N$ and the first $k - 1$ digits in the expansion, $y_k = x - \left( N + \sum_{j = 1}^{k-1} \frac{x_j}{10^j} \right)$. By construction, $y_k \in [0, 1)$. Then, $x_k$ is the largest integer satisfying $x_k \le 10y_k$. Since $y_k \in [0, 1)$, $10y_k \in [0, 10)$. Thus, $x_k \in \{ 0, \dots, 9 \}$.  
        \end{proof}
        \item [(b)] Show that for each $k$ there is an $l \ge k$ such that $x_l \neq 9$. 
        \begin{proof}
            Suppose not. Then there is an $x$ such that $\forall l \ge k$, $x_l = 9$, i.e., infinite string of nines starting at a certain $x_k$. Then we can express this as $$x = N + \sum_{j = 1}^{k-1} \frac{x_j}{10^j} + \sum_{m = 0}^\infty \frac{9}{10^{k+m}}.$$ Consider the term $\sum_{m = 0}^\infty \frac{9}{10^{k+m}}$. This is a geometric series with first term $\frac{9}{10^k}$ and ratio $\frac{1}{10}$. Then, the sum of this series is simply $\displaystyle \frac{\frac{9}{10^k}}{\frac{9}{10}} = \frac{1}{10^{k - 1}}$. Adding this back to our initial expression for $x$, we get $$x = N + \sum_{j = 1}^{k-1} \frac{x_j}{10^j} + \frac{1}{10^{k - 1}}.$$ Our assumption was that $\forall l \ge k$, $x_l = 9$, but we found that the terms $x_l$ for $l \ge k$ are equivalent to $\frac{1}{10^{k-1}}$, which leads to a contradiction. 
        \end{proof}
        \item [(c)] Conversely, show that for each such expansion $N.x_1x_2\dots$ not terminating in an infinite string of nines, the set $\{ N, N + \frac{x_1}{10}, \dots \}$ is bounded and its least upper bound is a real number $x$ with decimal expansion $N.x_1x_2\dots$. 
        \begin{proof}
            We define a sequence of partial sums $S_k = N + \frac{x_1}{10} + \frac{x_2}{10^2} + \dots + \frac{x_k}{10^k}$. The sequence $\{ S_k \}$ is bounded above. Since each digit $x_k \in \{ 0, \dots, 9 \}$ as we showed above, each partial sum $S_k$ is less than $N + 1$, i.e., for all $k$, $S_k \le N + 1$. Since each partial sum $S_k$ is obtained by adding a nonnegative term $\frac{x_k}{10^k}$ to $S_{k -1}$, the sequence $\{ S_k \}$ is increasing. We know that a bounded, increasing sequence has a least upper bound. Let $x = \sup \{ S_k \}$. For any $k$, $S_k$ approximates $x$ up to the $k$-th digit of the decimal expansion, i.e. $x = \displaystyle \lim_{k \to \infty} S_k = N + \displaystyle \sum_{j = 1}^\infty \frac{x_j}{10^j}$. Thus $x$ has the exact decimal expansion $N.x_1x_2\dots$. 
        \end{proof}
        \item [(d)] Repeat the exercise with a general base in place of 10. 
        \begin{solution}For an arbitrary base $b$,
        \begin{itemize}
            \item  We want to show that each $x_k$ is digit between 0 and $b-1$. We express $x$ as $x = N + \sum_{k = 1}^\infty \frac{x_k}{b^k}$, where each $x_k$ is defined recursively as, $$x_k = \left\lfloor b \left( x - \left( N + \sum_{j = 1}^{k -1} \frac{x_j}{b^j} \right) \right) \right\rfloor.$$ Let $y_k$ be the remaining fractional part of $x$ after subtracting $N$ and the first $k - 1$ digits in the expansion, $y_k = x - \left( N + \sum_{j = 1}^{k-1} \frac{x_j}{10^j} \right)$. Since $0 \le y_k < \frac{1}{b^{k-1}}$, multiplying both sides by $b$ gives us $$0 \le by_k < \frac{1}{b^{k-2}}.$$ Then for $k \ge 2$, $by_k < b$, so $by_k \in [0, b)$ and $x_k = \lfloor by_k \rfloor$ is an integer satisfying $0 \le x_k \le b - 1$. 
            \item We want to show that for each $k$ there is an $l \ge k$ such that $x_l \neq b-1$. Suppose not. Then there is an $x$ such that $\forall l \ge k$, $x_l = b-1$. Then we can express this as $$x = N + \sum_{j = 1}^{k-1} \frac{x_j}{b^j} + \sum_{m = 0}^\infty \frac{b-1}{b^{k+m}}.$$ Consider the term $\sum_{m = 0}^\infty \frac{b-1}{b^{k+m}}$. This is a geometric series with first term $\frac{b-1}{b^k}$ and ratio $\frac{1}{b-1}$. Then, the sum of this series is simply $\frac{1}{b^{k - 1}}$. Adding this back to our initial expression for $x$, we get $$x = N + \sum_{j = 1}^{k-1} \frac{x_j}{b^j} + \frac{1}{b^{k - 1}}.$$ Our assumption was that $\forall l \ge k$, $x_l = b-1$, but we found that the terms $x_l$ for $l \ge k$ are equivalent to $\frac{1}{b^{k-1}}$, which leads to a contradiction. 
            \item We want to show that for each such expansion $N.x_1x_2\dots$ not terminating in an infinite string of $(b-1)$s, the set $\{ N, N + \frac{x_1}{b}, \dots \}$ is bounded and its least upper bound is a real number $x$ with decimal expansion $N.x_1x_2\dots$. Define the sequence of partial sums:
\[
S_n = N + \sum_{k=1}^n \frac{x_k}{b^k}.
\]
Each \( S_n \) represents the truncated expansion of \( x \) up to \( n \) digits in base \( b \). Since each digit \( x_k \) satisfies \( 0 \leq x_k \leq b - 1 \), the maximum value of each term is \( \frac{b - 1}{b^k} \). The sum of the maximum possible values beyond the \( n \)-th term is $\frac{1}{b^n}$. Therefore, for all \( n \):
\[
S_n \leq N + \sum_{k=1}^n \frac{b - 1}{b^k} = N + (b - 1) \left( \sum_{k=1}^n \frac{1}{b^k} \right).
\]
\[
\sum_{k=1}^n \frac{1}{b^k} = \frac{\frac{1}{b} \left(1 - \left( \frac{1}{b} \right)^{n} \right)}{1 - \frac{1}{b}} = \frac{1 - \frac{1}{b^{n}}}{b - 1}.
\]
\[
S_n \leq N + (b - 1) \left( \frac{1 - \frac{1}{b^{n}}}{b - 1} \right) = N + 1 - \frac{1}{b^{n}}.
\]
Thus:
\[
S_n \leq N + 1.
\]
As \( n \to \infty \), \( \frac{1}{b^{n}} \to 0 \), so \( S_n \) is bounded above by \( N + 1 \).
Since all terms \( \frac{x_k}{b^k} \geq 0 \), adding more terms cannot decrease the sum:
\[
S_n \leq S_{n+1}.
\]
Therefore, \( \{ S_n \} \) is a non-decreasing sequence.

Since \( \{ S_n \} \) is non-decreasing and bounded above, it converges to its least upper bound \( x \):
\[
x = \lim_{n \to \infty} S_n.
\]
This \( x \) is a real number whose base \( b \) expansion is given by \( N.x_1x_2x_3\ldots \) because each \( S_n \) includes the digits of \( x \) up to position \( n \).

        \end{itemize}
        \end{solution}
    \end{itemize}
\end{exercise}

\begin{exercise}{19}
    Formulate the definition of greatest lower bound of a set of real numbers. State a greatest lower bound property of $\bbR$ and show that it is equivalent to the least upper bound property of $\bbR$.
\end{exercise}
\begin{solution}
    The greatest lower bound of a set of real numbers $X$ is a real number $l \in \bbR$ such that $\forall x \in X$, $x \ge l$, and for every $y$ that is a lower bound of $X$, $l \ge y$. The greatest lower bound property of $\bbR$ would be that if $X \subset \bbR$ is nonempty and bounded below, then $X$ has a greatest lower bound. To show that this is equivalent to the least upper bound property of $\bbR$, we show that they imply each other. \\
    LUB $\implies$ GLB) Let $X$ be a subset of $\bbR$ that is nonempty and bounded below. Define a new set $Y = \{ -x : x \in X \}$. Since $X$ is nonempty, $Y$ is nonempty. Since $X$ is bounded below, there exists $m \in \bbR$ such that $\forall x \in X$, $x \ge m$. Then for all $x \in X$, $-x \le -m$. This is true because \begin{equation*}
        \begin{gathered}
            x \ge m \\
        x + (-x) \ge m + (-x) \\
        0 \ge m + (-x) \\
        -m \ge -x
        \end{gathered}
    \end{equation*} Thus, we see that $Y$ is bounded above by $-m$. Since $Y$ is nonempty and bounded above, by the least upper bound property, it has a supremum, say $M = \sup Y$. We claim that $-M = \inf X$. It is clear that $-M$ is a lower bound of $X$ since for all $-x \in Y$, $-x \le M$, which implies that for all $x \in X$, $x \ge -M$. Now we prove that this is the greatest lower bound. Suppose there is a greater lower bound of $X$, call it $l$. Then $l > -M$ which implies $-l < M$. Since $-l < M$, we know that $-l \in Y$. Then $l \in X$ by construction, so it is not a lower bound of $X$. Thus, $-M = \inf X$. \\
    The other direction is symmetric, replacing lower with greater.
    
\end{solution}

\begin{exercise}{20}
    Prove that limits are unique. 
\end{exercise}
\begin{proof}
    Suppose $(a_n)$ converges to both $b$ and $b'$. Then, for every $\epsilon > 0$, there exists $N_1$ such that if $n \ge N_1$, then $|a_n - b| < \epsilon/2$ and there also exists $N_2$ such that if $n \ge N_2$, then $|a_n - b'| < \epsilon/2$. Take $N = \max\{N_1, N_2 \}$. Then if $n \ge N$, $|a_n - b| < \epsilon/2$ and $|a_n - b'| < \epsilon/2$. It follows that $|b - b'| \le |b - a_n| + |a_n - b'| < \epsilon/2 + \epsilon/2 = \epsilon$, which implies that $b = b'$. 
\end{proof}

\end{document}
