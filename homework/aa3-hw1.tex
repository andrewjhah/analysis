\documentclass[11pt]{article}
\usepackage{master}
\DeclareMathOperator{\diam}{diam}
\newcommand{\interior}[1]{%
  {\kern0pt#1}^{\mathrm{o}}%
}

\let\eps\varepsilon

\title{Accelerated Analysis HW1}
\author{Andrew Hah}
\begin{document}

\pagestyle{plain}
\begin{center}
{\Large MATH 20510. Accelerated Analysis III} \\
{\Large Homework 1} \\
\vspace{.2in}  
Andrew Hah \\
Due April 2, 2025
\end{center}

\begin{exercise}{11.15}
    \begin{proof}
        First consider two disjoint intervals $(a, b)$ and $(c, d)$ such that $a, c \neq 0$. Note that the set function $\phi$ on each of these intervals is simply the length function, i.e. $\mathrm{vol}$ in one dimension. By definition of $\mathrm{vol}$ on $\bbR$, $\phi((a, b) \cup (c, d)) = \phi((a, b)) + \phi((c, d)) = b - a + c - d$, i.e. $\phi$ is additive on disjoint intervals. We also note that elementary subsets of $(0, 1] \subset \sE$ can be written as a finite union of disjoint intervals, so $\phi$ is additive for all $A, B \in \sR$ when none of the intervals have an open end at $0$. \\
        
        Now consider the case where for some $A \in \sR$, $A = I_1, \dots, I_n \in \sR$ after writing $A$ as a finite union of disjoint intervals and there is an interval $I = (0, b)$ or $(0, b]$ among the intervals. There can only be one since the intervals are disjoint. For simplicity, let this $I = I_1$. Then for the other intervals $I_2, \dots, I_n$, by the above, we have $\phi(I_2 \cup \dots \cup I_n) = \sum_2^n \phi(I_k)$. Since $(I_2 \cup \dots \cup I_n)$ is disjoint from $I_1$, we have that $$\phi(A) = \phi(I_1 \cup (I_2 \cup \dots \cup I_n)) = \phi(I_1) + \phi(I_2 \cup \dots \cup I_n) = \phi(I_1) + \sum_2^n \phi(I_k) = \sum_1^n \phi(I_k)$$ Note for any pair $A, B \in \sR$ that are disjoint, only one can be open at $0$ so we are done. \\

        We see that $\phi$ is not regular. Consider the interval $(0, 1]$, for which $\phi((0, 1]) = 2$, but there is no possible closed subset $F$ of $(0, 1]$ such that $\phi((0, 1]) \le \phi(F) + \eps$. More formally, for any $\delta > 0$, the closed interval $[\delta, 1]$ has $\phi([\delta, 1]) = 1 - \delta$, i.e. no matter what $F$ we choose, $\phi(F) < 1$, so for every closed subset $F$, we have $\phi(F) < 2 - \eps$. \\

        We see that $\phi$ is not countably additive, i.e. there does not exist a countably additive extension $\Tilde{\phi}$ to some $\sigma$-ring $\sS \supseteq \sR$.  Suppose for a contradiction that such a $\Tilde{\phi}$ exists. Fix $b \in (0, 1]$ and consider the countable disjoint cover of $(0, b)$ given by $$(0, b) = \bigcup_{n = 1}^\infty \left[ \frac{b}{n + 1}, \frac{b}{n} \right)$$ Since these intervals are all in $\sR$ and avoid $0$, we can use $\phi$ to compute $$\Tilde{\phi}\left(  \left[ \frac{b}{n + 1}, \frac{b}{n} \right) \right) = \phi \left(  \left[ \frac{b}{n + 1}, \frac{b}{n} \right) \right) = \frac{b}{n} - \frac{b}{n + 1}$$ Then by countable additivity of $\Tilde{\phi}$ we get 
        
        \begin{equation*}
            \begin{split}
                \Tilde{\phi} ((0, b)) = \Tilde{\phi}\left( \bigcup_{n = 1}^\infty \left[ \frac{b}{n + 1}, \frac{b}{n} \right) \right) & = \sum_{n = 1}^\infty \left( \frac{b}{n} - \frac{b}{n + 1} \right) = b
            \end{split}
        \end{equation*}
        However, we also have that $\Tilde{\phi}((0, b)) = \phi((0, b)) = 1 + b$, which is a contradiction. Thus $\phi$ cannot be extended to a countably additive set function on a $\sigma$-ring. 
    \end{proof}
\end{exercise}

\begin{exercise}{1}
    \begin{enumerate} [(a)]
        \item \begin{proof}
            Let $$\mathcal{B} = \left\{ \prod_1^n (a_i, b_i) \subseteq \bbR^n : a_i, b_i \in \bbQ, a_i < b_i \right\}$$ be the collection of open boxes with rational endpoints. Since $\bbQ$ is countable and a finite product of countable sets is countable, $\mathcal{B}$ is countable. We see that any open set $U \subseteq \bbR^n$ can be written as a union of elements from $\mathcal{B}$, i.e. $\mathcal{B}$ forms a basis on $\bbR^n$. This is because since $U$ is open, $\forall x \in U$, there exists an open box $B \in \mathcal{B}$ such that $x \in B \subseteq U$ so $U$ is the union of such boxes. Each such box is in $\mathfrak{M}(m)$, and since $\mathfrak{M}(m)$ is a $\sigma$-ring, $U = \bigcup_1^\infty B_i$ is also in $\mathfrak{M}(m)$.
        \end{proof}
        \item \begin{proof}
            Let $A \in \mathfrak{M}(m)$ and let $\eps > 0$ be given. Since $A \in \mathfrak{M}(m)$, $m(A) = m^*(A) = \inf \sum_1^\infty \mathrm{vol}(A_n)$ where $\inf$ is taken over all countable open covers of $A$. In other words, for any $\eps > 0$ there exists a countable collection of open sets $\{ A_n \}$ such that $A \subseteq \bigcup_1^\infty A_n = G \quad \text{and} \quad \sum_1^\infty \mathrm{vol}(A_n) < m(A) + \eps$. Since $G \supseteq A$ and $G \in \mathfrak{M}(m)$, we have $$m(G \setminus A) = m(G) - m(A) < (m(A) + \eps) - m(A) = \eps$$ Now consider $A^c \in \mathfrak{M}(m)$. Similarly, there exists an open set $U \supseteq A^c$ such that $$m(U \setminus A^c) = m(A \setminus U^c) < \eps$$ Let $F = U^c$. Then $F$ is closed, $F \subseteq A$, and $m(A \setminus F) < \eps$.
        \end{proof}
        \item \begin{proof}
            For two Borel sets $A, B \in \mathscr{B}$, their union $A \cup B$ is clearly also in $\sB$ since it is just one more operation after the countably many operations used to obtain $A$ and $B$. For $A \setminus B$, note that $A \setminus B = A \cap B^c$, which immediately gives that $A \setminus B \in \sB$. Finally, for any $\{ A_n \} \subseteq \sB$, we see that $\bigcup_1^\infty A_n \in \sB$ since the countable union of sets in $\sB$ is still a countable number of operations. Thus $\sB$ is a $\sigma$-ring.
        \end{proof} 
        \item \begin{proof}
            By part b) we have that for $A \in \mathfrak{M}(m)$, and $\forall \eps > 0$, there exist sets $F, G$ such that $F \subset A \subset G$, $F$ closed, $G$ open, and $m(G \setminus A) < \eps$, $m(A \setminus F) < \eps$. We now take a sequence of such approximations, i.e. let $\eps_n = \frac{1}{n}$ for all $n \in \bbN$, so that we obtain a sequence of sets $F_n \subset A$, $F_n$ closed, $m(A \setminus F_n) < \frac{1}{n}$ and $G_n \supset A$, $G_n$ open, $m(G_n \setminus A) < \frac{1}{n}$. We can now define $F = \bigcup_1^\infty F_n$ and $G = \bigcap_1^\infty G_n$. Clearly $F, G$ are both Borel sets. Now $$m(A \setminus F) \le m(A \setminus F_n) < \frac{1}{n} \implies m(A \setminus F) \to 0 \text{ as } n \to \infty$$ And similarly, $$m(G \setminus A) \le m(G_n \setminus A) < \frac{1}{n} \implies m(G \setminus A) \to 0 \text{ as } n \to \infty$$
        \end{proof}
        \item \begin{proof}
            Consider $A, B$ such that $\mu(A) = \mu(B) = 0$. Then for every $\mu$, $$\mu(A \cup B) \le \mu(A) + \mu(B) = 0 \implies \mu(A \cup B) = 0$$ We also see that $$A \setminus B \subseteq A \implies \mu(A \setminus B) \le \mu(A) = 0 \implies \mu(A \setminus B) = 0$$ Finally, let $\{ A_n \}$ be a sequence of sets of measure zero. Then $$\mu \left( \bigcup_1^\infty A_n \right) \le \sum_1^\infty \mu(A_n) = 0 \implies \mu \left( \bigcup_1^\infty A_n \right) = 0$$ Thus the sets of measure zero form a $\sigma$-ring.
        \end{proof}
    \end{enumerate}
\end{exercise}

\begin{exercise}{2}
    \begin{proof}
        Given $x \in \bbR^n$ and $A \subseteq \bbR^n$, consider a countable open cover $\{ A_n \}$ of $A$. Then $\{ x + A_n \}$ is a cover of $x + A$. We also note that $\mathrm{vol}(x + A_n) = \mathrm{vol}(A_n)$. Thus, $$m^*(x + A) \le \sum_1^\infty \mathrm{vol}(x + A_n) = \sum_1^\infty \mathrm{vol}(A_n)$$ Taking the infimum over all such coverings, we obtain $$m^*(x + A) \le m^*(A)$$ Similarly in the other direction, we can find a cover $\{ A_n \}$ of $x + A$. Then $\{ A_n - x \}$ is a cover of $A$, and these volumes are equal, leading to the conclusion that $m^* (A) \le m^*(x + A)$. Putting these together, we have $$m^*(x+A) = m^*(A)$$
    \end{proof}
\end{exercise}

\begin{exercise}{3}
    \begin{proof}
        $m^*(X) = 0$, since $X$ is a line, we can cover it with open intervals of arbitrarily small total length, so when we take $\mathrm{vol}$ and then $\inf$, this can get arbitrarily close to 0. To show that $X \in \mathfrak{M}(m)$, let $X_n = \{ (x, 0) \mid x \in [-n, n] \}$. Then for any $\eps > 0$, let $E_n = [-n, n] \times \left[ - \frac{\eps}{4n}, \frac{\eps}{4n} \right]$. Then, $E_n$ is elementary, $X_n \subseteq E_n$, and $m(E_n) = 2n \cdot \frac{\eps}{2n} = \eps$. Thus, $m^*(E_n \triangle X_n) \le m(E_n) = \eps$. Thus, we have shown that $X_n$ is finitely $m$-measurable, i.e. $X_n \in \mathfrak{M}_F(m)$. Taking the union of these, we get $X = \bigcup_1^\infty X_n$, and thus $X \in \mathfrak{M}(m)$.
    \end{proof}
\end{exercise}

\begin{exercise}{4}
    \begin{proof}
        We are given that $A$ and $B$ are disjoint. By subadditivity of outer measure, we are given that $m^*(A \cup B) \le m^*(A) + m^*(B)$. It suffices to show the reverse inequality. Since $A$ and $B$ are separated by positive distance $d$, any set $E$ of diameter less than $d$ can intersect at most one of $A$ or $B$. Consider a countably open cover of $A \cup B$ such that each set in the cover must have diameter strictly less than $d$. Thus, each set in the cover lies entirely in a neighborhood of $A$ or entirely in a neighborhood of $B$. Putting all sets that intersect $A$ into one family covers $A$, and putting all sets that intersect $B$ into one family covers $B$. Then the total volume of the original cover is exactly the sum of the volumes used to cover $A$ and $B$. Thus taking infima, we get $m^*(A) + m^*(B) \le m^*(A \cup B)$. 
    \end{proof}
\end{exercise}

\begin{exercise}{5}
    \begin{proof}
        Consider the graph on the interval $[-N, N]$ and let $\eps > 0$. Because the interval $[-N, N]$ is compact, $f$ is uniformly continuous on it. We partition $[-N, N]$ into subintervals of length $< \delta$. Over each subinterval $[x_i, x_{i + 1}]$, the values of $f$ differ by at most $\eps$ (by uniform continuity), so the portion of the graph lying above this subinterval can be covered by a rectangle of height $\eps$ and width $< \delta$. The total area of all these rectangles is $$\sum (x_{i + 1} - x_i) \eps = 2N\eps$$ If we let $N \to \infty$, we see that we can cover all of $\mathrm{Graph}(f)$ by countably many rectangles whose total area is bounded. Thus the outer measure is 0, which intuitively makes sense since the height needed to cover each small subinterval can be made arbitrarily small, so the total area of the covering goes to 0.
    \end{proof}
\end{exercise}

\end{document}