\documentclass[12pt]{article}
\usepackage{master}

\title{Honors Analysis HW1}
\author{Andrew Hah}
\begin{document}

\pagestyle{plain}
\begin{center}
{\Large MATH 20310. Accelerated Analysis Homework 3} \\ 
\vspace{.2in}  
Andrew Hah \\
Due October 23, 2024
\end{center}

\begin{exercise}{12}
    \begin{proof}
        Let $\{ G_\alpha : \alpha \in \mathscr{A} \}$ be an open cover of $K$. Since $0 \in K$, at least one of the open sets, let's say $G_{\alpha_0}$, must contain 0. Since $G_{\alpha_0}$ is open, $\exists \epsilon > 0$ such that $(-\epsilon, \epsilon) \subseteq G_{\alpha_0}$. Since $\frac{1}{n}$ converges to 0 as $n \to \infty$, there exists $N \in \bbN$ such that $n \ge N \implies |\frac{1}{n} - 0| < \epsilon \implies \frac{1}{n} \in (-\epsilon, \epsilon) \subset G_{\alpha_0}$. Hence, the points $\frac{1}{n}$ for $n \ge N$ are covered by $G_{\alpha_0}$. Each of the remaining points $\frac{1}{1}, \frac{1}{2}, \dots, \frac{1}{N-1}$, must lie in at least one of the open sets, so we can find $G_{\alpha_1}, G_{\alpha_2}, \dots, G_{\alpha_{N-1}}$ that cover these points. We now have a finite collection of open sets, $G_{\alpha_1}, G_{\alpha_2}, \dots, G_{\alpha_{N-1}}$ that cover $\frac{1}{1}, \frac{1}{2}, \dots, \frac{1}{N-1}$, and $G_{\alpha_0}$ that covers 0 and all $\frac{1}{n}$ for $n \ge N$. Thus, this is a finite subcover of $K$ and $K$ is compact. 
    \end{proof}
\end{exercise}

\begin{exercise}{14}
    \begin{solution}Consider the collection of open sets $\{ U_n \}_{n = 1}^\infty$ where $U_n = \left( \frac{1}{n+1}, 1 \right)$ for $n \in \bbN$. This is clearly an open cover of $(0, 1)$ since each $U_n$ is an open interval and $\forall x \in (0, 1)$, there exists $n \in \bbN$ such that $x \in \left( \frac{1}{n+1}, 1 \right)$. We claim that there is no finite subcover. Suppose we take a finite number of these sets, say $U_{n_1}, U_{n_2}, \dots, U_{n_k}$. Let $N = \max \{ n_1, n_2, \dots, n_k \}$. Ths union of these finite sets covers $\left( \frac{1}{N + 1}, 1 \right)$. Consider $\frac{1}{N + 2}$. This point remains uncovered, and more specifically, anything greater than 0 to the left of $\frac{1}{N+1}$ is uncovered. Thus, no finite subcover exists and $(0, 1)$ is not compact. 
    \end{solution}
\end{exercise}

\begin{exercise}{15}
    \begin{solution}We provide counterexamples for both cases. \\
    If we replace the word ``compact" in Theorem 2.36 with ``closed," then we can consider a collection $\{ K_n \}$ of closed subsets such that each $K_n \coloneqq [n, \infty)$. Then clearly the intersection of every finite subcollection of $\{ K_n \}$ is nonempty since a finite intersection of, say $K_1, \dots, K_m$, is $[m, \infty)$. However, $\bigcap_{n = 1}^\infty K_n$ is empty because as $n \to \infty$, no points are in all the sets since the lower bound grows without bound. This also disproves the corollary since $K_n \supset K_{n + 1} \supset K_{n + 2} \supset \dots$ \\
    If we replace ``compact" with ``bounded," then we consider a collection $\{ K_n \}$ of bounded subsets such that each $K_n \coloneqq \left( 0, \frac{1}{n} \right)$. Clearly each subset is bounded by $(0, 1)$. We see that the intersection of every finite subcollection is nonempty since for $K_1, \dots, K_m$, we get the intersection $\left( 0, \frac{1}{m} \right)$. However, $\bigcap_{n = 1}^\infty$ is empty because as $n \to \infty$, no points are in all the sets since the upper bound converges to 0. This also disproves the corollary since $K_n \supset K_{n + 1} \supset K_{n + 2} \supset \dots$  
    \end{solution}
\end{exercise}

\begin{exercise}{22}
    \begin{proof}
        Consider the set of points which have only rational coordinates, $\bbQ^k = \{ (q_1, \dots, q_k) : \forall 1 \le i \le k, q_i \in \bbQ \} \subset \bbR^k$. Then $\forall x \in \bbR^k$ with $x \notin \bbQ^k$, $x = (x_1, \dots, x_k)$. Then for every $x_i$ for $1 \le i \le k$, given $\epsilon > 0$, there exists $q_i \in \bbQ$ such that $|x_i - q_i| < \frac{\epsilon}{k}$ since the rationals are dense in the reals. Then if we denote $q = (q_1, \dots, q_k)$, we see that $$|x - q| = \sqrt{\sum_{i = 1}^k |x_i - q_i|^2} < \frac{\epsilon}{\sqrt{k}} \le \epsilon$$ Thus, any $\epsilon$ neighborhood of $x \in \bbR^k, x \notin \bbQ^k$ has a point $q \in \bbQ^k$ with $q \neq p$, i.e., $x$ is a limit point of $\bbQ^k$. This implies that $\forall x \in \bbR^k$, either $x \in \bbQ^k$ or $x$ is a limit point of $\bbQ^k$, which is the definition of $\bbQ^k$ being dense in $\bbR^k$. Then, since $\bbQ^k$ is a finite product of $\bbQ$ which we know is countable, $\bbQ^k$ is a countable dense subset of $\bbR^k$. Hence $\bbR^k$ is separable. 
    \end{proof}
\end{exercise}

\begin{exercise}{23}
    \begin{proof}
        Since our metric space $X$ is separable, it has a countable dense subset, say $S$. Let $H \coloneqq \{ B \left( s, \frac{1}{n} \right) : s \in S, n \in \bbN \}$. In words, this is the set of neighborhoods with center in $S$ and radius $\frac{1}{n}$ for $n \in \bbN$. We see that $H$ is countable because it is bijective to $S \times \bbN$ which is the product of two countable sets. Then $\forall x \in X$, and every open set $G \subset X$ such that $x \in G$, we have that there exists $B(x, \epsilon) \subset G$ from the openness of $G$. We also have that there exists a ball in $H$ centered at $h \in S$ such that if we take $n > \frac{2}{\epsilon}$, $x \in B(h, \frac{1}{n})$. This is by the density of $S$ in $X$. Consider $y \in B(h, \frac{1}{n})$. Then $$d(x, y) \le d(x, h) + d(y, h) \le \frac{1}{n} + \frac{1}{n} = \frac{2}{n} < \epsilon$$ This implies that $y \in B(x, \epsilon)$, and since this $y$ was arbitrary, $B(h, \frac{1}{n}) \subseteq B(x, \epsilon)$. Thus, for every $x \in X$ and every open set $G \subset X$, we have that $x \in B(h, \frac{1}{n}) \subseteq B(x, \epsilon) \subset G$, hence $H$ is a countable base for $X$. 
    \end{proof}
\end{exercise}

\begin{exercise}{24}
    \begin{proof} 
        Fix \(\delta > 0\) and pick \(x_1 \in X\). Having chosen \(x_1, \dots, x_j \in X\), choose \(x_{j + 1} \in X\) if possible such that \(d(x_i, x_{j + 1}) \ge \delta\) for all \(i = 1, \dots, j\). If such a process were to continue indefinitely, we would obtain an infinite set \(X' = \{x_1, x_2, \dots\}\) where \(d(x_i, x_j) \ge \delta\) for all \(i \neq j\). However, this would contradict the assumption that every infinite subset of \(X\) has a limit point. Specifically, the set \(X'\) would not have a limit point because for any \(p \in X\) and any \(\epsilon < \frac{\delta}{2}\), the ball \(B(p, \epsilon)\) contains at most one element of \(X'\), meaning no point in \(X\) can be a limit point of \(X'\). Therefore, the process must stop after a finite number of steps, say \(k\), meaning we have chosen a finite set of points \(\{x_1, \dots, x_k\}\) such that every point in $X$ is within a distance $\delta$ of one of these points. In other words, this set of points covers $X$ with neighborhoods of radius $\delta$. Now, let \(\delta = \frac{1}{n}\) for \(n = 1, 2, \dots\). For each \(n\), we obtain a finite cover of \(X\) by balls of radius \(\frac{1}{n}\), with centers at a finite set of points. Taking the union of these finite sets gives a countable set \(S\), since the union of countably many finite sets is countable. Finally, \(S\) is dense in \(X\) because \(\forall x \in X\) and \(\forall \epsilon > 0\), we can choose \(n\) such that \(\frac{1}{n} < \epsilon\), and by construction, there is a point \(y \in S\) such that \(d(x, y) < \frac{1}{n} < \epsilon\). Thus, \(S\) is a countable dense subset of \(X\), and hence \(X\) is separable. 
    \end{proof}
\end{exercise}

\begin{exercise}{25}
    \begin{proof}
        For every $x \in K$, take the open ball $B(x, \frac{1}{n})$. The collection of such open balls is an open cover of $K$. Since $K$ is compact, there exists a finite subcover, which means that there exist $x_1, \dots, x_k$ such that the collection $\mathcal{B} = B(x_1, \frac{1}{n}) \cup \dots \cup B(x_k, \frac{1}{n})$ covers $K$. Then, $\forall x \in K$ and every open set $G \subset X$, $x \in B(x, \epsilon) \subset G$ for $\epsilon > 0$. Also, $x \in B(x_i, \frac{1}{n})$ for some $x_i \in \{ x_1, \dots, x_k \}$. Taking $n > \frac{2}{\epsilon}$ gives that $x \in B(x_i, \frac{1}{n}) \subset B(x, \epsilon) \subset G$. Thus, $\mathcal{B}$ is a countable base for $K$. We next show that $K$ is separable. Let us take the set of centers $D = \{ x_1, \dots, x_k \}$. $D$ is countable since $\mathcal{B}$ is countable. Moreover, $D$ is dense in $K$ because $\forall x \in K$ and $\forall \epsilon > 0$, there exists some $B(x_i, \frac{1}{n}) \in \mathcal{B}$ with $n > \frac{1}{\epsilon}$ such that $x \in B(x_i, \frac{1}{n}) \subset B(x, \epsilon)$. Thus, $D$ is a countable dense subset of $K$, and $K$ is separable. 
    \end{proof}
\end{exercise}

\begin{exercise}{26}
    \begin{proof} 
        By exercises 23 and 24, we know that \( X \) has a countable base. This means that every open cover of \( X \) has a countable subcover. Assume for the sake of contradiction that there exists a countable open cover \( \{ G_n : n \in \mathbb{N} \} \) such that no finite subcollection of \( \{ G_n \} \) covers \( X \). Define \( F_n = X \setminus (G_1 \cup \dots \cup G_n) \). By assumption, each \( F_n \) is non-empty. Note that \( F_1 \supseteq F_2 \supseteq \dots \) since each successive \( F_n \) is missing coverage from one more open set. Additionally, we have \( \bigcap_{n = 1}^\infty F_n = \emptyset \) because \( \{ G_n \} \) is an open cover of \( X \), so eventually the sets \( G_n \) must cover the entire space. We construct an infinite set \( E = \{ x_1, x_2, \dots \} \), where each \( x_n \in F_n \). This set \( E \) is infinite by construction. By the exercise's assumption, \( E \) has a limit point \( p \in X \). Since \( \{ G_n \} \) is an open cover, there exists some \( m \in \mathbb{N} \) such that \( p \in G_m \). For \( n \ge m \), the points \( x_n \in F_n \subset X \setminus G_m \), so \( x_n \notin G_m \). Now, take the points \( x_1, x_2, \dots, x_{m-1} \). Define \( \epsilon = \min \{ d(p, x_1), d(p, x_2), \dots, d(p, x_{m-1}) \} \). The open ball \( B(p, \frac{\epsilon}{2}) \) does not contain any points \( x_1, \dots, x_{m-1} \). However, since \( p \) is a limit point of \( E \), the ball \( B(p, \frac{\epsilon}{2}) \) must contain some points from \( E \setminus \{ p \} \), contradicting the fact that for \( n \ge m \), \( x_n \notin G_m \). This implies that there must exist a finite subcollection of \( \{ G_n \} \) that covers \( X \). Therefore, \( X \) is compact. 
    \end{proof}
\end{exercise}

\begin{exercise}{27}
    \begin{proof}
        We first show that $P$ is perfect. Let $p$ be a limit point of $P$. We show $p$ must be in $P$. Since $p$ is a limit point of $P$, every neighborhood of $p$ must contain a point $q \in P$ such that $q \neq p$. This means that $\forall \epsilon > 0$, $q \in B(p, \epsilon)$ where $q$ is a condensation point. By definition, every neighborhood of $q$ must contain uncountably many points. Then if we take a small enough neighborhood, $B(q, \delta) \subset B(p, \epsilon)$, the neighborhood around $q$ that contains uncountably many points must be contained in the neighborhood around $p$. Thus, every neighborhood around $p$ also contains uncountably many points, so $p \in P$ is a condensation point, hence $P$ is closed. Now, consider any point $p \in P$. By definition, every neighborhood of $p$ must contain uncountably many points of $P$, so there must be a point $q$ in this neighborhood such that $q \in P$ with $q \neq p$, making $p$ a limit point of $P$. Hence, $P$ is closed and every point of $P$ is a limit point, thus $P$ is perfect. \\
        Let $\{ V_n \}$ be a countable base for $\bbR^k$ and define $W$ as the union of those $V_n$ for which $E \cap V_n$ is at most countable. By construction, $W$ is the union of countably many neighborhoods $V_n$ where $E \cap V_n$ is countable, so $W$ itself contains at most countably many points of $E$. Thus, $E \cap W$ is countable. Now, consider the set $(E \setminus P) \cap E$. If $p \in (E \setminus P)$, then by definition $p$ is not a condensation point, so there exists a neighborhood of $p$ that contains at most countably many points of $E$. This implies that $p$ belongs to some $V_n \subset W$, so $p \in W$. Therefore, $(E \setminus P) \cap E \subset W$, and since $W$ contains at most countable many points of $E$, it follows that $(E \setminus P) \cap E$ is countable. 
    \end{proof}
\end{exercise}

\begin{exercise}{28}
    \begin{proof}
        Case 1) $F$ is countable. If \( F \) is countable, we can set \( P = \emptyset \) and \( C = F \). Since the empty set is trivially perfect, the decomposition \( F = P \cup C \) satisfies the theorem. \\
        Case 2) \( F \) is uncountable. Define \( P \) as the set of all condensation points of \( F \). Suppose \( \{x_n\} \) is a sequence in \( P \) that converges to some \( x \in X \). Since \( F \) is closed and \( P \subseteq F \), the limit \( x \) must also belong to \( F \). Consider any neighborhood \( U \) of \( x \). Since \( x_n \to x \), there exists \( N \) such that for all \( n \geq N \), \( x_n \in U \). Each \( x_n \) is a condensation point, so \( U \) contains uncountably many points of \( F \). Therefore, \( x \) is also a condensation point of \( F \) because every neighborhood of \( x \) contains uncountably many points of \( F \). Hence, \( x \in P \), proving that \( P \) is closed. We also show that $P$ has no isolated points. Assume, for contradiction, that \( p \in P \) is an isolated point in \( P \). Then there exists a neighborhood \( U \) of \( p \) such that \( U \cap P = \{p\} \). However, since \( p \) is a condensation point of \( F \), \( U \) must contain uncountably many points of \( F \). These points must lie in \( F \setminus P \), as \( U \cap P = \{p\} \). But points in \( F \setminus P \) are not condensation points, meaning each such point has a neighborhood intersecting \( F \) in at most countably many points. This contradicts the fact that \( U \) contains uncountably many points of \( F \). Therefore, \( P \) cannot have any isolated points. This makes $P$ perfect. \\
        Next, we need to show that $C$ is at most countable. Since $X$ is separable, let $\{ V_n \}$ be a countable base for $X$. Define $W$ as the union of those $V_n$ for which $F \cap V_n$ is at most countable. By construction, $W$ is the union of countably many neighborhoods $V_n$ where $F \cap V_n$ is countable, so $W$ itself contains at most countably many points of $F$. Thus, $F \cap W$ is countable. Therefore, $C \subset \bigcup_{n = 1}^\infty (V_n \cap F)$. Each $V_n \cap F$ is countable by construction. A countable union of countable sets is countable, hence $C$ is at most countable. 
    \end{proof}
\end{exercise}

\begin{exercise}{29}
        \begin{proof} 
            Let \( U \subseteq \mathbb{R}^1 \) be an open set. Since \( U \) is open, for every point \( x \in U \), there exists an open interval \( I_x = (a_x, b_x) \) such that \( x \in I_x \subseteq U \). Define \( I_x \) as the largest open interval around \( x \) that is contained in \( U \), meaning that \( I_x \) cannot be extended further while remaining in \( U \). Now, let \( \mathcal{I} = \{ I_x : x \in U \} \) be the collection of all such maximal open intervals. Since \( I_x \) is maximal, any two distinct intervals in \( \mathcal{I} \) must be disjoint. If two intervals \( I_x = (a_x, b_x) \) and \( I_y = (a_y, b_y) \) were not disjoint, then their union could form a larger interval, contradicting the maximality of \( I_x \) and \( I_y \). Next, we claim that the collection \( \mathcal{I} \) is at most countable. Since \( \mathbb{R}^1 \) is separable, it contains a countable dense subset \( D \subset \mathbb{R}^1 \). We can assign each disjoint interval \( I_x = (a_x, b_x) \) to a distinct point in \( D \) because each interval contains at least one point from the dense set \( D \). Therefore, the collection of disjoint intervals \( \mathcal{I} \) must be at most countable, as each interval can be associated with a unique point from \( D \). Thus, every open set in \( \mathbb{R}^1 \) is the union of at most countably many disjoint open intervals. 
        \end{proof}
\end{exercise}

\end{document}