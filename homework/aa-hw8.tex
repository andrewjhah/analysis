\documentclass[11pt]{article}
\usepackage{master}
\DeclareMathOperator{\diam}{diam}
\newcommand{\interior}[1]{%
  {\kern0pt#1}^{\mathrm{o}}%
}

\title{Accelerated Analysis HW8}
\author{Andrew Hah}
\begin{document}

\pagestyle{plain}
\begin{center}
{\Large MATH 20310. Accelerated Analysis Homework 8} \\ 
\vspace{.2in}  
Andrew Hah \\
\end{center}

\begin{exercise}{10}
    \begin{proof}
        Suppose, for the sake of contradiction, that $f: X \to Y$ is not uniformly continuous. Then there exists some $\epsilon > 0$ such that for every $\delta > 0$ we can find points in $X$ whose images under $f$ are separated by at least $\epsilon$ in $Y$ even though their preimages are arbitrarily close in $X$. More precisely, if $f$ is not uniformly continuous, there exist sequences $\{p_n\}$ and $\{q_n\}$ in $X$ such that
\[
d_X(p_n, q_n) \to 0 \quad \text{as } n \to \infty
\]
but
\[
d_Y(f(p_n), f(q_n)) > \epsilon \quad \text{for all } n.
\]

Since $X$ is compact, it is sequentially compact. Thus, from the infinite sequence $\{p_n\}$, we may extract a convergent subsequence $\{p_{n_k}\}$ such that $p_{n_k} \to x$ for some $x \in X$.

Because $d_X(p_n, q_n) \to 0$, it follows that $q_{n_k}$ must also approach $x$. To see this, note that
\[
d_X(q_{n_k}, x) \leq d_X(q_{n_k}, p_{n_k}) + d_X(p_{n_k}, x).
\]
As $k \to \infty$, $d_X(q_{n_k}, p_{n_k}) \to 0$ and $d_X(p_{n_k}, x) \to 0$, so $d_X(q_{n_k}, x) \to 0$ as well. Thus, $q_{n_k} \to x$.

Since $f$ is continuous, we have
\[
f(p_{n_k}) \to f(x) \quad \text{and} \quad f(q_{n_k}) \to f(x).
\]

It follows that $d_Y(f(p_{n_k}), f(q_{n_k})) \to d_Y(f(x), f(x)) = 0$. However, this contradicts our original assumption that $d_Y(f(p_n), f(q_n)) > \epsilon$ for all $n$. We have found a subsequence for which the images under $f$ are arbitrarily close, contradicting the uniform non-continuity assumption.

Therefore, our assumption was false and $f$ must be uniformly continuous on $X$.
    \end{proof}
\end{exercise}

\begin{exercise}{11}
    \begin{proof}
        First, we prove that if $f: X \to Y$ is uniformly continuous and
$\{x_n\}$ is a Cauchy sequence in $X$, then $\{f(x_n)\}$ is a
Cauchy sequence in $Y$.

Let $\epsilon > 0$. Since $f$ is uniformly continuous, there exists
$\delta > 0$ such that whenever $d_X(x,y) < \delta$, we have
$d_Y(f(x), f(y)) < \epsilon$.

Because $\{x_n\}$ is a Cauchy sequence in $X$, there is an
$N \in \mathbb{N}$ such that for all $m,n > N$, $d_X(x_m, x_n) < \delta$.
It follows that $d_Y(f(x_m), f(x_n)) < \epsilon$ for all $m,n > N$.
Hence, $\{f(x_n)\}$ is Cauchy in $Y$.

\end{proof}

\begin{proof}

Next, we give an alternative proof of Exercise 13 using
this result.

Let $X$ be a metric space, $E$ a dense subset of $X$, and $f: E \to \mathbb{R}$
a uniformly continuous function. We want to show $f$ can be extended to a
continuous function $g: X \to \mathbb{R}$.

Fix $p \in X$. Since $E$ is dense in $X$, choose a sequence $\{p_n\} \subset E$
such that $p_n \to p$. Then $\{p_n\}$ is a Cauchy sequence in $X$, and by the
previous result, $\{f(p_n)\}$ is a Cauchy sequence in $\mathbb{R}$. Since
$\mathbb{R}$ is complete, $\{f(p_n)\}$ converges. Define
\[
g(p) = \lim_{n \to \infty} f(p_n).
\]

If $\{q_n\} \subset E$ also converges to $p$, then $\{q_n\}$ is Cauchy in $X$
and hence $\{f(q_n)\}$ is Cauchy in $\mathbb{R}$. By considering differences,
one can show $\lim_{n \to \infty} f(q_n) = \lim_{n \to \infty} f(p_n)$, ensuring
$g$ is well-defined.

If $p \in E$, then taking $p_n = p$ for all $n$ immediately gives $g(p) = f(p)$,
so $g$ extends $f$.

To show $g$ is continuous, let $\epsilon > 0$. By the uniform continuity of $f$,
there is a $\delta > 0$ such that if $d_X(x,y) < \delta$, then
$|f(x)-f(y)| < \epsilon$. Suppose $d_X(p,x) < \delta$ and let $\{p_n\}$, $\{x_n\}$
be sequences in $E$ converging to $p$ and $x$, respectively. For large enough $n$,
$d_X(p_n, x_n)$ will also be small enough that $|f(p_n) - f(x_n)| < \epsilon$.
Taking limits as $n \to \infty$, we get $|g(p)-g(x)| \leq \epsilon$,
proving continuity of $g$.
    \end{proof}
\end{exercise}

\begin{exercise}{12}
    \begin{proof}
        Let $(X, d_X)$, $(Y, d_Y)$, and $(Z, d_Z)$ be metric spaces. Suppose $g: X \to Y$ and $f: Y \to Z$ are uniformly continuous. We aim to show that $f \circ g: X \to Z$ defined by $(f \circ g)(x) = f(g(x))$ is uniformly continuous.

Since $f$ is uniformly continuous, for every $\epsilon > 0$ there exists $\delta_1 > 0$ such that for all $y, y' \in Y$,
\[
d_Y(y, y') < \delta_1 \implies d_Z(f(y), f(y')) < \epsilon.
\]

Similarly, since $g$ is uniformly continuous, for this given $\delta_1 > 0$, there exists $\delta > 0$ such that for all $x, x' \in X$,
\[
d_X(x, x') < \delta \implies d_Y(g(x), g(x')) < \delta_1.
\]

Now consider the composition $f \circ g$. Let $\epsilon > 0$ be given. For this $\epsilon$, we have found a $\delta > 0$ as above. If $x, x' \in X$ satisfy $d_X(x, x') < \delta$, then $d_Y(g(x), g(x')) < \delta_1$. Applying the condition for $f$, we get
\[
d_Z((f \circ g)(x), (f \circ g)(x')) = d_Z(f(g(x)), f(g(x'))) < \epsilon.
\]
    \end{proof}
\end{exercise}

\begin{exercise}{13}
    \begin{proof}
        Let $X$ be a metric space and $E$ a dense subset of $X$. Suppose $f: E \to \mathbb{R}$ is uniformly continuous. We wish to construct a continuous extension $g: X \to \mathbb{R}$ of $f$.

For each $p \in X$ and each positive integer $n$, define
\[
V_n(p) = \{ q \in E : d(p,q) < 1/n \}.
\]
Since $E$ is dense in $X$, each $V_n(p)$ is nonempty.

By the uniform continuity of $f$, for every $\epsilon > 0$ there exists $\delta > 0$ such that if $\diam(A) < \delta$ for some $A \subset X$, then $\diam(f(A)) < \epsilon$.

Consider the sets $f(V_n(p))$. Since $V_n(p)$ is contained in a ball of radius $1/n$ about $p$, the diameter of $V_n(p)$ tends to zero as $n \to \infty$. By uniform continuity, the diameter of $f(V_n(p))$ also tends to zero. Thus, the sets $f(V_n(p))$ form a nested sequence of sets in $\mathbb{R}$ with diameters tending to zero.

This implies that
\[
\bigcap_{n=1}^{\infty} \overline{f(V_n(p))}
\]
is a nonempty intersection of closed intervals (or sets) whose lengths tend to zero. Such an intersection consists of exactly one point. Denote this point by $g(p)$.

We define $g: X \to \mathbb{R}$ by $g(p)$ as above. If $p \in E$, then $V_n(p)$ eventually becomes very small and $f(V_n(p))$ becomes arbitrarily close to $f(p)$. Thus, the intersection above is just the singleton $\{ f(p) \}$, proving that $g$ extends $f$.

To check continuity of $g$, let $\epsilon > 0$. By uniform continuity of $f$, choose $\delta > 0$ such that whenever $\diam(A) < \delta$, $\diam(f(A)) < \epsilon$. If $p, x \in X$ and $d(p,x) < \delta$, then for sufficiently large $n$, both $V_n(p)$ and $V_n(x)$ lie in a ball of radius $1/n < \delta$, so $\diam(V_n(p) \cup V_n(x)) < \delta$. Thus, $\diam(f(V_n(p)) \cup f(V_n(x))) < \epsilon$. Since $g(p)$ and $g(x)$ belong to the closures of these sets, $|g(p)-g(x)| \leq \epsilon$. Hence, $g$ is uniformly continuous and therefore continuous on $X$.
    \end{proof}
\end{exercise}

\begin{exercise}{14}
    \begin{proof}
        Let $I = [0,1]$ and let $f: I \to I$ be continuous. Consider the function
\[
g(x) = f(x) - x.
\]
Since $f$ is continuous and $x \mapsto x$ is continuous, their difference $g$ is also continuous on $[0,1]$.

Note that $f(I) \subseteq I$, so $f(0) \in [0,1]$ and $f(1) \in [0,1]$. This gives
\[
g(0) = f(0) - 0 \geq 0 \quad \text{and} \quad g(1) = f(1) - 1 \leq 0.
\]

Since $g$ is continuous, by the Intermediate Value Theorem there exists $x \in [0,1]$ such that $g(x) = 0$. But this means $f(x) - x = 0$, or $f(x) = x$.

Thus, $f$ has at least one fixed point in $[0,1]$.
    \end{proof}
\end{exercise}

\begin{exercise}{15}
    \begin{proof}
        Suppose $f:\mathbb{R} \to \mathbb{R}$ is continuous and open, but not monotonic.
Then there exist points $x_1 < x_2 < x_3$ such that $f(x_2)$ is not
ordered between $f(x_1)$ and $f(x_3)$ in a monotone way. Without loss of
generality, assume
\[
f(x_2) > f(x_1) \quad \text{and} \quad f(x_2) > f(x_3).
\]

This means $x_2$ is a local maximum of $f$. By the continuity of $f$, there is
an open interval $I$ around $x_2$ where $f(x) \le f(x_2)$ for all $x \in I$, and
$f(x) < f(x_2)$ for all $x \neq x_2$ in $I$.

Consider a smaller open interval $J \subset I$ around $x_2$. Because
$f$ attains its strict maximum at $x_2$ on $J$, the set $f(J)$ cannot be open in
$\mathbb{R}$. Any open neighborhood of $f(x_2)$ would include points larger than
$f(x_2)$, which are not attained by $f$ near $x_2$, contradicting the openness
of $f$.

Hence, our assumption that $f$ is not monotonic leads to a contradiction, so $f$ must be monotonic.
    \end{proof}
\end{exercise}

\begin{exercise}{16}
    \begin{proof}
        The function $[x]$, the greatest integer less than or equal to $x$, is
continuous for all $x$ that are not integers. However, it has a jump
discontinuity at every integer. More explicitly, if $n$ is an integer,
then
\[
\lim_{x \to n^-} [x] = n-1 \quad \text{and} \quad [n] = n.
\]

Similarly, the fractional part function $(x) = x - [x]$ is continuous for
all non-integers but has a jump discontinuity at every integer $n$. In
particular,
\[
\lim_{x \to n^-} (x) = 1 \quad \text{and} \quad (n) = 0.
\]

Thus, both $[x]$ and $(x)$ are discontinuous at all integers and continuous
elsewhere.
    \end{proof}    
\end{exercise}

\begin{exercise}{17}
    Let $f$ be a real function defined on the interval $(a,b)$. A simple (or
jump) discontinuity of $f$ at $x$ is characterized by the existence of the
left and right limits $f(x-)$ and $f(x+)$, which are finite but satisfy
$f(x-) \neq f(x+)$.

Define
\[
E = \{ x \in (a,b) : f(x-) < f(x+) \}.
\]
(The case $f(x-) > f(x+)$ is treated similarly.)

For each $x \in E$, there exists a rational number $p$ such that
\[
f(x-) < p < f(x+).
\]
Since $f(x-) = \lim_{t \to x^-} f(t)$, there is an interval to the left of
$x$ on which $f(t) < p$. Likewise, since $f(x+) = \lim_{t \to x^+} f(t)$, there
is an interval to the right of $x$ on which $f(t) > p$. Thus, we can find
rational numbers $q$ and $r$ with $q < x < r$ such that
\[
\text{for } a < q < t < x,\ f(t) < p
\quad \text{and} \quad
\text{for } x < t < r < b,\ f(t) > p.
\]

Thus, to each $x \in E$ we associate a triple $(p,q,r)$ of rationals:
\[
f(x-) < p < f(x+), \quad a < q < x, \quad x < r < b,
\]
together with the condition that $f(t) < p$ for $t \in (q,x)$ and
$f(t) > p$ for $t \in (x,r)$.

The set of all such triples of rationals $(p,q,r)$ is countable since $\mathbb{Q}$ is countable.
We now show that each triple $(p,q,r)$ can arise from at most one $x \in E$.
Suppose there were two points $x_1 < x_2$ in $E$ that produce the same triple
$(p,q,r)$. By construction, for $t \in (q,x_1)$, $f(t) < p$, and for
$t \in (x_1,r)$, $f(t) > p$. This already determines a unique position $x_1$
where $f$ "jumps" from values less than $p$ to values greater than $p$. Hence,
it is impossible for the same triple $(p,q,r)$ to describe another point $x_2$
since this would contradict the uniqueness.

Therefore, each triple corresponds to at most one point in $E$. Hence, $E$ is
at most countable. A similar argument shows that the set of points where
$f(x-) > f(x+)$ is at most countable.

Combining these results, the set of points at which $f$ has a simple
discontinuity is the union of two at most countable sets, hence it is at most
countable.
\end{exercise}

\begin{exercise}{18}
    \begin{proof}
        Let $p \in \mathbb{R}$ be irrational. We need to show $\lim_{x \to p} f(x) = f(p) = 0$.

Given $\epsilon > 0$, choose $N \in \mathbb{N}$ such that $1/N < \epsilon$. Among the rational numbers, those with denominator at most $N$ (in lowest terms)
are isolated points. Hence, we can choose $\delta > 0$ so that the interval
$(p-\delta, p+\delta)$ contains no rational number with denominator $\le N$,
except possibly a finite number we can exclude by further shrinking $\delta$. If $x \in (p-\delta, p+\delta)$ and $x$ is rational, say $x = m/n$ in lowest terms,
then $n > N$. Thus $f(x) = 1/n < 1/N < \epsilon$. If $x$ is irrational, then
$f(x)=0 < \epsilon$. This shows that for all $|x-p|<\delta$, $|f(x)-f(p)|<\epsilon$, and hence
$\lim_{x\to p} f(x)=0=f(p)$. Therefore, $f$ is continuous at every irrational point.

Let $p = m/n$ be rational, with $m \in \mathbb{Z}$, $n \in \mathbb{N}$, and $m,n$ having no common divisor. Then $f(p) = 1/n$. We claim that $\lim_{x\to p} f(x) = 0$. Given $\epsilon > 0$, choose $N$ such that $1/N<\epsilon$. As before, we can choose $\delta > 0$ so that for all $x \in (p-\delta,p+\delta)$, if $x$ is rational $m_x/n_x$ in lowest terms, then $n_x > N$, ensuring $f(x)=1/n_x<1/N<\epsilon$. If $x$ is irrational, $f(x)=0<\epsilon$. Thus $\lim_{x\to p} f(x)=0$, but $f(p)=1/n$. Hence, $f$ has a jump discontinuity
at every rational $p$.
    \end{proof}
\end{exercise}

\end{document}