\documentclass[11pt]{article}
\usepackage{master}
\DeclareMathOperator{\diam}{diam}
\newcommand{\interior}[1]{%
  {\kern0pt#1}^{\mathrm{o}}%
}

\title{Accelerated Analysis HW5}
\author{Andrew Hah}
\begin{document}

\pagestyle{plain}
\begin{center}
{\Large MATH 20310. Accelerated Analysis Homework 5} \\ 
\vspace{.2in}  
Andrew Hah \\
Due November 6, 2024
\end{center}

\begin{exercise}{3}
    \begin{proof}
        We first prove by induction that $s_n < 2$ for $n = 1, 2, \dots$. Base case $(n = 1)$: $s_1 = \sqrt{2} < 2$. Inductive step: We assume that $s_n < 2$. We show $s_{n+1} < 2$. We see that $$s_{n+1} = \sqrt{2 + \sqrt{s_n}} < \sqrt{2 +\sqrt{2}}$$ Now we just need to show that $\sqrt{2 + \sqrt{2}} < 2$. Squaring both sides, we get $2 + \sqrt{2} < 4$ which holds. Thus, $s_n < 2$ for all $n \in \bbN$. This shows that the sequence is bounded above, and thus it suffices to show that $\{ s_n \}$ is non-decreasing to prove convergence. We must show that for all $n \in \bbN$, $s_{n + 1} \ge s_n$. We prove by induction on $n$. Base case $(s_2 \ge s_1)$: $\sqrt{2 + \sqrt{\sqrt{2}}} \ge \sqrt{2}$ holds. Inductive step: We assume that $s_{k+1} \ge s_{k}$ for all $k \le n$. We show $s_{n+1} \ge s_n$. \begin{equation*}
            \begin{split}
                s_{n+1} & \ge s_n \\
                \sqrt{2 + \sqrt{s_n}} & \ge \sqrt{2 + \sqrt{s_{n-1}}} \\
                2 + \sqrt{s_n} & \ge 2 + \sqrt{s_{n - 1}} \\
                s_n & \ge s_{n-1}
            \end{split}
        \end{equation*}
        This holds by our inductive hypothesis, so we have proven that $s_{n+1} \ge s_n$ for all $n$. Thus, since $\{ s_n \}$ is non-decreasing and bounded above, it converges. 
    \end{proof}
\end{exercise}

\begin{exercise}{5}
    \begin{proof}
        Let $E_s$ be the set of all subsequential limits of $\{ s_n \}$, and $\limsup_{n \to \infty} s_n = \sup E_s$. Each subsequential limit $L_s \in E_s$ can be written as $L_s = \lim_{k \to \infty} (a_{n_k} + b_{n_k})$ where $a_{n_k}$ and $b_{n_k}$ are subsequences of $a_n$ and $b_n$ that converge to subsequential limits $L_a$ and $L_b$ respectively. Since $\lim_{k \to \infty} (a_{n_k} + b_{n_k}) = L_a + L_b$, every subsequential limit $L_s$ of $s_n$ satisfies $L_s = L_a + L_b \le \sup E_a + \sup E_b$ by definition, where $E_a$ and $E_b$ are the sets of all subsequential limits of $a_n$ and $b_n$. It follows that $$\limsup_{n \to \infty} (a_n + b_n) = \sup E_s \le \sup E_a + \sup E_b = \limsup_{n \to \infty} a_n + \limsup_{n \to \infty} b_n$$
    \end{proof}
\end{exercise}

\begin{exercise}{16}
    \begin{solution}
        \begin{enumerate} [(a)]
        \item \begin{proof}
            We want to show that $x_{n+1} = \frac{1}{2} (x_n + \frac{\alpha}{x_n})\le x_n$. \begin{equation*}
                \begin{split}
                    \frac{1}{2} \left( x_n + \frac{\alpha}{x_n} \right) & \le x_n \\
                    x_n + \frac{\alpha}{x_n} & \le 2 x_n \\
                    \frac{\alpha}{x_n} & \le x_n \\
                    \alpha & \le x_n^2 \\
                    \sqrt{\alpha} & \le x_n
                \end{split}
            \end{equation*}
            It suffices to show that $\sqrt{\alpha} \le x_n$ for all $n$, and we prove this ($x_n > \sqrt{\alpha}$) by induction. Base case $(n=1)$: By assumption, $x_1 > \sqrt{\alpha}$. Inductive step: We assume $x_n > \sqrt{\alpha}$ and show $x_{n+1} > \sqrt{\alpha}$. We use the AM-GM inequality with $a= x_n$ and $b = \frac{\alpha}{x_n}$. Then, we get that \begin{equation*}
                \begin{split}
                    x_{n+1} = \frac{x_n + \frac{\alpha}{x_n}}{2} \ge \sqrt{x_n \cdot \frac{\alpha}{x_n}} = \sqrt{\alpha}
                \end{split}
            \end{equation*}
            We show that this inequality is strict, since equality in AM-GM occurs if and only if $a = b$, i.e., $x_n = \frac{\alpha}{x_n}$, which implies that $x_n^2 = \alpha$, or $x_n = \sqrt{\alpha}$. However by the inductive hypothesis, $x_n > \sqrt{\alpha}$, so $x_n \neq \sqrt{\alpha}$. Thus by induction, we have that $x_{n+1} > \sqrt{\alpha}$ for all $n$. \\

            This shows that $x_{n+1} \le x_n$ for all $n$, which implies that the sequence is monotonically decreasing. \\

            Now, to find the limit, we let $\lim_{n \to \infty} x_n = L$. Then by definition, $\lim_{n \to \infty} x_{n + 1} = L$. Taking the limit on both sides of the recursion gives \begin{equation*}
                \begin{split}
                    \lim_{n \to \infty} x_{n + 1} & = \lim_{n \to \infty} \frac{1}{2} \left( x_n + \frac{\alpha}{x_n} \right) \\
                    L & = \frac{1}{2} \left( L + \frac{\alpha}{L} \right) \\
                    2L & = L + \frac{\alpha}{L} \\
                    L^2 & = \alpha \\
                    L & = \sqrt{\alpha} \\
                    \lim_{n \to \infty} x_n & = \sqrt{\alpha}
                \end{split}
            \end{equation*}
            Thus we have shown that $\{ x_n \}$ decreases monotonically and that $\lim x_n = \sqrt{\alpha}$. 
        \end{proof}
        \item \begin{proof} We want to show that given $\epsilon_n = x_n - \sqrt{\alpha}$, $\epsilon_{n+1} = \frac{\epsilon_n^2}{2x_n}$. 
            \begin{equation*}
                \begin{split}
                    \epsilon_{n+1} & = x_{n+1} - \sqrt{\alpha} \\
                    & = \frac{1}{2} \left( x_n + \frac{\alpha}{x_n} \right) - \sqrt{\alpha} \\
                    2 \epsilon_{n+1} & = x_n + \frac{\alpha}{x_n} - 2 \sqrt{\alpha} \\
                    & = (x_n - \sqrt{\alpha}) + \left( \frac{\alpha}{x_n} - \sqrt{\alpha} \right) \\
                    & = \epsilon_n + \left( \frac{\alpha}{x_n} - \sqrt{\alpha} \right) \\
                    & = \epsilon_n + \frac{\alpha - x_n \sqrt{\alpha}}{x_n} \\
                    & = \epsilon_n + \frac{\alpha - (\sqrt{\alpha} + \epsilon_n) \sqrt{\alpha}}{x_n} \\
                    & = \epsilon_n + \frac{\alpha - \alpha - \epsilon_n \sqrt{\alpha}}{x_n} \\
                    & = \epsilon_n - \frac{\epsilon_n \sqrt{\alpha}}{x_n} \\
                    & = \epsilon_n \left( 1 - \frac{\sqrt{\alpha}}{x_n} \right) \\
                    & = \epsilon_n \left( \frac{x_n - \sqrt{\alpha}}{x_n} \right) \\
                    & = \epsilon_n \left( \frac{\epsilon_n}{x_n} \right) \\
                    2 \epsilon_{n+1} & = \frac{\epsilon_n^2}{x_n} \\
                    \epsilon_{n+1} & = \frac{\epsilon_n^2}{2x_n} \\
                    & < \frac{\epsilon_n^2}{2 \sqrt{\alpha}}
                \end{split}
            \end{equation*}
            The last inequality, $\frac{\epsilon_n^2}{2x_n} < \frac{\epsilon_n^2}{2 \sqrt{\alpha}}$ holds because we proved earlier that $x_n > \sqrt{\alpha}$. \\

            Let $\beta = 2 \sqrt{\alpha}$. From the inequality $\epsilon_{n+1} < \frac{\epsilon_n^2}{\beta}$, we define $s_n = \frac{\epsilon_n}{\beta}$. Then $s_{n+1} = \frac{\epsilon_{n+1}}{\beta} < \frac{1}{\beta} \frac{\epsilon_n^2}{\beta} = s_n^2$. This establishes a recursive inequality, $s_{n+1} < s_n^2$. We can show by induction that $s_n < s_1^{2^{n-1}}$. Base case ($n=1$): $s_1 < s_1^{2^{n-1}}$. Inductive step: Assuming $s_n < s_1^{2^{n-1}}$, we have $s_{n+1} < s_n^2 < (s_1^{2^{n-1}})^2 = s_1^{2^n}$. Therefore, $s_{n+1} < s_1^{2n}$. 

            From the above result, we can multiply both sides by $\beta$ to get $\epsilon_{n+1} = \beta s_{n+1} < \beta s_1^{2n} = \beta \left( \frac{\epsilon_1}{\beta} \right) ^{2^n}$. Thus, $\epsilon_{n+1} < \beta \left( \frac{\epsilon_1}{\beta} \right) ^{2^n}$.
        \end{proof}
        \item \begin{solution}
            Given $\alpha = 3, x_1 = 2$, we get that $\epsilon_1 = x_1 - \sqrt{\alpha} = 2 - \sqrt{3} \approx 2 - 1.73205 = 0.26795$. We get $\beta = 2 \sqrt{3} \approx 2 \cdot 1.73205 = 3.4641$. Then $\frac{\epsilon_1}{\beta} \approx \frac{0.26795}{3.4641} \approx 0.07735 < \frac{1}{10}$. Thus, $\frac{\epsilon_1}{\beta} < \frac{1}{10}$ holds. 

            Then, for $\epsilon_5$, we have 
            \begin{equation*}
                \begin{split}
                    \epsilon_5 & < \beta \left( \frac{\epsilon_1}{\beta} \right)^{2^4}
                \end{split}
            \end{equation*}
            This can be estimated as 
            \begin{equation*}
                \begin{split}
                    \epsilon_5 & < \beta \left( \frac{1}{10} \right) ^ {16} \\
                    & = \beta \cdot 10^{-16} \\
                    & < 4 \cdot 10^{-16}
                \end{split}
            \end{equation*}
            Since $\beta \approx 3.4641 < 4$, the last inequality holds. Similarly for $\epsilon_6$, we have \begin{equation*}
                \begin{split}
                    \epsilon_6 & < \beta \left( \frac{1}{10} \right) ^{32} \\
                    & = \beta \cdot 10^{-32} \\
                    & < 4 \cdot 10^{-32}
                \end{split}
            \end{equation*}
        \end{solution}
    \end{enumerate}
    \end{solution}
\end{exercise}

\begin{exercise}{20}
    \begin{proof}
        Given $\epsilon > 0$, since $p_n$ is Cauchy, there exists $N_1 \in \bbN$ such that for all $n, m \ge N_1$, $d(p_n, p_m) < \frac{\epsilon}{2}$. Since $p_{n_k}$ converges to $p$, there exists $K \in \bbN$ such that for all $k \ge K$, $d(p_{n_k}, p) < \frac{\epsilon}{2}$. Let $N_2 = n_K$, which is the index in the original sequence corresponding to the $K$-th term of the subsequence. Define $N = \max \{ N_1, N_2 \}$. Then for all $n \ge N$, $d(p_n, p) \le p(p_n, p_{n_K}) + d(p_{n_K}, p) < \frac{\epsilon}{2} + \frac{\epsilon}{2} = \epsilon$. Thus, $p_n \to p$.  
    \end{proof}
\end{exercise}

\begin{exercise}{21}
    \begin{proof}
        For each $n \in \bbN$, since $E_n$ is nonempty, select a point $x_n \in E_n$. Since $E_n \supset E_{n+1}$, we have $x_{n+1} \in E_{n+1} \subset E_n$. In other words, $x_n$ and $x_{n+1}$ both belong to $E_n$. Since $\lim \diam E_n = 0$, for any $\epsilon > 0$, there exists $N \in \bbN$ such that for all $n \ge N$, $\diam E_n < \epsilon$. Then, for all $m, n \ge N$ with $ m \ge n$, since $x_m \in E_m \subset E_n$, we have that $d(x_n, x_m) \le \diam E_n < \epsilon$. Thus, the sequence $x_n$ we have constructed is Cauchy. Since $X$ is complete, every Cauchy sequence in $X$ converges, so $x_n$ converges to some point $x \in X$. It must be that $x \in E_n$ for all $n$ because since $x_m \in E_n$ for all $m \ge n$, and since $x_m$ converges to $x$, it follows that $x$ is a limit point of $E_n$. Since $E_n$ is closed, it contains all of its limit points, so $x \in E_n$ for all $n$. Then, since $x$ belongs to every $E_n$, we have that $x \in \bigcap_{n=1}^\infty E_n$. We now prove that $x$ is the only point in this intersection. Suppose not. Assume for the sake of contradiction that there exists another point $y \in \bigcap_{n=1}^\infty E_n$ such that $y \neq x$. Since $x \neq y$, $d(x, y) > 0$. Given that $\lim \diam E_n = 0$, for $\epsilon = \frac{1}{2} d(x, y) > 0$, there exists $N$ such that for all $n \ge N$, $\diam E_n < \epsilon$. Since $x, y \in \bigcap_{n=1}^\infty E_n$, they are both in $E_N$. However, because $\diam E_N < \epsilon$, we have that $d(x, y) < \epsilon = \frac{1}{2} d(x, y)$ which is a contradiction. Thus, $\bigcap_{n=1}^\infty$ contains exactly one point. 
    \end{proof}
\end{exercise}

\begin{exercise}{22}
    \begin{proof}
        Let $U_1$ be an arbitrary nonempty open subset of $X$. We want to find a point $x \in U_1$ such that $x \in \bigcap_{n=1}^\infty G_n$. We will construct a sequence of closed balls $\{ \overline{E_n} \}$ such that $\overline{E_n} \subset G_n$ for all $n$, $\overline{E_{n+1}} \subset \overline{E_n}$ for all $n$, and the diameters of $\overline{E_n}$ tend to zero as $n \to \infty$. We construct inductively as follows: Base case ($n=1$): Since $G_1$ is dense and open, the intersection $G_1 \cap U_1$ is nonempty and open. Choose a closed ball $\overline{E_1}$ with radius $r_1 > 0$ such that $\overline{E_1} \subset G_1 \cap U_1$. This is possible because $G_1 \cap U_1$ is open, so it contains some open ball, and we can take its closure to get $\overline{E_1}$. For $n \ge 1$, suppose we have constructed $\overline{E_n} \subset G_n$ with radius $r_n > 0$. Since $G_{n+1}$ is dense and open, and $\interior{(\overline{E_n})}$ the interior of $\overline{E_n}$ is nonempty and open, their intersection is nonempty and open, i.e., $G_{n+1} \cap \interior{\overline{E_n}} \neq \emptyset$. Choose a closed ball $\overline{E_{n+1}}$ with radius $r_{n+1} > 0$ such that $\overline{E_{n+1}} \subset G_{n+1} \cap \interior{\overline{E_n}}$ and $r_{n+1} < \frac{1}{2} r_n$. This ensures that $\overline{E_{n+1}} \subset \overline{E_n}$ and the diameters satisfy $\diam \overline{E_{n+1}} = 2r_{n+1} < r_n = \frac{1}{2} \diam \overline{E_n}$. 

        By this construction, the sequence $\{ E_n \}$ is a nested sequence of nonempty, closed, and bounded subsets, with the diameters tending to zero. By the previous exercise, we know that the intersections of these $E_n$'s consists of exactly one point, i.e., $\bigcap_{n=1}^\infty \overline{E_n} = \{ x \}$ for some $x \in X$. For each $n$, $x \in \overline{E_n} \subset G_n$, since $\overline{E_n} \subset G_n$. Thus $x \in G_n$ for all $n$, so $x \in \bigcap_{n=1}^\infty G_n$. Since $\overline{E_1} \subset U_1$ and $x \in \overline{E_1}$, it follows that $x \in U_1$. 

        Thus, we have shown that every nonempty open subset $U_1$ of $X$ contains a point $x \in \bigcap_{n=1}^\infty G_n$. Therefore, $\bigcap_{n=1}^\infty G_n$ is dense in $X$, and in particular, $\bigcap_{n=1}^\infty G_n$ is nonempty. 
    \end{proof}
\end{exercise}

\begin{exercise}{23}
    \begin{proof}
        Using the triangle inequality in $X$, we see that \begin{itemize}
        \item $d(p_n, q_n) \le d(p_n, p_m) + d(p_m, q_m) + d(q_m, q_n)$
        \item $d(p_m, q_m) \le d(p_m, p_n) + d(p_n, q_n) + d(q_n, q_m)$
    \end{itemize}
    Subtracting each inequality from the other, we get \begin{itemize}
        \item $d(p_n, q_n) - d(p_m, q_m) \le d(p_n, p_m) + d(q_m, q_n)$
        \item $d(p_m, q_m) - d(p_n, q_n) \le d(p_n, p_m) + d(q_n, q_m)$
    \end{itemize}
    Thus, we see that $|d(p_n, q_n) - d(p_m, q_m)| \le d(p_n, p_m) + d(q_n, q_m)$. Since $p_n$ and $q_n$ are both Cauchy, there exists $N \in \bbN$ such that for all $m, n \ge N$, $d(p_n, p_m) < \frac{\epsilon}{2}$ and $d(q_n, q_m) < \frac{\epsilon}{2}$. Then for all $n, m \ge N$, $$|d(p_n, q_n) - d(p_m, q_m)| \le d(p_n, p_m) + d(q_n, q_m) < \frac{\epsilon}{2} + \frac{\epsilon}{2} = \epsilon$$
    Thus, since each $d(p_n, q_n)$ is in $\bbR$, and the sequence $\{ d(p_n, q_n) \}$ is Cauchy, the sequence converges by completeness of $\bbR$. 
    \end{proof}
\end{exercise}

\begin{exercise}{24}
    \begin{enumerate} [(a)]
        \item \begin{proof}
            \begin{itemize}
                \item Reflexivity: For all $n \in \bbN$, $d(p_n, p_n) = 0$, therefore $\lim_{n \to \infty} d(p_n, p_n) = 0$. Hence $p_n \sim p_n$. 
                \item Symmetry: Assume $p_n \sim q_n$, so $\lim_{n \to \infty} d(p_n, q_n) = 0$. Since the metric $d$ is symmetric, $d(p_n, q_n) = d(q_n, p_n)$ for all $n$. Therefore, $\lim_{n \to \infty} d(q_n, p_n) = \lim_{n \to \infty} d(p_n, q_n) = 0$. Hence, $q_n \sim p_n$. 
                \item Transitivity: Assume $p_n \sim q_n$ and $q_n \sim r_n$. Then by the triangle inequality, $d(p_n, r_n) \le d(p_n, q_n) + d(q_n, r_n)$. Taking limits on both sides, we get $$\lim_{n \to \infty} d(p_n, r_n) \le \lim_{n \to \infty} d(p_n, q_n) + \lim_{n \to \infty} d(q_n, r_n) = 0$$ Since $d(p_n, r_n) \ge 0$ for all $n$, it follows that $0 \le \lim_{n \to \infty} d(p_n, r_n) \le 0$ which implies that $\lim_{n \to \infty} d(p_n, r_n) = 0$. Thus, $p_n \sim r_n$. 
            \end{itemize}
        \end{proof}
        \item \begin{proof}
            We want to show that if $p_n \sim p_n'$ and $q_n \sim q_n'$ then $\lim_{n \to \infty} d(p_n, q_n) = \lim_{n \to \infty} d(p_n', q_n')$. Using the triangle inequality, we have $|d(p_n, q_n) - d(p_n', q_n')| \le d(p_n, p_n') + d(q_n, q_n')$. Taking limits on both sides, we have $$\lim_{n \to \infty} |d(p_n, q_n) - d(p_n', q_n')| \le \lim_{n \to \infty} d(p_n, p_n') + \lim_{n \to \infty} d(q_n, q_n') = 0$$ This implies $\lim_{n \to \infty} d(p_n, q_n) = \lim_{n \to \infty} d(p_n', q_n')$. 

            We now prove that $\Delta$ is a metric on $X^*$.
            \begin{itemize}
                \item Non-negativity: $\Delta (P, Q) = \lim_{n \to \infty} d(p_n, q_n) \ge 0$ holds because $d(p_n, q_n) \ge 0$ for all $n$. 
                \item If $\Delta (P, Q) = 0$ then $P = Q$: If $\Delta (P, Q) = 0$, then $\lim_{n \to \infty} d(p_n, q_n) = 0$. By definition, $p_n \sim q_n$ which implies $P = Q$. We also show that if $P = Q$ then $\Delta (P, Q) = 0$: Since $P = Q$, $p_n \sim q_n$ which implies $\lim_{n \to \infty} d(p_n, q_n) = 0$. 
                \item Symmetry: $\Delta (P, Q) = \lim_{n \to \infty} d(p_n, q_n) = \lim_{n \to \infty} d(q_n, p_n) = \Delta (Q, P)$.
                \item Triangle Inequality: We want to show that for any $P, Q, R \in X^*$, $\Delta (P, R) \le \Delta (P, Q) + \Delta (Q, R)$. By the triangle inequality in $X$, we have $d(p_n, r_n) \le d(p_n, q_n) + d(q_n, r_n)$. Taking limits on both sides, $\lim_{n \to \infty} d(p_n, r_n) \le \lim_{n \to \infty} d(p_n, q_n) + \lim_{n \to \infty} d(q_n, r_n)$. By substituting definitions, we get what we wanted to show, $\Delta (P, R) \le \Delta (P, Q) + \Delta (Q, R)$.
            \end{itemize}
        \end{proof}
        \item Let $\{ P_m \}$ be a Cauchy sequence in $X^*$, where each $P_m$ is an equivalence class of Cauchy sequences in $X$. For each $m$, select a representative Cauchy sequence $\{ p_n^{(m)} \}$ from $P_m$. Since $\{ P_m \}$ is Cauchy in $X^*$, for any $\epsilon > 0$, there exists $M$ such that for all $m, l \ge M$,  we have $\Delta (P_m, P_l) = \lim_{k \to \infty} d(p_k^{(m)}, p_k^{(l)} ) < \epsilon$. For a fixed $n$, consider the sequence $\{ p_n^{(m)} \}_{m = M}^\infty$. For $m, l \ge M$, $d(p_n^{(m)}, p_n^{(l)}) \le \limsup_{k \to \infty} d(p_k^{(m)}, p_k^{(l)}) < \epsilon$. Thus, $\{ p_n^{(m)} \}_{m = M}^\infty$ is a Cauchy sequence in $X$. Although $\{ p_n^{(m)} \}$ may not converge in $X$, their equivalence classes in $X^*$ do exist. Define $q_n \in X^*$ as the equivalence class represented b the Cauchy sequence $\{ p_n^{(m)} \}_{m = M}^\infty$. We need to show that $\{ q_n \} $ is a Cauchy sequence in $X^*$. For $n, l \ge N$, $\Delta (q_n, q_l) = \lim_{m \to \infty} d(p_n^{(m)}, p_l^{(m)} ) \le \lim_{m \to \infty} d(p_n^{(m)}, p_n^{(l)}) + \lim_{m \to \infty} d(p_n^{(l)}, p_l^{(l)}) < 2 \epsilon$. Thus, $\{ q_n \}$ is Cauchy in $X^*$. Since $X^*$ includes all equivalence classes of Cauchy sequences, there exists $P \in X^*$ represented by $\{ q_n \}$. We need to show that $\Delta (P_m, P) \to 0$ as $m \to \infty$. Note that $\Delta (P_m, P) = \lim_{n \to \infty} d(p_n^{(m)}, q_n) = \lim_{n \to \infty} (\lim_{l \to \infty} d(p_n^{(m)}, p_n^{(l)}) )$. Since $\{ p_n^{(m)} \}_m$ is Cauchy in $X$ for fixed $n$, and $p_n^{(l)}$ approaches $q_n$, it follows that $\Delta (P_m, P) \to 0$. Thus, every Cauchy sequence $\{ P_m \}$ in $X^*$ converges to some $P \in X^*$, hence $(X^*, \Delta)$ is copmlete. 
    \end{enumerate}
\end{exercise}

\begin{exercise}{B}
    \begin{proof}
        Assume for the sake of contradiction that $X$ is countable. Then we can enumerate its elements as $X = \{ x_1, x_2, \dots \}$. Since $X$ has no isolated points, for each $x_n$ and for any $\epsilon > 0$, there exists another point $y \in X$ such that $0 < d(x_n, y) < \epsilon$. Define $G_n = X \setminus \{ x_n \}$ for each $n$. Then $G_n$ is open, since $\{ x_n \}$ is a singleton and is closed. Additionally, $G_n$ is dense in $X$. For any nonempty open subset $U \subset X$, $U \cap G_n$ is nonempty because $U$ cannot be entirely contained in $\{ x_n \} $ since $x_n$ is not an isolated point. Let us consider the intersection of all $G_n$, $$\bigcap_{n = 1}^\infty G_n = X \setminus \bigcup_{n = 1}^\infty \{ x_n \} = X \setminus X = \emptyset$$ This means that $\bigcap_{n=1}^\infty G_n$ is empty. However, according to exercise 22, the intersection of countably many dense open subsets in a complete metric space should be nonempty. Thus, we have a contradiction and $X$ must be countable. 
    \end{proof}
\end{exercise}

\begin{exercise}{C}
    \begin{proof}
        Let $P \subset X$ be a perfect subset of $X$. Since $P$ is a closed subset of the complete metric space, it follows that $P$ is complete. By the definition of a perfect set, $P$ has no isolated points; every point is a limit point. Applying exercise B, since $P$ is nonempty, complete, and has no isolated points, $P$ is uncountable. 
    \end{proof}
\end{exercise}

\end{document}