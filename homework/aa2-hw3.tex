\documentclass[11pt]{article}
\usepackage{master}
\DeclareMathOperator{\diam}{diam}
\newcommand{\interior}[1]{%
  {\kern0pt#1}^{\mathrm{o}}%
}

\title{Accelerated Analysis HW3}
\author{Andrew Hah}
\begin{document}

\pagestyle{plain}
\begin{center}
{\Large MATH 20410. Accelerated Analysis II Homework 3} \\ 
\vspace{.2in}  
Andrew Hah \\
Due January 29, 2025
\end{center}

\begin{exercise}{9.6}
    \begin{proof}
        For $(x,y)\neq (0,0)$, we can compute the partial 
derivatives directly (using the quotient rule):

\[
\frac{\partial f}{\partial x}(x,y) 
= \frac{(x^2 + y^2)\,y \;-\; x\,y\,(2x)}{(x^2 + y^2)^2} 
= \frac{y\bigl(y^2 - x^2\bigr)}{(x^2 + y^2)^2},
\]
\[
\frac{\partial f}{\partial y}(x,y) 
= \frac{(x^2 + y^2)\,x \;-\; x\,y\,(2y)}{(x^2 + y^2)^2} 
= \frac{x\bigl(x^2 - y^2\bigr)}{(x^2 + y^2)^2}.
\]
These expressions are well-defined for all $(x,y)\neq (0,0)$.

At $(0,0)$. By definition of the partial derivative,
\[
\frac{\partial f}{\partial x}(0,0)
= \lim_{h \to 0} \frac{f(h,0) - f(0,0)}{h}
= \lim_{h \to 0} \frac{0 - 0}{h} 
= 0.
\]
Similarly,
\[
\frac{\partial f}{\partial y}(0,0)
= \lim_{k \to 0} \frac{f(0,k) - f(0,0)}{k}
= \lim_{k \to 0} \frac{0 - 0}{k}
= 0.
\]
Hence both partial derivatives exist at the origin and are equal to $0$.

To see that $f$ is not continuous at $(0,0)$, observe that
\[
\lim_{t \to 0} f(t,t) 
= \lim_{t \to 0} \frac{\,t \cdot t\,}{t^2 + t^2}
= \lim_{t \to 0} \frac{t^2}{2t^2}
= \frac12,
\]
which differs from $f(0,0) = 0$.  Thus $f$ is not continuous at $(0,0)$. \\

Thus, all partial derivatives exist everywhere in $\mathbb{R}^2$, 
yet $f$ itself fails to be continuous at $(0,0)$.
    \end{proof}
\end{exercise}

\begin{exercise}{9.8}
    \begin{proof}
        Since $f$ has a local maximum at $\mathbf{x}$ and $E$ is open, there exists 
an $\epsilon>0$ such that
\(
\|\mathbf{h}\| < \epsilon
\quad\Longrightarrow\quad
\mathbf{x} + \mathbf{h} \in E
\quad\text{and}\quad
f(\mathbf{x}+\mathbf{h}) \le f(\mathbf{x}).
\)
Define $L = f'(\mathbf{x})$, 
which is a linear functional on $\mathbb{R}^n$. For any nonzero direction $\mathbf{v} \in \mathbb{R}^n$, consider the function 
\[
\phi(t) \;=\; f(\mathbf{x} + t\,\mathbf{v})
\]
for $t$ in some small interval about $0$.  Since $\mathbf{x}$ is a local maximum of $f$, 
$\phi(t)$ attains a local maximum at $t=0$.  Therefore,
\(
\phi'(0) \;=\; 0.
\)
On the other hand, by definition of the derivative of $f$, we have
\[
\phi'(0)
\;=\;
\lim_{t\to0} \frac{f(\mathbf{x} + t\,\mathbf{v}) - f(\mathbf{x})}{t}
\;=\;
L(\mathbf{v}),
\]
where $L = f'(\mathbf{x})$.  Hence
\(
L(\mathbf{v}) \;=\; 0
\quad\text{for all}\quad \mathbf{v}\in \mathbb{R}^n.
\)
A linear functional $L$ on $\mathbb{R}^n$ that is zero on all $\mathbf{v}\in\mathbb{R}^n$ 
must be the zero map.  Consequently, 
\(
L \;=\; f'(\mathbf{x}) \;=\; 0.
\)
    \end{proof}
\end{exercise}

\begin{exercise}{9.11}
    \begin{proof}
        Let $\mathbf{x} = (x_1, \dots, x_n) \in \mathbb{R}^n$.  By definition,
\[
\nabla(fg)(\mathbf{x})
\;=\;
\biggl(\,
\frac{\partial (fg)}{\partial x_1}(\mathbf{x}), 
\;\dots,\;
\frac{\partial (fg)}{\partial x_n}(\mathbf{x})
\biggr).
\]
Since $f$ and $g$ are (separately) differentiable, for each $i = 1,\dots,n$, we use 
the single‐variable product rule on the function $x_i \mapsto f(x_1, \dots, x_i, \dots, x_n)\,g(x_1, \dots, x_i, \dots, x_n)$:
\[
\frac{\partial (fg)}{\partial x_i}(\mathbf{x}) 
\;=\;
f(\mathbf{x}) \;\frac{\partial g}{\partial x_i}(\mathbf{x})
\;+\;
g(\mathbf{x}) \;\frac{\partial f}{\partial x_i}(\mathbf{x}).
\]
Consequently,
\[
\nabla(fg)(\mathbf{x})
\;=\;
\bigl(\,
f \,\tfrac{\partial g}{\partial x_1} + g\,\tfrac{\partial f}{\partial x_1}, 
\;\dots,\;
f\,\tfrac{\partial g}{\partial x_n} + g\,\tfrac{\partial f}{\partial x_n}
\bigr).
\]
Factoring $f$ and $g$ out of each coordinate, we get
\[
\nabla(fg)(\mathbf{x})
= 
f(\mathbf{x})\,
\bigl(\tfrac{\partial g}{\partial x_1}, \dots, \tfrac{\partial g}{\partial x_n}\bigr)
\;+\;
g(\mathbf{x})\,
\bigl(\tfrac{\partial f}{\partial x_1}, \dots, \tfrac{\partial f}{\partial x_n}\bigr).
\]
That is,
\[
\nabla(fg)(\mathbf{x}) 
\;=\; 
f(\mathbf{x})\,\nabla g(\mathbf{x})
\;+\;
g(\mathbf{x})\,\nabla f(\mathbf{x}).
\]
Since $\mathbf{x}$ was arbitrary, this identity holds everywhere in $\mathbb{R}^n$. 
\quad$\square$
    \end{proof}
\end{exercise}

\begin{exercise}{9.13}
    \begin{proof}
        Since $\lVert f(t)\rVert = 1$ for all $t$, we have
\[
f(t)\cdot f(t) \;=\; \lVert f(t)\rVert^2 \;=\; 1
\quad\text{for all }t.
\]
Differentiate both sides with respect to $t$:
\[
\frac{d}{dt}\bigl[f(t)\cdot f(t)\bigr] 
\;=\;
\frac{d}{dt}(1)
\;=\;
0.
\]
By the product rule for dot products,
\[
\frac{d}{dt}\bigl[f(t)\cdot f(t)\bigr]
\;=\;
f'(t)\cdot f(t) \;+\; f(t)\cdot f'(t)
\;=\;
2\,\bigl(f'(t)\cdot f(t)\bigr).
\]
Hence
\[
2\,\bigl(f'(t)\cdot f(t)\bigr) 
\;=\;
0
\quad\Longrightarrow\quad
f'(t)\cdot f(t)
\;=\;
0.
\]

Since $f(t)$ always lies on the unit sphere (because $\lVert f(t)\rVert=1$), 
the vector $f'(t)$ must be tangent to the sphere at $f(t)$. 
This is equivalent to saying $f'(t)$ is orthogonal to the radius vector $f(t)$, 
so their dot product is zero.
    \end{proof}
\end{exercise}

\begin{exercise}{1}
    \begin{proof}
Write $f(x,y)=(u_1,u_2,u_3)$ where 
\[
u_1(x,y)=x^2,\quad 
u_2(x,y)=x+y,\quad
u_3(x,y)=x\,y.
\]
Then $Df(x,y)$ is 
\[
Df(x,y)
\;=\;
\begin{pmatrix}
\partial u_1/\partial x & \partial u_1/\partial y\\
\partial u_2/\partial x & \partial u_2/\partial y\\
\partial u_3/\partial x & \partial u_3/\partial y
\end{pmatrix}
\;=\;
\begin{pmatrix}
2x & 0\\[6pt]
1 & 1\\[6pt]
y & x
\end{pmatrix}.
\]
Hence
\[
Df(a,b)
\;=\;
\begin{pmatrix}
2\,a & 0\\[4pt]
1 & 1\\[4pt]
b & a
\end{pmatrix}.
\]

\bigskip
\noindent
Similarly, write $g(x,y,z) = (v_1,v_2)$ where
\[
v_1(x,y,z)=x,\qquad 
v_2(x,y,z)=x+y+2\,z.
\]
Then
\[
Dg(x,y,z)
\;=\;
\begin{pmatrix}
\partial v_1/\partial x & \partial v_1/\partial y & \partial v_1/\partial z\\
\partial v_2/\partial x & \partial v_2/\partial y & \partial v_2/\partial z
\end{pmatrix}
\;=\;
\begin{pmatrix}
1 & 0 & 0\\[4pt]
1 & 1 & 2
\end{pmatrix}.
\]
Note that these partial derivatives are all constant, so 
\[
Dg\bigl(f(a,b)\bigr)
\;=\;
Dg\bigl(a^2,\;a+b,\;a\,b\bigr)
\;=\;
\begin{pmatrix}
1 & 0 & 0\\[4pt]
1 & 1 & 2
\end{pmatrix}.
\]

By the chain rule, we have
\[
D\bigl(g\circ f\bigr)(a,b)
\;=\;
Dg\bigl(f(a,b)\bigr)\;\cdot\;Df(a,b).
\]
Hence
\[
D\bigl(g\circ f\bigr)(a,b)
\;=\;
\begin{pmatrix}
1 & 0 & 0\\[4pt]
1 & 1 & 2
\end{pmatrix}
\begin{pmatrix}
2\,a & 0\\[4pt]
1 & 1\\[4pt]
b & a
\end{pmatrix}
\;=\;
\begin{pmatrix}
2\,a & 0\\[6pt]
2\,a + 1 + 2\,b & 1 + 2\,a
\end{pmatrix}.
\]
    \end{proof}
\end{exercise}

\begin{exercise}{2}
    \begin{proof}
 ($\implies$) If $f$ is differentiable, each $f_j$ is differentiable.\\
Suppose $f$ is differentiable at $x$, and let $A= Df(x)$.  Consequently,
\[
f(x+\mathbf{h}) \;-\; f(x)\;-\; A(\mathbf{h})
\;=\;
\bigl(\,
f_1(x+\mathbf{h})-f_1(x)-A_1(\mathbf{h}),\;\dots,\;
f_m(x+\mathbf{h})-f_m(x)-A_m(\mathbf{h})
\bigr).
\]
Hence 
\[
\bigl\|\,f(x+\mathbf{h}) \;-\; f(x)\;-\; A(\mathbf{h})\bigr\|_{\mathbb{R}^m}^2
\;=\;
\sum_{j=1}^m
\bigl[\,
f_j(x+\mathbf{h})-f_j(x)-A_j(\mathbf{h})
\bigr]^2.
\]
If 
\(\lim_{\mathbf{h}\to \mathbf{0}}\|\,f(x+\mathbf{h}) - f(x) - A(\mathbf{h})\|/\|\mathbf{h}\|=0,\)
then each term in the above sum must vanish ``fast enough'', i.e.
\[
\lim_{\mathbf{h}\to\mathbf{0}}
\frac{
\bigl|\,
f_j(x+\mathbf{h})-f_j(x)-A_j(\mathbf{h})
\bigr|
}{\|\mathbf{h}\|}
\;=\;
0
\quad\text{for each }j.
\]
But this is precisely the statement that $f_j$ is differentiable at $x$ 
with derivative $A_j$, i.e.\ $df_j(x)(\mathbf{h}) = A_j(\mathbf{h})$.  
Thus each $f_j$ is differentiable at $x$.

($\impliedby$) If each $f_j$ is differentiable, then $f$ is differentiable.\\
Conversely, suppose that each component $f_j$ is differentiable at $x$.  
This means for each $j=1,\dots,m$, there is a linear functional 
$L_j:\mathbb{R}^n \to \mathbb{R}$ such that
\[
\lim_{\mathbf{h}\to\mathbf{0}}
\frac{\bigl|\,f_j(x+\mathbf{h})-f_j(x)-L_j(\mathbf{h})\bigr|}
{\|\mathbf{h}\|}
\;=\;0.
\]
Define $A:\mathbb{R}^n\to\mathbb{R}^m$ by
\[
A(\mathbf{h})
\;=\;
\bigl(\,L_1(\mathbf{h}),\,L_2(\mathbf{h}),\,\ldots,\,L_m(\mathbf{h})\bigr).
\]
This $A$ is linear (componentwise) and will serve as our candidate for $Df(x)$.  
Indeed,
\[
f(x+\mathbf{h}) \;-\; f(x)\;-\; A(\mathbf{h})
\;=\;
\bigl(\,
f_1(x+\mathbf{h})-f_1(x)-L_1(\mathbf{h}),\;\dots,\;
f_m(x+\mathbf{h})-f_m(x)-L_m(\mathbf{h})
\bigr).
\]
Again, by summing squares or using standard vector norms, we see that 
\[
\lim_{\mathbf{h}\to\mathbf{0}}
\frac{\|\,f(x+\mathbf{h}) - f(x) - A(\mathbf{h})\|}{\|\mathbf{h}\|}
\;=\;
0,
\]
because each component tends to zero at the required rate.  
Hence $f$ is differentiable at $x$ and $Df(x)=A$.
    \end{proof}
\end{exercise}

\begin{exercise}{3}
    \begin{proof}
        \begin{enumerate} [(1)]
            \item $f(x) = \|x\|^2$. \\

            Let $x = (x_1,x_2,\ldots,x_n)\in\mathbb{R}^n$.  Then
\[
  f(x) \;=\; \|x\|^2 \;=\; x_1^2 + x_2^2 + \cdots + x_n^2.
\]
This is a polynomial in the coordinates $x_1,\ldots,x_n$, so it is differentiable everywhere in $\mathbb{R}^n$.  In fact, we can compute the gradient directly:
\[
  \nabla f(x)
  \;=\;
  \begin{pmatrix}
    \frac{\partial}{\partial x_1} (x_1^2 + \cdots + x_n^2)\\[6pt]
    \vdots\\[6pt]
    \frac{\partial}{\partial x_n} (x_1^2 + \cdots + x_n^2)
  \end{pmatrix}
  \;=\;
  \begin{pmatrix}
    2x_1\\[6pt]
    \vdots\\[6pt]
    2x_n
  \end{pmatrix}
  \;=\;
  2x.
\]
Hence $f$ is differentiable at every point $x \in \mathbb{R}^n$.
    \item $f(x_1,\ldots,x_n,y_1,\ldots,y_n) = x_1y_1 + \cdots + x_ny_n$. \\

        This function is a polynomial (bilinear) in the $2n$ variables 
\[
  x_1,\ldots,x_n,y_1,\ldots,y_n.
\]
Since polynomials are differentiable everywhere, it follows that $f$ is differentiable at all points of $\mathbb{R}^{2n}$.
        \end{enumerate}
    \end{proof}
\end{exercise}

\end{document}